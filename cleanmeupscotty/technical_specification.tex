\documentclass[DIV=calc, paper=a4, fontsize=11pt]{scrartcl} % Документ принадлежит классу article, а также будет печататься в 12 пунктов.
\usepackage{ucs}
\usepackage[T1,T2A]{fontenc}
\usepackage[utf8x]{inputenc} % Включаем поддержку UTF8
\usepackage[russian]{babel} % Пакет поддержки русского языка
\usepackage{titling} % Allows custom title configuration

%for frames
\usepackage{framed}

%For image using
\usepackage{graphicx}

%Numbering subsubsubsections etc
\setcounter{secnumdepth}{5}

%For code highlightning
\usepackage{listings}

%Further enumeration
\usepackage{enumitem}
\setenumerate[1]{label=\theparagraph.\arabic*.}
\setenumerate[2]{label*=\arabic*.}
\setenumerate[3]{label*=\arabic*.}


%For referencing within enumeration lists
\usepackage{enumitem}

%Packages for word-like comment style
\usepackage{todonotes}

%Package for images
\usepackage{float}
\floatstyle{boxed}
\restylefloat{figure}

%For a nicer reference
\usepackage{fancyref}

\usepackage{titlesec}

\usepackage{xcolor}

% For nice JSON output 
\colorlet{punct}{red!60!black}
\definecolor{background}{HTML}{EEEEEE}
\definecolor{delim}{RGB}{20,105,176}
\colorlet{numb}{magenta!60!black}

\lstdefinelanguage{json}{
    basicstyle=\normalfont\ttfamily,
    numbers=left,
    numberstyle=\scriptsize,
    stepnumber=1,
    numbersep=8pt,
    showstringspaces=false,
    breaklines=true,
    frame=lines,
    backgroundcolor=\color{background},
    literate=
     *{0}{{{\color{numb}0}}}{1}
      {1}{{{\color{numb}1}}}{1}
      {2}{{{\color{numb}2}}}{1}
      {3}{{{\color{numb}3}}}{1}
      {4}{{{\color{numb}4}}}{1}
      {5}{{{\color{numb}5}}}{1}
      {6}{{{\color{numb}6}}}{1}
      {7}{{{\color{numb}7}}}{1}
      {8}{{{\color{numb}8}}}{1}
      {9}{{{\color{numb}9}}}{1}
      {:}{{{\color{punct}{:}}}}{1}
      {,}{{{\color{punct}{,}}}}{1}
      {\{}{{{\color{delim}{\{}}}}{1}
      {\}}{{{\color{delim}{\}}}}}{1}
      {[}{{{\color{delim}{[}}}}{1}
      {]}{{{\color{delim}{]}}}}{1},
}


\titleformat*{\section}{\LARGE\bfseries}
\titleformat*{\subsection}{\Large\bfseries}
\titleformat*{\subsubsection}{\large\bfseries}
\titleformat*{\paragraph}{\large\bfseries}
\titleformat*{\subparagraph}{\large\bfseries}

%For some math formulas if needed
\usepackage{mathtools}

% Some nice visualization
%\usepackage[svgnames]{xcolor} % Enabling colors by their 'svgnames'
\usepackage{fullpage}
%\renewcommand{\headrulewidth}{0.0pt} % No header rule
%\renewcommand{\footrulewidth}{0.4pt} % Thin footer rule
% End visualization

%smart enumeration
\renewcommand{\labelenumi}{\arabic{enumi}.}
\renewcommand{\labelenumii}{\arabic{enumi}.\arabic{enumii}}


% General document settings
% % %
% Accepted file upload formats
\newcommand{\AcceptedFormats}{.txt, .csv, .xls, .xlsx}

\title{ТЗ на ЛК и мобильное приложение для химчистки}

\date{01/04/2016}

\begin{document}

\maketitle

\section{Описание проекта}
Рабочее название проекта - cleanmeupscotty. Состоит из двух частей - личного кабинета пользователя (представляет собой веб-сайт на Битриксе) и двух мобильных приложений (или одним с разными доступами и разным функционалом).
\\subsection{Личный кабинет (сайт)}
Суть заключается в следующем - в личный кабинет (ЛК) на сайт поступает информация от сервера, который обрабатывает данные ПО химчистки. Например о статусе определенной вещи. Также видна некая статистика - сколько вещей было сдано, сколько денег потрачено, сколько бонусных баллов заработано.
\subsection{Мобильное приложение для клиента}
Дублирует основной функционал личного кабинета (выводит данные сервера химчистки). Дополнительно позволяет вызывать водителя к текущему местоположению для того, что бы он забрал вещи.
\subsection{Мобильное приложение для водителя}
Позволяет принимать заказ от клиента. Отправляет данные на сервер химчистки.


\section{Схема работы}

	\begin{figure}[H]
        \centering
        \includegraphics[width=480px]{scheme.png}
        \caption{Схема связей в общей системе\label{fig:scheme.png}}
    \end{figure}
    
Из точек приема 1...N поступают данные о заказах на общий сервер. Сервер отправляет данные также и обратно (эта сторона не касается проекта, данный обмен налажен на стороне ПО химчистки).
\\[0.5cm]
У сервера есть интерфейс приложения (application interface, API), с помощью которого можно запрашивать данные с сервера. Эти данные показываются клиенту в его личном кабинете (на сайте или в мобильном приложении).
\\[0.5cm]
Заказывая сбор вещей на дом, клиент вызывает водителя, который забирает грязные вещи. Водитель через свое приложение отмечает что он их принял. \todo[inline]{Нужно ли? Водитель разве не просто отдает вещи в одной из точек приема? Тогда нижняя стрелка не нужна}


\section{Параметры}


\subsection{Личный кабинет}

\subsubsection{Запрос}
К серверу химчистки отправляется GET-запрос, который содержит в себе уникальный идентификатор клиента.
Пример запроса: 
\begin{displaymath}
	http://api.cleanmeupscotty.ru/params?clientid=7817236178
\end{displaymath}	

\subsubsection{Ответ}
От сервера химчистки приходят данные в формате XML или JSON. Перечень данных:
\begin{itemize}
	\item Кол-во вещей в обработке
	\item Накопленные баллы
	\item Общее кол-во сданных вещей
	\item Перечень вещей в обработке (для каждой вещи получаем уникальный айди от химчистки, наименование, статус по химчистке, цену на 
	\item Общий перечень всего сданных вещей
	\item Статус клиента
\end{itemize}

Пример ответа может выглядеть так:
\begin{lstlisting}[language=json,firstnumber=1]
{"client": {
  "id": "7817236178",
  "status": "vip",
  "current_items": 3,
  "bonus_miles": 9351,
  "clothes_current": {
    "cloth": [
      {"id": 4125124, "value": "Platje", "status": "clean", "price": 490},
      {"id": 89172387, "value": "Rubashka", "status": "being cleaned", "price": 199},
      {"id": 171273, "value": "Kostjum", "status": "preparation", "price": 2500}
    ],
    "furniture": [
    	{"id": 7887182, "value" "Pokrivalo on divana", "status": "preparation", "price": 4900}
    ]
  },
  "clother_all": {
  	"listing_all_clothes": ...
  }
}}
\end{lstlisting}

\subsection{Страница результата поиска (СЕРП)}
Серп показывает салоны, соответствующие критериям поиска, списком. Критерии поиска следующие:

\begin{itemize}
	\item Категория услуги - задается на главной странице
	\item Подкатегория услуги - по умолчанию "все" (пользователь может уточнить в фильтрах)
	\item Дата - по умолчанию "любая" (пользователь может уточнить в фильтрах)
	\item Цена - по умолчанию "любая" (фильтруемо)
	\item Локация - по умолчанию "рядом со мной" (см. \ref{subsubsec:location})
\end{itemize}

Каждая строка с салоном содержит в себе следующую информацию:

\begin{itemize}
	\item Название салона
	\item Цены салона на выбранную в фильтрах категорию (или подкатегорию) начиная от самой низкой
	\item Рейтинг салона
	\item Фотографии салона (желательно что бы в виде слайдера)
\end{itemize}

\subsubsection{Фильтры}
При нажатии на один из компонентов фильтра открывается (единая для всех параметров) страница. На данной странице пользователь может уточнить дату, время, подуслугу, адрес и ценовой диапазон. При нажатии на "применить" применяются установленные фильтры и пользователь возвращается снова на страницу СЕРПа.

\subsubsection{Локация} \label{subsubsec:location}
Приложение должно запрашивать геопозицию пользователя. Если пользователь дал разрешение, то показываем салоны в радиусе 1км. Если пользователь разрешения не дал, то показываем список самых популярных салонов Москвы.


\subsection{Детальная салона}
После нажатия на строку салона пользователь попадает на детальную страницу салона. Данная страница состоит из:

\begin{itemize}
	\item Шапка
	\item Слайдер с фотографиями салона (именно самого салона)
	\item Название салона, его рейтинг и кол-во отзывов
	\item Кнопка сброса фильтра ("показать все услуги")
	\item Перечень услуг, которые соответствуют поисковому запросу 
\end{itemize}

\subsubsection{Сброс фильтра}
Когда пользователь переходит на данную страницу со страницы СЕРПа, у него в параметрах есть категория (и иногда и подкатегория) запроса услуги. По умолчанию на странице салона нужно показывать только те услуги, которые относятся к запросу. И только после нажатия на "показать все услуги" нужно показывать все услуги салона.

\subsection{Детальная услуги}
При нажатии на одну из услуг салона пользователь попадает на данную страницу. Она состоит из:

\begin{itemize}
	\item Шапка
	\item Слайдер (тут уже фотографии именно конкретной услуги, а не салона)
	\item Панель с названием салона, отзывами, рейтингом и названием услуги
	\item Панель выбора мастера - при нажатии выезжает вниз ещё одна панель, на которой пользователь может радиобаттонами выбрать мастера. Также, ниже каждого мастера виден график его занятости в виде график-бара. Данный бар показывает отрезками разного цвета, когда мастер свободен и когда занят. Сам бар делится на три или четыре отрезка - утро, день, вечер, ночь. Соответственно если мастер занят с 9:00 до 11:00, то часть отрезка "утро" закрашено цветом, который соответствует статусу "занят", а остальная часть отрезка "утро" - цветом "свободен".
	\item Панель выбора времени - аналогично мастеру
\end{itemize}

Здесь важен один момент - обе панели должны подхватывать данные друг друга (пользователь может начать с любой). Т.е. если я выбираю определенного мастера, то время мне должно выдавать только то, в которое мастер свободен.
\\[0.5cm]
Аналогично и обратное - если я выбираю определенное время, то панель мастеров должна выдавать только тех мастеров, которые работают в выбранное время.

\subsection{Страница подтверждения брони}
При нажатии на "перейти к броне" (на странице услуги) пользователь попадает на данную страницу. Здесь он видит обзор своего заказа - какой салон, какая услуга, какой мастер, дата услуги. Пользователь также может выбрать между способами оплаты - кредитной картой или наличными в салоне.

\subsection{Отзывы}

\subsubsection{Список}

Нажав на рейтинг или отзывы конкретного салона пользователь попадает на данную страницу. Список состоит из:

\begin{itemize}
	\item Шапка
	\item Панель с данными салона (название, адрес)
	\item Средний балл салона
	\item Перечень категорий и индивидуальная оценка каждой категории
\end{itemize}

\subsubsection{Детальная}
При нажатии на категорию или оценку категории пользователь попадает на данную страницу. Страница состоит из перечня всех отзывов, которые относятся к данной услуге. Компоненты:

\begin{itemize}
	\item Автор отзыва
	\item Дата отзыва
	\item Текст отзыва
	\item Оценка 
\end{itemize}

\subsection{Профиль}
Страница настройки пользователя (если незалогинин, то предлагаем это сделать или зарегистрироваться). Следующие пункты можно менять:

\begin{itemize}
	\item когда получать пуш-уведомления (за 24 часа до брони, за 3 часа до брони и т.д.)
	\item фио
	\item почту
	\item прикрепить кредитную карту (или отвязать существующую)
	\item поменять пароль
\end{itemize}

\subsection{Меню}
Состоит из следующих пунктов:

\begin{itemize}
	\item Главная
	\item Мои бронирования
	\item Профиль
	\item Пригласить друга
	\item Новости
	\item Акции
	\item Ссылки на соц. сети
\end{itemize}

\subsubsection{Список броней}
Если пользователь залогинин, то на данной странице отображаются плашки с посещениями пользователя (если не залогинин, то предлагаем это сделать или пройти регистрацию). Фильтр - по дате.
Каждая плашка содержит в себе - название салона, дату и стоимость услуги. Также есть оценка пользователя или плейсхолдер, если пользователь не оценил услугу. Оценить услугу можно только тогда, когда салон оформил клиента.

\paragraph{Форма отзыва}
При нажатии на плейсхолдер отзыва пользователь попадает на данную страницу, на которой может оставить отзыв о салоне (о конкретной услуге).



\section{Взаимодействие с Тетрадкой}
После успешной оплаты данные о брони должны попадать в соответствующий салон в календарь (также должно приходить оповещение).

\end{document}








