\documentclass[DIV=calc, paper=a4, fontsize=11pt]{scrartcl} % Документ принадлежит классу article, а также будет печататься в 12 пунктов.
\usepackage{ucs}
\usepackage[T1,T2A]{fontenc}
\usepackage[utf8x]{inputenc} % Включаем поддержку UTF8
\usepackage[russian]{babel} % Пакет поддержки русского языка
\usepackage{titling} % Allows custom title configuration

%for frames
\usepackage{framed}

%For image using
\usepackage{graphicx}

%Numbering subsubsubsections etc
\setcounter{secnumdepth}{5}

%For code highlightning
\usepackage{listings}

%for nice ruby highlight
\lstloadlanguages{Ruby}
\lstset{%
basicstyle=\ttfamily\color{black},
commentstyle = \ttfamily\color{red},
keywordstyle=\ttfamily\color{blue},
stringstyle=\color{orange}}

%Further enumeration
\usepackage{enumitem}
\setenumerate[1]{label=\theparagraph.\arabic*.}
\setenumerate[2]{label*=\arabic*.}
\setenumerate[3]{label*=\arabic*.}


%For referencing within enumeration lists
\usepackage{enumitem}

%Packages for word-like comment style
\usepackage{todonotes}

%Package for images
\usepackage{float}
\floatstyle{boxed}
\restylefloat{figure}

%For a nicer reference
\usepackage{fancyref}

\usepackage{titlesec}

\usepackage{xcolor}

% For nice JSON output 
\colorlet{punct}{red!60!black}
\definecolor{background}{HTML}{EEEEEE}
\definecolor{delim}{RGB}{20,105,176}
\colorlet{numb}{magenta!60!black}

\lstdefinelanguage{json}{
    basicstyle=\normalfont\ttfamily,
    numbers=left,
    numberstyle=\scriptsize,
    stepnumber=1,
    numbersep=8pt,
    showstringspaces=false,
    breaklines=true,
    frame=lines,
    backgroundcolor=\color{background},
    literate=
     *{0}{{{\color{numb}0}}}{1}
      {1}{{{\color{numb}1}}}{1}
      {2}{{{\color{numb}2}}}{1}
      {3}{{{\color{numb}3}}}{1}
      {4}{{{\color{numb}4}}}{1}
      {5}{{{\color{numb}5}}}{1}
      {6}{{{\color{numb}6}}}{1}
      {7}{{{\color{numb}7}}}{1}
      {8}{{{\color{numb}8}}}{1}
      {9}{{{\color{numb}9}}}{1}
      {:}{{{\color{punct}{:}}}}{1}
      {,}{{{\color{punct}{,}}}}{1}
      {\{}{{{\color{delim}{\{}}}}{1}
      {\}}{{{\color{delim}{\}}}}}{1}
      {[}{{{\color{delim}{[}}}}{1}
      {]}{{{\color{delim}{]}}}}{1},
}


\titleformat*{\section}{\LARGE\bfseries}
\titleformat*{\subsection}{\Large\bfseries}
\titleformat*{\subsubsection}{\large\bfseries}
\titleformat*{\paragraph}{\large\bfseries}
\titleformat*{\subparagraph}{\large\bfseries}

%For some math formulas if needed
\usepackage{mathtools}

% Some nice visualization
%\usepackage[svgnames]{xcolor} % Enabling colors by their 'svgnames'
\usepackage{fullpage}
%\renewcommand{\headrulewidth}{0.0pt} % No header rule
%\renewcommand{\footrulewidth}{0.4pt} % Thin footer rule
% End visualization

%smart enumeration
\renewcommand{\labelenumi}{\arabic{enumi}.}
\renewcommand{\labelenumii}{\arabic{enumi}.\arabic{enumii}}


% General document settings
% % %
% Accepted file upload formats
\newcommand{\AcceptedFormats}{.txt, .csv, .xls, .xlsx}

\title{ТЗ на ЛК и мобильное приложение для химчистки}

\date{01/04/2016}

\begin{document}

\maketitle

\section{Описание проекта}
Рабочее название проекта - cleanmeupscotty. Состоит из двух частей - личного кабинета пользователя (представляет собой веб-сайт на Битриксе) и двух мобильных приложений (или одним с разными доступами и разным функционалом).
\subsection{Личный кабинет (сайт)}
Суть заключается в следующем - в личный кабинет (ЛК) на сайт поступает информация от сервера, который обрабатывает данные ПО химчистки. Например о статусе определенной вещи. Также видна некая статистика - сколько вещей было сдано, сколько денег потрачено, сколько бонусных баллов заработано. Сайт также отдает информацию на сервер химчистки.
\subsection{Мобильное приложение для клиента}
Дублирует основной функционал личного кабинета (выводит данные сервера химчистки). Дополнительно позволяет вызывать водителя к текущему местоположению для того, что бы он забрал вещи.
\subsection{Мобильное приложение для водителя}
Позволяет принимать заказ от клиента. Отправляет данные на сервер химчистки.

\section{Понятия}

\begin{itemize}
	\item Сайт, личный кабинет, ЛК - веб-сайт, размещенный на хостинге, основанный на системе 1С Битрикс
	\item Сервер химчистки - физический сервер, на котором крутится определенное ПО, которое использует химчистка. Сервер собирает данные с компьютеров каждой точки и центрально их хранит. Условно можно сказать - ПО точек это интерфейс, сервер - это база данных.
	\item Мобильное приложение, МП - мобильное приложение для сайта, которое дублирует его функции.
\end{itemize}


\section{Схема работы}

	\begin{figure}[H]
        \centering
        \includegraphics[width=480px]{{scheme_v.2.0}.png}
        \caption{Схема связей в общей системе\label{fig:scheme.png}}
    \end{figure}
    
Из точек приема 1...N поступают данные о заказах на общий сервер. Сервер отправляет данные также и обратно (эта сторона не касается проекта, данный обмен налажен на стороне ПО химчистки).
\\[0.5cm]
У сервера есть интерфейс приложения (application interface, API), с помощью которого можно запрашивать данные с сервера. Эти данные показываются клиенту в его личном кабинете (на сайте или в мобильном приложении).
\\[0.5cm]
Заказывая сбор вещей на дом, клиент вызывает водителя, который забирает грязные вещи. Водитель через свое приложение отмечает что он их принял. Клиент в свою очередь отмечает, что он их отдал (нужно обязательно).


\section{Параметры}

\subsection{Архитектура проекта}
Практически все данные проект принимает, обрабатывает и отправляет на сервер химчистки. На своей стороне хранится самый необходимый минимум.
\\[0.5cm]
Проект изначально должен соблюдать правила объектно ориентированного программирования. Т.е. любой пользователь должен быть объектом класса "пользователь". Это распространяется на все сущности. Все действия системы должны быть методами своих классов. Псевдокод:
\begin{lstlisting}[language=Ruby]

#Class description

class User < ActiveRecord::Base 

def initialize(name, phone, mail)
	@name = name
	@phone = phone
	@mail = mail
end

# Register method
def register(user_params)
	User.create(user_params)
end

# Allow only whitelisted params
private
	def user_params
		params.require(:user).permit(:name, :phone, :mail)
	end
\end{lstlisting}

Программирование должно осуществляться по подходу TDD - test driven development. Это означает, что перед написанием функционала необходимо написать тест (который заведомо провалится, так как функционала нет). Далее пишется минимальный функционал, который позволяет пройти данный тест.
\\[0.5cm]
Псевдокод теста:
\begin{lstlisting}[language=Ruby]
require 'user'
	
Rspec.describe User, "#register" do
  context "when user is completely new" do
	it "creates a new entry with valid data" do
	  user = User.new
      user.name = "Vasya Pupkin"
	  user.phone = "8 903 227 88 74"
	  user.mail = "vasya@pupkin.ru"
	  user.register
				
	  expect(User.count).to change{User.count}.by(1)
	end
    it "does not create a new entry with invalid data" do
      user = User.new
	  user.name = "Vasya Pupkin"
	  user.phone = "j1lkh3i1u2y98"
	  user.mail = "someinvalidtext"
	  user.register	
		
      expect(User.count).not_to change{User.count}.by(1)
	end
  end
end
\end{lstlisting}
Т.е. в данном случае мы тестируем создание нового пользователя с валидными данными (первый случай) и с невалидными данными. В первом случае мы ожидаем увеличение общего кол-ва пользователей на 1, во втором случае мы не ожидаем увеличений.

\subsection{Сущности системы}

\begin{itemize}
	\item Аккаунт пользователя (пользователь)
	\item Прайс-лист
	\item Квитанция
	\item Изделие
	\item Скидка
\end{itemize}

\subsubsection{Структура сущностей}

К лицевому счету (аккаунту) привязаны квитанции (одна или несколько). К квитанциям привязаны изделия (одно или несколько). Эти данные не редактируются на сайте, а подгружаются как есть с сервера.

\paragraph{Товары}\label{paragraph:goods}
Изделия группируются также в свою очередь. Уровней 3:
\begin{itemize}
	\item Раздел - принадлежность к типу химической обработки, например "хим.чистка основной группы". Может быть "хим. чистка кожаных изделий".
	\item Категория - общая группа товаров, например "костюмная группа"
	\item Товар - конкретный товар, например "пиджак"
\end{itemize}

\paragraph{Прайслист}
Прайслист - есть один (по умолчанию). К нему привязаны, если нет иного, все изделия. Если к изделию привязывается другой прайс-лист, то он превалирует на дефолтным. Данные прайс-листа подгружаем с сервера, т.е. мы должны только контролировать какой сейчас актуальный прайс-лист для определенного товара.
\\[0.5cm]
Пример - дефолтно стоимость чистки шелковой сорочки составляет 490 рублей. Мы хотим провести акцию "Сорочки по 200!" - для этого создается новый прайс-лист с ограниченным сроком действия и для товаров категории "сорочки" прописываем цену "200 руб". Теперь на время действия прайслиста цена чистки любой сорочки должна быть 200 рублей.

\paragraph{Скидки}
Скидки не должны суммироваться. Т.е. если у клиента есть 5\% скидка и активен прайс-лист, по которому сорочки чистятся за 200 рублей, то данный клиент также платит 200 рублей. Иными словами личные скидки распространяются только на дефолтный прайслист.


\subsection{Личный кабинет}
В личном кабинете клиент может зарегистрироваться или авторизоваться, если его учетная запись уже есть в системе.

\subsubsection{Авторизация}
Если клиент уже есть в системе, то при первичной авторизации, нужно предложить ему ввести номер телефона и подтвердить смс-сообщением, после авторизации сразу предложить сменить пароль и дальнейшую авторизацию он может осуществлять с помощью логина и пароля.

\subsubsection{Регистрация}
Для регистрации клиенту необходимо внести только телефон. На данный номер приходит смс с паролем, который можно скопировать. Клиент может быть уже зарегистрирован в системе (так как сдавал вещи в точке приема и при этом оставил свой номер телефона).
\\[0.5cm]
При первичной регистрации (проверяется по отсутствию телефона в базе сервера) схема должна работать следующим образом - клиент заполняет фио, телефон и почту, эти данные отправляются на сервер. Сервер создает у себя новую учетную запись и дополняет её своими полями. Далее сайт забирает уже полную карточку клиента.

\paragraph{Поля}
\begin{itemize}
	\item ФИО - отдельные поля
	\item email
	\item телефон
	\item адрес
	\item лицевой счет
	\item класс клиента - содержит информацию о скидке клиента (например класс 12 означает скидку в 3\%)
	\item идентификатор - равен номеру дисконтной карты (не является ID в системе)
	\item тип идентификатора - может быть магнитная карта или промокод
	\item тип лицевого счета
	\item прайслист - привязка к сущности "прайслист". У клиента может быть только один
	\item аватарка - с функцией кропа под размер сайта
	\item дата рождения
	\item тип клиента (юрик/физик)
\end{itemize}

Так эти поля выглядят в ПО химчистки:

	\begin{figure}[H]
        \centering
        \includegraphics[width=480px]{profile_cleaning.png}
        \caption{Обзор полей в ПО химчистки\label{fig:profile_cleaning.png}}
    \end{figure}
    
\paragraph{Доступ}
Пользователь сайта/мобильного приложения может менять только следующие данные:

\begin{itemize}
	\item Телефон
	\item ФИО
	\item Почту
	\item Адрес
	\item Фото
\end{itemize}

\subsubsection{Запросы}
Личный кабинет и сервер обмениваются данными каждые 30 минут. В идеале мы отдаем и принимаем только изменения и делаем это в режиме реального времени. Но если можно передавать только целиком скоп данных, то делаем это каждые 30 минут.
\paragraph{Примеры}
\begin{itemize}
	\item Клиент регистрируется на сайте -> информация попадает на сервер ПО химчистки
	\item Клиент приходит в точку в городе, сдает одежду, его регистрируют в ПО конкретной точки. Эти данные попадают на сервер, мы их оттуда забираем. Клиент в итоге оказывается в системе сайта.
	\item Клиент меняет номер телефона на сайте -> отсылаем информацию на сервер
	\item Клиент приходит в точку приема и просит поменять телефон -> мы забираем обновившиеся данные с сервера
\end{itemize}


\paragraph{POST Request}
На сервер химчистки отправляется POST-запрос, который меняет данные на сервере. Передаем сущность, которую хотим поменять, идентификатор сущности и значения.
\begin{displaymath}
	POST http://api.cleanmeupscotty.ru/params?client\&id=7817236178\&name=Oleg
\end{displaymath}	

\paragraph{GET Request}
К серверу химчистки отправляется GET-запрос, который содержит в себе сущность, к которой обращаются и уникальный идентификатор сущности.
Пример запроса: 
\begin{displaymath}
	GET http://api.cleanmeupscotty.ru/params?client\&id=7817236178
\end{displaymath}	


\subsubsection{Ответ}\label{subsubsec:response_lk}
От сервера химчистки приходят данные в формате XML или JSON. Перечень данных:
\begin{itemize}
	\item Кол-во вещей в обработке
	\item Общее кол-во сданных вещей
	\item Накопленные баллы
	\item Статус клиента
	\item Перечень квитанций (статус квитанции, сумма по квитанции, дата оплаты и прочие свойства квитанций)
	\item Перечень вещей в обработке (для каждой вещи получаем уникальный айди от химчистки, наименование, статус по химчистке, цену на 
	\item Общий перечень всего сданных вещей
	\item Вся статичная информация клиента (фио, телефон, почта и прочее)
	\item Прайслисты
\end{itemize}

Пример ответа может выглядеть так:
\begin{lstlisting}[language=json,firstnumber=1]
{"client": {
  "id": "7817236178",
  "status": "vip",
  "current_items": 3,
  "bonus_miles": 9351,
  "clothes_current": {
    "cloth": [
      {"id": 4125124, "value": "Platje", "status": "clean", "price": 490},
      {"id": 89172387, "value": "Rubashka", "status": "being cleaned", "price": 199},
      {"id": 171273, "value": "Kostjum", "status": "preparation", "price": 2500}
    ],
    "furniture": [
    	{"id": 7887182, "value" "Pokrivalo on divana", "status": "preparation", "price": 4900}
    ]
  },
  "clother_all": {
  	"listing_all_clothes": ...
  }
}}
\end{lstlisting}

\subsubsection{Вызов водителя}
Аналогично мобильному приложению (см. пункт \ref{paragraph:driver_call})

\subsection{Мобильное приложение клиента}
Мобильное приложение является полным дублем веб-личного кабинета и показывает в удобном интерфейсе те же данные (см. пункт \ref{subsubsec:response_lk}). Дополнительно мобильное приложение позволяет вызвать водителя для сбора вещей, которые необходимо отдать в чистку.

\subsubsection{Запросы}
Мобильное приложение основано на веб-личном кабинете, соответственно мобильное приложение "общается" с сайтом, а тот, в свою очередь с сервером ПО химчистки. Дополнительно к общему функционалу у мобильного приложение есть уникальный функционал - заказ водителя.

\paragraph{Вызов водителя}\label{paragraph:driver_call}
При нажатии на кнопку у пользователя есть выбор двух опций - время и место. Время по умолчанию стоит "сейчас". Пользователь может альтернативно выбрать определенное время. Место стоит по умолчанию "здесь" и выбирает текущее местоположение пользователя, альтернативно пользователь может выбрать местоположение.
\\[0.5cm]
Для местоположения - так как сервис рассчитан на Москву выбираем только улицу, дом и квартиру.

\paragraph{Подтверждение заказа}
После того, как водитель у себя ввел все данные забранных вещей мы должны получить от клиента подтверждение, что все окей. Предложение из двух вариантов - либо в приложение водителя делаем экран, где клиент может расписаться, либо в приложении клиента после того, как водитель ввел все данные выводим кнопку "подтвердить заказ". Здесь на усмотрение разработчика.

\paragraph{Оплата}
При создании новой квитанции (это означает, что вещи были приняты в точке приема) необходимо создавать пуш-уведомление. По переходу показываем экран с суммой заказа (и на развороте перечень вещей) и кнопкой "оплатить". Если у клиента уже прикреплена карта, то списываем с нее сумму денег. Если нет, то переадресовываем на экран, где он эти данные должен ввести.
\\[0.5cm]
Очень важный момент - при создании новой квитанции на стороне сайта проверяем обязательно, если ли корреспондирующий заказ у водителя. Так как квитанцию можно создать и при приходи клиента в точку приема лично (и он таким бы образом оплатил заказ дважды).
\\[0.5cm]
Если корреспондирующий заказ есть, то все работает по вышеописанной схеме (пуш-уведомление, которое ведет на страницу оплаты), если "водительского" заказа нет, то вообще ничего не происходит.
\\[0.5cm]
Также в мобильном приложении важно иметь список квитанций со статусами оплачен/не оплачен, что бы в случае потери пуш-уведомления у клиента все равно была возможность оплатить квитанцию.

\subsection{Мобильное приложение водителя}
Данное приложение получает заявки от мобильных приложений клиентов. 

\subsubsection{Схема работы}
Водитель может принять заказа или проигнорировать его. После того, как он принял заказ должен открываться навигатор и показывать маршрут к месту назначения клиента. Принятые вызовы выводим списком и сортируем по дате, на которую вызвали. Т.е. если кто-то вызвал водителя на завтра, этот заказ будет в списке более ниже, чем тот, где вызвали на сегодня.

\paragraph{Подтверждение забора вещей}
По прибытии и забора вещей водитель должен отметить в системе, что он забрал вещи (или не забрал, в этом случае нужно указать причину) и какие вещи он забрал. Для этого ему необходимо вывести простой список вещей (не такой, как в пункте \ref{paragraph:goods}! Более простой, т.е. например "пиджак", "сорочка", "брюки", а не "пиджак замшевый", "пиджак шерстяной" и т.д.)
\\[0.5cm]
Данный лист мы не подтягиваем с сервера химчистки, этот лист является неким дигитальным вариантом обычного бумажного описного листа. Этот лист также никуда не передаем, просто храним у себя на стороне Битрикса.
\\[0.5cm]
Если навигатор отдал назад ответ, что водитель приехал и он не нажал ни одну из кнопок, то экран должен мигать красным.

\paragraph{Прием заказа}
Заказ попадает ко всем водителям сразу. 

%к водителям в определенном радиусе от клиента. Всегда проверяем по шагу, равному 1км. Если водитель(и) в радиусе 1км от местоположения клиента (или адреса заказа) не отреагировали (или таковых вообще нет), то ищем в радиусе 2км и т.д. Можно поставить лимит 200км, что бы приложение не уходило в бесконечный цикл. Увеличиваем шаг каждые 5 минут.%

%Пример - клиент размещает заказ на свое текущее местоположение. Приложение ищет водителей, находит одного в радиусе 1км. Тот занят, так как везет обратно вещи другому клиенту. Через некоторое время приложение расширяет радиус поиска до 2км. Там приложение находит двух водителей. Тот из них, который ближе к точке клиента получает запрос быстрее. Каждому водителю в текущем радиусе показываем запрос с 30и секундным запозданием, чем дальше он от точки назначения. Первый водитель ближе к клиенту, но он пропустил сообщение. Через 30 секунд запрос отображается второму водителю, который сразу подтверждает. Он и получает заказ, нажимает "поехали" и включается навигация к точке клиента.%

\paragraph{Навигация}
Имеет смысл использовать стороннюю, гугл или яндекс навигаторы например. От них необходим ответ, что по GPS координатам дивайс прибыл к цели.

\end{document}








