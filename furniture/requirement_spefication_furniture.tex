\documentclass[DIV=calc, paper=a4, fontsize=11pt]{scrartcl} % Документ принадлежит классу article, а также будет печататься в 12 пунктов.
\usepackage{ucs}
\usepackage[T1,T2A]{fontenc}
\usepackage[utf8x]{inputenc} % Включаем поддержку UTF8
\usepackage[russian]{babel} % Пакет поддержки русского языка
\usepackage{titling} % Allows custom title configuration

%for frames
\usepackage{framed}

%For image using
\usepackage{graphicx}

%Numbering subsubsubsections etc
\setcounter{secnumdepth}{5}

%For code highlightning
\usepackage{listings}

%Further enumeration
\usepackage{enumitem}
\setenumerate[1]{label=\theparagraph.\arabic*.}
\setenumerate[2]{label*=\arabic*.}
\setenumerate[3]{label*=\arabic*.}


%For referencing within enumeration lists
\usepackage{enumitem}

%Packages for word-like comment style
\usepackage{todonotes}

%Package for images
\usepackage{float}
\floatstyle{boxed}
\restylefloat{figure}

%For a nicer reference
\usepackage{fancyref}

\usepackage{titlesec}


\titleformat*{\section}{\LARGE\bfseries}
\titleformat*{\subsection}{\Large\bfseries}
\titleformat*{\subsubsection}{\large\bfseries}
\titleformat*{\paragraph}{\large\bfseries}
\titleformat*{\subparagraph}{\large\bfseries}

%For some math formulas if needed
\usepackage{mathtools}

% Some nice visualization
%\usepackage[svgnames]{xcolor} % Enabling colors by their 'svgnames'
\usepackage{fullpage}
%\renewcommand{\headrulewidth}{0.0pt} % No header rule
%\renewcommand{\footrulewidth}{0.4pt} % Thin footer rule
% End visualization

%smart enumeration
\renewcommand{\labelenumi}{\arabic{enumi}.}
\renewcommand{\labelenumii}{\arabic{enumi}.\arabic{enumii}}


\title{Техническое задания для сайта мебели (рабочее название - Furniture Project или FP)}
\date{03/08/2015}

\begin{document}

\maketitle

\section{Описание проекта}

Сайт представляет собой стандартный тип - business presentation. На усмотрение исполнителя это может быть одностраничник (в формате лендинга) или сайт с несколькими разделами (отдельными страницами)
На сайте клиент размещает информацию о своих проектах, описание данных проектов, а также описание своей компании.

\subsection{Общие требования к проекту}
Дизайн и верстка должны быть выполнены адаптивно.

\section{Страницы}

Тематически на сайте должны быть представлены ниже описанные разделы. Оформление (лендинг или мультистраничник) может быть вариабельным.

\subsection{Главная}

Главная представляет собой список проектов. Также должны присутствовать стандартные элементы, как то - шапка, футер. Приветствуются идеи анимации.
\\[0.5cm]
Список проектов имеет форму плиток (или иной userfriendly вид). Каждая плитка кликабельна и ведет на страницу конкретного проекта.

\subsection{Страница проекта}
Содержит в себе:
\begin{itemize}
	\item Рубрику
	\item Тематику
	\item Название проекта
	\item Фотографии проекта (открываются в превью)
	\item Описание проекта
	\item Участники проекта (с ссылкой на участника)
\end{itemize}

\subsection{Контакты}

Стандартная страница контактов. Содержит в себе:

\begin{itemize}
	\item Адрес
	\item Телефон
	\item Карта
	\item Форма обратной связи
\end{itemize}

\subsection{Перечень услуги}
Данная страница содержит перечень услуг. Перечень может быть в плиточном формате или в табличном. При mouseover'е должен меняться фон и всплывать описание с возможностью нажать на описание и попасть на детальную страницу услуги.

\subsection{Страница услуги}
Данная страница содержит в себе:
\begin{itemize}
	\item Название услуги
	\item Фотографию услуги (одну или несколько - можно сделать одну большую область, где показываем фотку в полный размер, а под ней идет слайдер со всеми фотографиями услуги)
	\item Описание услуги
\end{itemize}

\subsection{О нас}
Данная страница содержит в себе описание компании (текст) и перечень сотрудников с фотографиями (как идея - круглые тамбнейлы). Два варианта - при нажатии на фото сотрудника открывается отдельная страница или просто выезжает блок с небольшим кол-вом информации (в последнем случае на странице проектов нужно сделать так же, а не вести на страницу сотрудника)




\end{document}