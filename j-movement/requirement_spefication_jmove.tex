\documentclass[DIV=calc, paper=a4, fontsize=11pt]{scrartcl} % Документ принадлежит классу article, а также будет печататься в 12 пунктов.
\usepackage{ucs}
\usepackage[T1,T2A]{fontenc}
\usepackage[utf8x]{inputenc} % Включаем поддержку UTF8
\usepackage[russian]{babel} % Пакет поддержки русского языка
\usepackage{titling} % Allows custom title configuration

%for frames
\usepackage{framed}

%For image using
\usepackage{graphicx}

%Numbering subsubsubsections etc
\setcounter{secnumdepth}{5}


%Further enumeration
\usepackage{enumitem}
\setenumerate[1]{label=\thesubsection.\arabic*.}
\setenumerate[2]{label*=\arabic*.}


%For referencing within enumeration lists
\usepackage{enumitem}

%Packages for word-like comment style
\usepackage{todonotes}

%Package for images
\usepackage{float}
\floatstyle{boxed}
\restylefloat{figure}

%For a nicer reference
\usepackage{fancyref}

\usepackage{titlesec}


\titleformat*{\section}{\LARGE\bfseries}
\titleformat*{\subsection}{\Large\bfseries}
\titleformat*{\subsubsection}{\large\bfseries}
\titleformat*{\paragraph}{\large\bfseries}
\titleformat*{\subparagraph}{\large\bfseries}

%For some math formulas if needed
\usepackage{mathtools}

% Some nice visualization
%\usepackage[svgnames]{xcolor} % Enabling colors by their 'svgnames'
\usepackage{fullpage}
%\renewcommand{\headrulewidth}{0.0pt} % No header rule
%\renewcommand{\footrulewidth}{0.4pt} % Thin footer rule
% End visualization

%smart enumeration
\renewcommand{\labelenumi}{\arabic{enumi}.}
\renewcommand{\labelenumii}{\arabic{enumi}.\arabic{enumii}}


\title{Макет технического задания для EUROSTARS}
\date{17/09/2014}

\begin{document}

\maketitle

\section{Описание проекта}
Сайт представляет собой один из (примерно) 15и сайтов одинаковой тематики - еврейского молодежного движения. В первый подход будет реализован только первый сайт. Пользователь может регистрироваться на любом сайте, его данные должны быть действительны для всех последующих сайтов (поддержка мультисайтов).
\subsection{Суть сайта}
Веб-страница содержит в себе информацию по движению, а также некоторые интерактивные элементы. Участники движения с соответствующими правами могут создавать event'ы и внутри event'а много небольших событий. Пример такой вложенности может быть:
\\ Event - поездка в Израиль
\\ subevents - музей номер 1 в понедельник, музей номер 2 в среду, лекции в четверг
\subsection{Пользователи}
Сайт заточен под несколько ролей пользователей. На данный момент в их числе - админ, лектор и обычный пользователь. Лектор и пользователь всегда привязаны к месту жительства (центральный аспект системы).
\\[0.5cm]
Обычные пользователи могут записываться на события и принимать в них участие если лектор одобряет данного пользователя. Также обычные пользователи могут подавать заявки на получение визы для определенного события и могут через сайт отслеживать прогресс получения визы.
\\[0.5cm]
Также обычные пользователи могут просматривать прошедшие мероприятия, которые они посещали.
\\[0.5cm]
Пользователи могут просматривать все события лектора.
\subsection{Локация}
С привязкой пользователей к месту обитания связаны следующие функциональные части - лектор может на сайте (не в админке) просматривать всех пользователей своего города. Лектор также может в соответствующем разделе скачивать файлы, относящиеся к его локации.

\section{Технические уточнения}

\subsection{Термины}
    \begin{enumerate}
        \item Блок - некий визуальный элемент, выделяющийся либо графически (в виде рамок, очертаний), либо по смыслу (совокупность похожих элементов)
        \item Компонент - часть содержания, имеющего закрытое визуально представление. Одна страница сайта может состоять из нескольких компонентов.
        \item Фронтэнд - для пользователя видимая оболочка веб-страницы
        \item Бэкэнд - невидимые для пользователя математические алгоритмы
        \item CMS - все работы происходят на основе системы управления содержанием - CMS 1C Bitrix (1С Битрикс)
        \item Модуль - является описанием общего функционала, который не может быть классифицирован как привязанный к определенной странице. Он может встречаться на любой странице в любом месте. Модуль может состоять из нескольких компонентов. Также модуль может содержать в себе части логики фронтэнда и бэкэнда.
        \item Хэдер - верхняя часть сайта, обладающая определенной структурой, которая видна сквозняком на всех или почти всех страницах сайта. Также используется обозначение "шапка".
        \item Футер - нижняя часть сайта. Функционал аналогичен хэдеру. Также используется обозначение "подвал".
    \end{enumerate}


\subsection{Технические требования к сайту}
Сайт должен быть адаптивным, поддерживать IE 8+ и использовать технологию композита.


\section{Модули}

\subsection{События}
Событие представляет собой информационный блок, который содержит в себе кроме перечня подсобытий также следующую информацию:

    \begin{enumerate}
        \item Создателя события (администратор) 
        \item Дату начала и конца события
        \item Страну проведения события - возможен мультивыбор 
    \end{enumerate}
    
\subsection{Подсобытие}
"Подсобытие" представляет собой небольшое мероприятия на время проведения всей поездки. Для каждого подсобытия необходима возможность сохранять и показывать следующую информацию:


    \begin{enumerate}
        \item Дата и время подсобытия
        \item Тип подсобытия (музей, выставка, лекция...) и название
        \item Лектор подсобытий - для определенных типов (лекция на начальном этапе) нужно будет запрашивать при создании лектора. Т.е. тип "лекция" не может быть создан без указания лектора.
        \item Максимальное кол-во участников
        \item Фотографии
        \item Описание
    \end{enumerate}
Модерация участников проходит целиком на все событие, т.е. как только пользователь подтверждается администратором как одобренный для всех подсобытий, для которых существуют ограничения действует принцип - кто первый пришел, того и место.

\subsection{Рассылка}
Помимо стандартного функционала Битрикса касательно рассылки необходимо добавить фильтры пользователей и их мероприятий при создании рассылки, т.е. например отослать только пользователем одного города, только определенного пола, только тем, кто посетил определенные мероприятия.

\section{Страницы}

\subsection{Главная}

На главной странице отображены блоки последующих страниц. Т.е. это могут быть новости, описание программы, расписание текущего/грядущего события, фрагмент галлереи.

\subsection{Новости}
Страница с лентой новостей. На этой же странице возможен поиск по архиву, фильтрация новостных записей.

\subsection{О программе}
Страница содержит текстовое описание программы и большую ссылку на расписание.
\subsubsection{Расписание}
В начале страницы пользователь видит переключающие ссылки между текущим событием и архивом. На все движение всегда действует одно активное событие.

\paragraph{Текущие событие}
Текущее событие представлено в виде двухмерной ленты. По вертикальной ленте отображаются дни, по горизонтальной отображаются часы. Пример:
        \begin{figure}[H]
        \centering
        \includegraphics[width=320px]{j_move_event_feeds.png}
        \caption{Лента события с подсобытиями\label{fig:j_move_event_feeds.png}}
        \end{figure}

Подсобытия кликабелны и ведут каждое на свою собственную страницу с подробным описанием, фотографиями, лектором. Лектор является кликабельной ссылкой которая ведет на профиль лектора с обзором всех его событий.
        
\paragraph{Архив \label{archive}} 
Сразу после завершения события оно переходит в архив. Пользователь может просматривать архив. Архив представляет собой список, фильтры которого можно настраивать в начале списка - дату, место проведения. Каждая строка данного списка состоит из даты события и его места проведения. При нажатии на строку показываем DIV с лентой события, которая соответствует ленте активного события, только без возможности записаться на данное событие.

\subsection{Страница подсобытия}
На данной странице расположены:
    \begin{enumerate}
        \item Заглавная фотография подсобытия (если таковая имеется)
        \item Текст с описанием мероприятия
        \item Техническая информация - сколько участников (если ограничено), кто лектор, где, когда
        \item Список участников
    \end{enumerate}


\subsection{Галлерея}
Делится на время и места, при просмотре фотографии можно скачивать. Пример вложенности:
2014 - лето - Польша - музей ВанГога

\subsection{Личный кабинет и связанные с ним страницы}
\subsubsection{Регистрация}
Помимо стандартных полей запрашиваем обязательно город.Город должен обязательно выбираться из списка. Если город в списке имеется, то пользователю приходит на почту оповещение, что по ссылке он может пройти полноценную регистрацию, во время которой он заполняет все поля\todo[color=blue!20!white]{Список полей будет позже}
\\[0.5cm]Если города нет, пользователь выбирает пункт "другой" и вводит город сам, ему приходит мейл (и всплывает оповещение на сайте), что "к сожалению, в вашем городе не проводятся мероприятия, мы обязательно вам сообщим как это случится" далее этого пользователя регистрируем, но на события записаться он не может. Одновременно отправляем администратору сайта запрос на добавление нового города. Как только город был добавлен администратором пользователю становится доступной функция регистрации.

\subsubsection{Заявка на визу/статус визы}
В виде отдельной закладки (таб) пользователь может в ЛК просматривать статус своей визы на текущее мероприятие (или подать заявку на получение визы).
\\[0.5cm]
Для конкретного события пользователь может подавать заявления на получение визы. Статус обработки необходимо отображать в виде визуального элемента (красный, например, квадрат - означат отказ в получении; желтый - заявка находится на стадии рассмотрения; зеленый - виза одобрена)
\\[0.5cm]
Важно сохранять значения визы относительно страны проведения мероприятия и срок действия визы. Т.е. если в будущем появляется новое мероприятие в стране, для которой действует любая активная виза пользователя, то в закладке показываем, что статус визы для этого мероприятия активен. Программа должна проверять для каждого нового мероприятия, на которое регистрируется пользователь, есть ли у данного пользователя виза, действующая для страны мероприятия и является ли она активной до окончания мероприятия. Также программа должна проверять данные для всех стран мероприятия (если их несколько).
\\[0.5cm]
\textbf{\textit{Для бэкэнда - необходимо присвоить страны визовым блокам, таких 3 - безвизовая страна, страна шенгена и страна с индивидуальной визой.}}
\subsubsection{Посещенные мероприятия}
Также в виде закладки (таб) показываем списком посещенные пользователем мероприятия. Здесь нужно использовать модуль события, который подгружаем аналогично архиву (см. пункт \ref{archive})

\subsubsection{Файлы для скачивания}
Админы могут рассылать определенные файлы лекторам. Лектор видит данные файлы в ЛК в соответствующей закладке. Обзор файлов представлен общим списком, поделенным на категории.


\section{Примеры использования}

\begin{framed}
\subsection{Пользователь Леонид}
Леонид узнал о сайте совсем недавно. Леонид очень любознательный молодой человек и технологично продвинутый (и активный) и хочет больше узнать о Еврейской общине и видит в движении ЕвроСтарс подходящий способ получить позновательную информацию. Так как сайт обладает приятным, стильным интерфейсом это ещё больше сподвигает Леонида на пользование ресурсом.
\\[0.5cm]
Помимо просмотра доступных страниц Леонид может зарегистрироваться на сайте, указав свои данные - логин, почту, пароль и где он живет. Так как его город был сразу доступен для выбора, Леонид видит приветствующее окно, говорящее ему, что Леонид может теперь записываться на события и подсказывает ему сразу ближайшее событие, на которое он может записаться.
\\[0.5cm]
Просмотрев событие Леонид определил что это время и тематика ему по душе. Он записывается на поездку, отправляя при этом запрос. Статус подтверждения приходит ему после некоторого времени как на почту, так и на самом сайте в виде всплывающего окна. После подтверждения Леонид просматривает ещё раз список подсобытий и выбирает для себя подходящие (простым нажатием на принятие участия).

Так как у Леонида нет визы для данной поездки он в личном кабинете может в режиме онлайн загрузить необходимые документы - список чекбоксов подскзаывает ему, какие док-ты нужны. После отправки заявки на получение визы он также в личном кабинете может просматривать статус заявки, а на почту ему приходят оповещения о смене статуса.
\end{framed}

\begin{framed}
\subsection{Лектор Александр}
\end{framed}

\begin{framed}
\subsection{Администратор Иосиф}
\end{framed}





\end{document}