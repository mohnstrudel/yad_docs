\documentclass[DIV=calc, paper=a4, fontsize=11pt]{scrartcl} % Документ принадлежит классу article, а также будет печататься в 12 пунктов.
\usepackage{ucs}
\usepackage[T1,T2A]{fontenc}
\usepackage[utf8x]{inputenc} % Включаем поддержку UTF8
\usepackage[russian]{babel} % Пакет поддержки русского языка
\usepackage{titling} % Allows custom title configuration

%for frames
\usepackage{framed}

%For image using
\usepackage{graphicx}

%Numbering subsubsubsections etc
\setcounter{secnumdepth}{5}

%For code highlightning
\usepackage{listings}

%for nice ruby highlight
\lstloadlanguages{Ruby}
\lstset{%
basicstyle=\ttfamily\color{black},
commentstyle = \ttfamily\color{red},
keywordstyle=\ttfamily\color{blue},
stringstyle=\color{orange}}

%Further enumeration
\usepackage{enumitem}
\setenumerate[1]{label=\theparagraph.\arabic*.}
\setenumerate[2]{label*=\arabic*.}
\setenumerate[3]{label*=\arabic*.}


%For referencing within enumeration lists
\usepackage{enumitem}

%Packages for word-like comment style
\usepackage{todonotes}

%Package for images
\usepackage{float}
\floatstyle{boxed}
\restylefloat{figure}

%For a nicer reference
\usepackage{fancyref}

\usepackage{titlesec}

\usepackage{xcolor}

% For nice JSON output 
\colorlet{punct}{red!60!black}
\definecolor{background}{HTML}{EEEEEE}
\definecolor{delim}{RGB}{20,105,176}
\colorlet{numb}{magenta!60!black}

\lstdefinelanguage{json}{
    basicstyle=\normalfont\ttfamily,
    numbers=left,
    numberstyle=\scriptsize,
    stepnumber=1,
    numbersep=8pt,
    showstringspaces=false,
    breaklines=true,
    frame=lines,
    backgroundcolor=\color{background},
    literate=
     *{0}{{{\color{numb}0}}}{1}
      {1}{{{\color{numb}1}}}{1}
      {2}{{{\color{numb}2}}}{1}
      {3}{{{\color{numb}3}}}{1}
      {4}{{{\color{numb}4}}}{1}
      {5}{{{\color{numb}5}}}{1}
      {6}{{{\color{numb}6}}}{1}
      {7}{{{\color{numb}7}}}{1}
      {8}{{{\color{numb}8}}}{1}
      {9}{{{\color{numb}9}}}{1}
      {:}{{{\color{punct}{:}}}}{1}
      {,}{{{\color{punct}{,}}}}{1}
      {\{}{{{\color{delim}{\{}}}}{1}
      {\}}{{{\color{delim}{\}}}}}{1}
      {[}{{{\color{delim}{[}}}}{1}
      {]}{{{\color{delim}{]}}}}{1},
}


\titleformat*{\section}{\LARGE\bfseries}
\titleformat*{\subsection}{\Large\bfseries}
\titleformat*{\subsubsection}{\large\bfseries}
\titleformat*{\paragraph}{\large\bfseries}
\titleformat*{\subparagraph}{\large\bfseries}

%For some math formulas if needed
\usepackage{mathtools}

% Some nice visualization
%\usepackage[svgnames]{xcolor} % Enabling colors by their 'svgnames'
\usepackage{fullpage}
%\renewcommand{\headrulewidth}{0.0pt} % No header rule
%\renewcommand{\footrulewidth}{0.4pt} % Thin footer rule
% End visualization

%smart enumeration
\renewcommand{\labelenumi}{\arabic{enumi}.}
\renewcommand{\labelenumii}{\arabic{enumi}.\arabic{enumii}}


% General document settings
% % %
% Accepted file upload formats
\newcommand{\AcceptedFormats}{.txt, .csv, .xls, .xlsx}
\newcommand{\ProjectTitle}{WorkWell}

\title{Техническое задание на разработку портала}

\date{27/07/2016}

\begin{document}

\maketitle

\section{Описание проекта}
Рабочее название проекта - \ProjectTitle . Проект представляет собой платформу, состоящую из нескольких сервисов. Главная цель проекта - позволять пользователям бронировать места в коворкингах, которые принадлежат владельцу.

\subsection{Цели платформы для клиента}
Клиент (пользователь платформы) может регистрироваться в системе (различные абонементы). Он может переписываться с другими участникам (чат), он может принимать участие в специальных мероприятиях, создавать и откликаться на объявления, получать бонусы и бронировать помещения/столы.

\subsection{Цели сайта для заказчика}
Проект дает заказчику возможность управлять своим контентом. Он может создавать новые здания с интерактивными планами, управлять данными планами, получать статистику загруженности помещений.

\section{Технические уточнения}

\subsection{Термины}

        \begin{itemize}
        \item Блок - некий визуальный элемент, выделяющийся либо графически (в виде рамок, очертаний), либо по смыслу (совокупность похожих элементов)
        \item Компонент - часть содержания, имеющего закрытое визуально представление. Одна страница сайта может состоять из нескольких компонентов.
        \item Фронтэнд - для пользователя видимая оболочка веб-страницы
        \item Бэкэнд - невидимые для пользователя математические алгоритмы
        \item CMS - все работы происходят на основе системы управления содержанием - CMS 1C Bitrix (1С Битрикс)
        \item Модуль - является описанием общего функционала, который не может быть классифицирован как привязанный к определенной странице. Он может встречаться на любой странице в любом месте. Модуль может состоять из нескольких компонентов. Также модуль может содержать в себе части логики фронтэнда и бэкэнда.
        \item Хэдер - верхняя часть сайта, обладающая определенной структурой, которая видна сквозняком на всех или почти всех страницах сайта. Также используется обозначение "шапка".
        \item Футер - нижняя часть сайта. Функционал аналогичен хэдеру. Также используется обозначение "подвал".
        \item Администраторская панель ("админка") - часть сайта, скрытая для обычных пользователей, позволяющая редактировать содержимое сайта
        \item Публичная часть ("паблик") - часть сайта, доступная для просмотра обычным пользователям.
        \item Слайдер - блок с фотографиями и (опционально) текстом, которые можно листать (влево/вправо). Обычно размещается в верхней части страницы и занимает всю ширину экрана. Также возможны более узкие слайдеры.

    \end{itemize}
    
\subsection{Технические требования к сайту}
Сайт располагается на хостинге или выделенном сервере (virtual dedicated server), в зависимости от предпочтений заказчика.



\section{Архитектура проекта}

\subsection{Подход API first}
Для данного проекта используется подход "API first". Это означает, что разрабатывается RESTful API, которая позволяет получить доступ к данным вне зависимости от приложения. Таким образом мы разграничиваем обработку данных и вывод данных клиенту.

\subsection{Структура}
Проект имеет следующую структуру:
\begin{itemize}
	\item Основное приложение
	\item API-надстройка
	\item Фронтэнд-приложение (может быть одно или несколько)
	\item Мобильное приложение
\end{itemize}
Имеется основная установка платформы, которая фигурирует как администрируемая база данных (на основе Битрикс: Управление Сайтом). Для создания API используется дополнительный фреймворк (так как методами Битрикса создать API невозможно, а писать чистый PHP очень долго) - это может быть ZEND, Yii 2 или Symfony (на обоснованный выбор разработчика).

\subsection{Модули проекта}
Модули проекта - это сайт или приложение с определенным набором функционала и как правило, обладающий собственным названием. Перечень модулей:
\begin{itemize}
	\item Личный кабинет пользователя
	\item Мобильное приложение
	\item Лендинг
	\item Оплата
\end{itemize}


\subsection{Компоненты модулей}

\subsubsection{Личный кабинет}

Модуль состоит из следующий компонентов (могут пересекаться с понятием "страница", но не обязательно).

\begin{itemize}
	\item Дашборд
	\item Акции
	\item Друзья
	\item Сообщества
	\item Места (брони)
	\item Мероприятия
\end{itemize}


\subsection{Общее}

Проект изначально должен соблюдать правила объектно ориентированного программирования. Т.е. любой пользователь должен быть объектом класса "пользователь". Это распространяется на все сущности. Все действия системы должны быть методами своих классов. Псевдокод:
\begin{lstlisting}[language=Ruby]

#Class description

class User < ActiveRecord::Base 

def initialize(name, phone, mail)
	@name = name
	@phone = phone
	@mail = mail
end

# Register method
def register(user_params)
	User.create(user_params)
end

# Allow only whitelisted params
private
	def user_params
		params.require(:user).permit(:name, :phone, :mail)
	end
\end{lstlisting}

Программирование должно осуществляться по подходу TDD - test driven development. Это означает, что перед написанием функционала необходимо написать тест (который заведомо провалится, так как функционала нет). Далее пишется минимальный функционал, который позволяет пройти данный тест.
\\[0.5cm]
Псевдокод теста:
\begin{lstlisting}[language=Ruby]
require 'user'
	
Rspec.describe User, "#register" do
  context "when user is completely new" do
	it "creates a new entry with valid data" do
	  user = User.new
      user.name = "Vasya Pupkin"
	  user.phone = "8 903 227 88 74"
	  user.mail = "vasya@pupkin.ru"
	  user.register
				
	  expect(User.count).to change{User.count}.by(1)
	end
    it "does not create a new entry with invalid data" do
      user = User.new
	  user.name = "Vasya Pupkin"
	  user.phone = "j1lkh3i1u2y98"
	  user.mail = "someinvalidtext"
	  user.register	
		
      expect(User.count).not_to change{User.count}.by(1)
	end
  end
end
\end{lstlisting}
Т.е. в данном случае мы тестируем создание нового пользователя с валидными данными (первый случай) и с невалидными данными. В первом случае мы ожидаем увеличение общего кол-ва пользователей на 1, во втором случае мы не ожидаем увеличений.



\subsection{Модели}
Перечень:

\begin{itemize}
	\item Пользователь
	\item Стейтмент
	\item Комментарий
	\item Страна
	\item Город
	\item Здание
	\item Этаж
	\item Отсек
	\item Акция
	\item Мероприятие
\end{itemize}

\subsubsection{Структура моделей}

\paragraph{Пользователь}
Пользователь имеет связку has-and-belongs-to-many :through с другими пользователями. Это означает, что есть дополнительная таблица, которая ведет учет пользователей, которые находятся у изначального пользователя в друзьях и фолловерах.

\begin{framed}
Пример
\\[0.5cm]
Original-user.friends - выдает всех друзей оригинального пользователя.
\\[0.5cm]
Original-user.followers - выдает всех фолловеров оригинального пользователя.
\end{framed}

Пользователь может писать стейтмент, т.е. у одного пользователя может быть несколько стейтментов, но у одного стейтмента только один пользователь.

\paragraph{Стейтмент}
Может иметь один или несколько комментариев. Принадлежит в свою очередь одной категории - пользовательский стейтмент, стейтмент в разделе акция или мероприятие. 
\\[0.5cm]
Стейтментам можно ставить лайк. Таким образом стейтмент появляется в ленте всех тех людей, у который есть в друзьях тот, кто поставил лайк. Также, стейтмент, составленный человеком, который есть у пользователя в друзьях или фолловерах, появится в ленте пользователя.

\paragraph{Гео-данные}
У страны может быть много городов. У города может быть много зданий. У здания может быть много этажей, у этажа может быть много отсеков.

\paragraph{Отсек}
Отсек может иметь категорию - комната, стол+стул, переговорка, конференц-зал и прочее.
\\[0.5cm]
Отсек можно бронировать через соответствующую форму. При этом он закрепляется за его забронировавшим пользователем и у пользователя списываются credits. После бронирования на карте отсек помечается красным, если выбрано соответствующее время.


%% New Section
\section{Модули}

\subsection{Оплата}
Необходимо встроить прием онлайн-оплаты через агрегатор "CloudPayments". При успешной оплате формируется заказ и клиент перенаправляется на соответствующую success-страницу.

\subsection{Интерактивная карта}
Сердце данного приложения.

%% New Section
\section{Страницы}

\subsection{Личный кабинет}


\subsection{Типовые страницы}

\begin{itemize}
	\item Контакты (impressum) - необходимо подгружать из настроек проекта
	\item О проекте - текстовая страница
	\item Соглашение пользователя - текстовая страница
	\item Конфиденциальность - текстовая страница
	\item Новости - управляется соответствующим блоком в администраторской панели (либо отдельный соответствующий модуль битрикса, либо инфоблок - на усмотрение разработчика)
	\item Часто задаваемые вопросы - управляется блоком в админке (инфоблок)
	\item Обратная связь - контактная форма, присылает сообщение на адрес из настроек магазина (проработать возможность подключить битриксовый модуль ТП, т.е. что бы ответы из модуля приходили клиенту на почту)
	\item Регистрация пользователя - клиент должен ввести мэйл и пароль. На почту ему приходит подтверждение о регистрации.
	\item Забытый пароль - страница позволяет ввести мэйл и восстановить пароль
\end{itemize}

\end{document}








