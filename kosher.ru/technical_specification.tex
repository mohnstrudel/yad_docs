\documentclass[DIV=calc, paper=a4, fontsize=11pt]{scrartcl} % Документ принадлежит классу article, а также будет печататься в 12 пунктов.
\usepackage{ucs}
\usepackage[T1,T2A]{fontenc}
\usepackage[utf8x]{inputenc} % Включаем поддержку UTF8
\usepackage[russian]{babel} % Пакет поддержки русского языка
\usepackage{titling} % Allows custom title configuration

%for frames
\usepackage{framed}

%For image using
\usepackage{graphicx}

%Numbering subsubsubsections etc
\setcounter{secnumdepth}{5}

%For code highlightning
\usepackage{listings}

%for nice ruby highlight
\lstloadlanguages{Ruby}
\lstset{%
basicstyle=\ttfamily\color{black},
commentstyle = \ttfamily\color{red},
keywordstyle=\ttfamily\color{blue},
stringstyle=\color{orange}}

%Further enumeration
\usepackage{enumitem}
\setenumerate[1]{label=\theparagraph.\arabic*.}
\setenumerate[2]{label*=\arabic*.}
\setenumerate[3]{label*=\arabic*.}


%For referencing within enumeration lists
\usepackage{enumitem}

%Packages for word-like comment style
\usepackage{todonotes}

%Package for images
\usepackage{float}
\floatstyle{boxed}
\restylefloat{figure}

%For a nicer reference
\usepackage{fancyref}

\usepackage{titlesec}

\usepackage{xcolor}

% For nice JSON output 
\colorlet{punct}{red!60!black}
\definecolor{background}{HTML}{EEEEEE}
\definecolor{delim}{RGB}{20,105,176}
\colorlet{numb}{magenta!60!black}

\lstdefinelanguage{json}{
    basicstyle=\normalfont\ttfamily,
    numbers=left,
    numberstyle=\scriptsize,
    stepnumber=1,
    numbersep=8pt,
    showstringspaces=false,
    breaklines=true,
    frame=lines,
    backgroundcolor=\color{background},
    literate=
     *{0}{{{\color{numb}0}}}{1}
      {1}{{{\color{numb}1}}}{1}
      {2}{{{\color{numb}2}}}{1}
      {3}{{{\color{numb}3}}}{1}
      {4}{{{\color{numb}4}}}{1}
      {5}{{{\color{numb}5}}}{1}
      {6}{{{\color{numb}6}}}{1}
      {7}{{{\color{numb}7}}}{1}
      {8}{{{\color{numb}8}}}{1}
      {9}{{{\color{numb}9}}}{1}
      {:}{{{\color{punct}{:}}}}{1}
      {,}{{{\color{punct}{,}}}}{1}
      {\{}{{{\color{delim}{\{}}}}{1}
      {\}}{{{\color{delim}{\}}}}}{1}
      {[}{{{\color{delim}{[}}}}{1}
      {]}{{{\color{delim}{]}}}}{1},
}


\titleformat*{\section}{\LARGE\bfseries}
\titleformat*{\subsection}{\Large\bfseries}
\titleformat*{\subsubsection}{\large\bfseries}
\titleformat*{\paragraph}{\large\bfseries}
\titleformat*{\subparagraph}{\large\bfseries}

%For some math formulas if needed
\usepackage{mathtools}

% Some nice visualization
%\usepackage[svgnames]{xcolor} % Enabling colors by their 'svgnames'
\usepackage{fullpage}
%\renewcommand{\headrulewidth}{0.0pt} % No header rule
%\renewcommand{\footrulewidth}{0.4pt} % Thin footer rule
% End visualization

%smart enumeration
\renewcommand{\labelenumi}{\arabic{enumi}.}
\renewcommand{\labelenumii}{\arabic{enumi}.\arabic{enumii}}


% General document settings
% % %
% Accepted file upload formats
\newcommand{\AcceptedFormats}{.txt, .csv, .xls, .xlsx}

\title{Техническое задание на смену системы}

\date{28/12/2016}

\begin{document}

\maketitle

\section{Описание проекта}
Рабочее название проекта - Кошер. Сайт представляет собой информационный портал Департамента Кашрута. Сайт уже работает в данный момент. Клиенту необходим редизайн, который произойдет со сменой стека технологии.

\subsection{Цели сайта для заказчика}

Клиенты могут получать на сайте информацию о кошерной продукции - о самих продуктах, их производителях, местах, где продукцию можно купить и ресторанах. Также клиенты могут получать эту же информацию через мобильное приложение.


\section{Технические уточнения}

\subsection{Термины}

        \begin{itemize}
        \item Блок - некий визуальный элемент, выделяющийся либо графически (в виде рамок, очертаний), либо по смыслу (совокупность похожих элементов)
        \item Компонент - часть содержания, имеющего закрытое визуально представление. Одна страница сайта может состоять из нескольких компонентов.
        \item Фронтэнд - для пользователя видимая оболочка веб-страницы
        \item Бэкэнд - невидимые для пользователя математические алгоритмы
        \item Фреймворк - набор правил и инструкций для определенного языка, которые облегчают разработку, задавая разработке направление.
        \item CMS - все работы происходят на основе системы управления содержанием - Yadadya CMS (Y.CMS)
        \item Модуль - является описанием общего функционала, который не может быть классифицирован как привязанный к определенной странице. Он может встречаться на любой странице в любом месте. Модуль может состоять из нескольких компонентов. Также модуль может содержать в себе части логики фронтэнда и бэкэнда.
        \item Хэдер - верхняя часть сайта, обладающая определенной структурой, которая видна сквозняком на всех или почти всех страницах сайта. Также используется обозначение "шапка".
        \item Футер - нижняя часть сайта. Функционал аналогичен хэдеру. Также используется обозначение "подвал".
        \item Администраторская панель ("админка") - часть сайта, скрытая для обычных пользователей, позволяющая редактировать содержимое сайта
        \item Публичная часть ("паблик") - часть сайта, доступная для просмотра обычным пользователям.
        \item Слайдер - блок с фотографиями и (опционально) текстом, которые можно листать (влево/вправо). Обычно размещается в верхней части страницы и занимает всю ширину экрана. Также возможны более узкие слайдеры.
        \item Шаблон - является уже сверстанным дизайном, т.е. представляет собой страницы с html разметкой и возможно некоторыми функциями JavaScript. Шаблон сам по себе не является рабочим сайтом.
        \item Мобильное приложение (или МП) - приложение для скачивания на смартфон. 

    \end{itemize}
    
\subsection{Технические требования к сайту}
Сайт располагается на хостинге или выделенном сервере (virtual dedicated server), в зависимости от предпочтений заказчика.



\section{Архитектура проекта}

\subsection{Общее}
Бэкэнд разрабатывается на языке Ruby и при использовании фреймворка Ruby on Rails, для фронтэнда используется язык JavaScript и фреймворк Angular2.
\\[0.5cm]
Проект изначально должен соблюдать правила объектно ориентированного программирования. Т.е. любой пользователь должен быть объектом класса "пользователь". Это распространяется на все сущности. Все действия системы должны быть методами своих классов. Псевдокод:
\begin{lstlisting}[language=Ruby]

#Class description

class User < ActiveRecord::Base 

def initialize(name, phone, mail)
	@name = name
	@phone = phone
	@mail = mail
end

# Register method
def register(user_params)
	User.create(user_params)
end

# Allow only whitelisted params
private
	def user_params
		params.require(:user).permit(:name, :phone, :mail)
	end
\end{lstlisting}

Программирование должно осуществляться по подходу TDD - test driven development. Это означает, что перед написанием функционала необходимо написать тест (который заведомо провалится, так как функционала нет). Далее пишется минимальный функционал, который позволяет пройти данный тест.
\\[0.5cm]
Псевдокод теста:
\begin{lstlisting}[language=Ruby]
require 'user'
	
Rspec.describe User, "#register" do
  context "when user is completely new" do
	it "creates a new entry with valid data" do
	  user = User.new
      user.name = "Vasya Pupkin"
	  user.phone = "8 903 227 88 74"
	  user.mail = "vasya@pupkin.ru"
	  user.register
				
	  expect(User.count).to change{User.count}.by(1)
	end
    it "does not create a new entry with invalid data" do
      user = User.new
	  user.name = "Vasya Pupkin"
	  user.phone = "j1lkh3i1u2y98"
	  user.mail = "someinvalidtext"
	  user.register	
		
      expect(User.count).not_to change{User.count}.by(1)
	end
  end
end
\end{lstlisting}
Т.е. в данном случае мы тестируем создание нового пользователя с валидными данными (первый случай) и с невалидными данными. В первом случае мы ожидаем увеличение общего кол-ва пользователей на 1, во втором случае мы не ожидаем увеличений.

\subsection{Модели}
Общее описание - есть каталог товаров, товары разделены по категориям.

\begin{itemize}
	\item Категория - имеет поля: название, описание, картинка
	\item Подкатегория - имеет поля: категория, название, описание, картинка
	\item Производитель - имеет поля: название, описание, картинка
	\item Торговая марка - имеет поля: название, описание, картинка
	\item Продукт - имеет поля: категория, подкатегория, название, описание, картинка (может быть несколько), штрихкод
	\item Знак кашрута (первый уровень) - имеет поля: название, описание, картинка
	\item Знак кашрута (второй уровень) - имеет поля: название, описание, картинка, знак кашрута первого уровня
	\item Новость - имеет поля: название, описание, картинка
	\item Город - имеет поля: название, картинка
	\item Магазин - имеет поля: название, описание, картинка, адрес, телефон, время открытия, сайт, город
	\item Ресторан - имеет поля: название, описание, картинка, адрес, телефон, время открытия, сайт, город
	\item Категория рецептов - имеет поля: название, описание, картинка	
	\item Рецепт - имеет поля: название, описание, картинка (несколько), категория рецептов
	\item ЧаВо (часто задаваемые вопросы или FaQ) - имеет поля: вопрос, ответ
	\item Партнер - имеет поля: название, описание, картинка, сайт
	\item Слайд - имеет поля: картинка
	\item Настройки - имеет поля: адрес, телефон, эл.почта
\end{itemize}

\subsubsection{Структура моделей}

Продукт принадлежит одной торговой марке, одному производителю, одной подкатегории и одной категории.
В категории, подкатегории, производителе и торговой марке может быть неограниченное кол-во продуктов.
\\[0.5cm]
Торговая марка принадлежит одному производителю, одной подкатегории и одной категории.
В производителе, подкатегории и категории может быть неограниченное кол-во торговых марок.
\\[0.5cm]
Производитель принадлежит одной подкатегории и одной категории.
В категории и подкатегории может быть неограниченное кол-во производителей.
\\[0.5cm]
Подкатегория принадлежит одной категории.
В категории может быть неограниченное кол-во подкатегорий


\subsubsection{Структура ссылок}
Сайт имеет следующую структуру ссылок:
site.name/category/subcategory/supplier/trademark/kosher-product
например
kosher.ru/алкогольные-напитки/бурбон/русский-север/ледокол/бурбон-веселый-роджер

\subsection{Языки}
Сайт должен поддерживать два языка - русский и английский. Основным языком является русский. При нажатии на флаг английской версии пользователь перенаправляется на специальный англоязычный лендинг.


%% New Section
\section{Модули}

\subsection{Поиск}
Поиск производится по следующим сущностям и их параметрам:
\begin{itemize}
	\item Продукты - название, описание
	\item Торговая марка - название, описание 
	\item Знаки кашрута - название, описание
	\item Рецепт - название, описание
	\item Магазин - название, описание
	\item Ресторан - название, описание
	\item Новость - название, описание
\end{itemize}
На странице выдачи поиска (SERP = search eninge result page) ссылки ведут на соответствующую страницу.

%% New Section
\section{Страницы}

\subsection{Структура страниц}
Структура страниц для всех сущностей одинаковая - есть страница со списком всех записей (index или индексная страница). Одна запись с этой страницы ведет на детальную страницу (show или шоу). Также имеются статичные страницы, такие как - главная, контакты.

\subsubsection{Верхнее меню}


Состоит из следующих элементов:
\begin{itemize}
	\item О департаменте - статичная страница
	\item Новости
	\item Фотогалерея
	\item FAQ
	\item Контакты
\end{itemize}
Также в верхнем меню доступна форма поиска и форма смены языка.

\subsubsection{Боковое меню}

Состоит из следующих элементов:
\begin{itemize}
	\item Кошерные продукты
	\item Кошерные магазины
	\item Кошерные рестораны
	\item Кошерные рецепты
	\item Знаки кошерности
	\item Вопросы/ответы
	\item Наши партнеры
\end{itemize}


\subsubsection{Нижнее меню}
Состоит из равномерно распределенных блоков (например col-md-4, col-md-4, col-md-4). 
\\[0.5cm]
Первый блок:
\begin{itemize}
	\item О департаменте - статичная страница
	\item Новости (пресса) - ссылка на страницу типа "blog"
	\item Фотогалерея
	\item FAQ
	\item Контакты
\end{itemize}

Второй блок содержит контакты.

Третий блок содержит форму для подписки на новостную рассылку.


\subsection{Главная}
Главная страница состоит из следующих блоков:
\begin{itemize}
	\item Шапка - содержит в себе главные данные (лого, телефон, название сайта)
	\item Слайдер (создается как отдельный элемент в админке)
	\item Последние новости департамента
	\item Вводный текст
\end{itemize}

\subsection{Странца продуктов (индекс)}
\begin{itemize}
	\item На данной странице идет по умолчанию список всей продукции. В фильтрах присутствуют - категории, подкатегории, производители и ТМ.
\end{itemize}

\subsection{Странца продукта (шоу)}
\begin{itemize}
	\item На данной странице показываем информацию о продукте - картинку, описание
\end{itemize}


\subsection{Список магазинов (индекс)}
Список магазинов представляет собой перечень городов, нажав на который подгружается список магазинов данного города. В списке видны следующие данные:
\begin{itemize}
	\item Название магазина
	\item Логотип магазина
	\item Короткое описание
	\item Время открытия 
	\item Адрес
\end{itemize}

Нажав на конкретный магазин, пользователь попадает на детальную страницу магазина.

\subsection{Страница магазина (шоу)}
\todo[inline]{Она вообще нужна?}

\subsection{Список ресторанов (индекс)}
Полная аналогия с магазинами.

\subsection{Страница ресторана (шоу)}
\todo[inline]{Она вообще нужна?}


\subsection{Страница рецептов (индекс)}
Список рецептов представляет собой перечень рецептных категорий. Нажав на рецепт, пользователь попадает на детальную страницу рецепта.

\subsection{Страница рецепта (шоу)}
Выводим информацию о рецепте - наименование, описание, картинки рецепта

\subsection{Знаки кошерности (индекс)}
Список знаков кашрута первого уровня. В списке видны следующие данные:
\begin{itemize}
	\item Картинка
	\item Название
	\item Короткое описание
\end{itemize}
Нажав на знак, пользователь попадает на детальную страницу.

\subsection{Знак кашрута второго уровня (шоу)}
На данной странице сверху видна информация о знаке кашрута первого уровня (картинка, название, полное описание). Ниже идут списком все знаки Кашрута второго уровня, которые принадлежать первому уровню. Нажать на них нельзя. В списке видна следующая информация:
\begin{itemize}
	\item Картинка
	\item Название
	\item Описание
\end{itemize}

\subsection{Вопросы/ответы}
Перечень вопросов с их ответами. Детальной страницы нет.

\subsection{Наши партнеры}
Перечень партнеров (логотип, название, описание, сайт). Детальной страницы нет.

\subsection{Типовые страницы}

\begin{itemize}
	\item Контакты - необходимо подгружать из настроек 
	\item О департаменте - текстовая страница
	\item Новости - управляется соответствующим блоком в администраторской панели
	\item Обратная связь - контактная форма, присылает сообщение на адрес из настроек
\end{itemize}

\end{document}








