\documentclass[DIV=calc, paper=a4, fontsize=11pt]{scrartcl} % Документ принадлежит классу article, а также будет печататься в 12 пунктов.
\usepackage{ucs}
\usepackage[T1,T2A]{fontenc}
\usepackage[utf8x]{inputenc} % Включаем поддержку UTF8
\usepackage[russian]{babel} % Пакет поддержки русского языка
\usepackage{titling} % Allows custom title configuration

%for frames
\usepackage{framed}

%For image using
\usepackage{graphicx}

%Numbering subsubsubsections etc
\setcounter{secnumdepth}{5}

%For code highlightning
\usepackage{listings}

%Further enumeration
\usepackage{enumitem}
\setenumerate[1]{label=\theparagraph.\arabic*.}
\setenumerate[2]{label*=\arabic*.}
\setenumerate[3]{label*=\arabic*.}


%For referencing within enumeration lists
\usepackage{enumitem}

%Packages for word-like comment style
\usepackage{todonotes}

%Package for images
\usepackage{float}
\floatstyle{boxed}
\restylefloat{figure}

%For a nicer reference
\usepackage{fancyref}

\usepackage{titlesec}


\titleformat*{\section}{\LARGE\bfseries}
\titleformat*{\subsection}{\Large\bfseries}
\titleformat*{\subsubsection}{\large\bfseries}
\titleformat*{\paragraph}{\large\bfseries}
\titleformat*{\subparagraph}{\large\bfseries}

%For some math formulas if needed
\usepackage{mathtools}

% Some nice visualization
%\usepackage[svgnames]{xcolor} % Enabling colors by their 'svgnames'
\usepackage{fullpage}
%\renewcommand{\headrulewidth}{0.0pt} % No header rule
%\renewcommand{\footrulewidth}{0.4pt} % Thin footer rule
% End visualization

%smart enumeration
\renewcommand{\labelenumi}{\arabic{enumi}.}
\renewcommand{\labelenumii}{\arabic{enumi}.\arabic{enumii}}


\title{Система для учета магазина - shopmate (рабочее название)}
\date{15/04/2015}

\begin{document}

\maketitle

\section{Описание проекта}
Система shopmate позволяет вести оперативную деятельность (продуктового) магазина, начиная с оформлени поставки и заканчивая фактической продажей. Shopmate делится на модули, каждый из которых может работать автономно, но раскрывает весь свой потенциал только при взаимодействии с другими модулями. 

\section{Технические уточнения}

\subsection{Термины}

        \begin{itemize}
        \item Блок - некий визуальный элемент, выделяющийся либо графически (в виде рамок, очертаний), либо по смыслу (совокупность похожих элементов)
        \item Компонент - часть содержания, имеющего закрытое визуально представление. Одна страница сайта может состоять из нескольких компонентов.
        \item Фронтэнд - для пользователя видимая оболочка веб-страницы
        \item Бэкэнд - невидимые для пользователя математические алгоритмы
        \item CMS - все работы происходят на основе системы управления содержанием - CMS 1C Bitrix (1С Битрикс)
        \item Модуль - является описанием общего функционала, который не может быть классифицирован как привязанный к определенной странице. Он может встречаться на любой странице в любом месте. Модуль может состоять из нескольких компонентов. Также модуль может содержать в себе части логики фронтэнда и бэкэнда.
        \item Хэдер - верхняя часть сайта, обладающая определенной структурой, которая видна сквозняком на всех или почти всех страницах сайта. Также используется обозначение "шапка".
        \item Футер - нижняя часть сайта. Функционал аналогичен хэдеру. Также используется обозначение "подвал".
        \item Тетрадка - рабочее и финальное название для кабинета салона красоты.
        \item Зеркало - рабочее название для сайта, который доступен клиенту (позже этот сайт будет по-умолчанию доступен по адресу getbam.ru)
    \end{itemize}


\subsection{Технические требования к сайту}
Сайт должен быть адаптивным, поддерживать IE 8+ и использовать технологию композита.
Тут ещё больше технического бла-бла.


\section{Модули}

Каждый модуль должен обладать индивидуальными правами доступа.

\subsection{Товары/Склад}

Данный модуль позволяет заносить новые товары, редактировать существующие, удалять (т.е. стандартные CRUD действия). Занесение и просмотр товаров вынесены отдельно, так как подразумевают специфичную логику. Например занесение должно в более поздних релизах работать через считывание штрих-кодов. Для хранения данных о товаре должен использоваться стандартный компонент товаров и складской системы Битрикс-магазина для того, что бы в любой момент можно было к системе shopmate подключить интернет-продажу (тем более оффлайн-продажа отличается только одной опцией от интернет-магазина - оплатой наличными) 

\subsubsection{Занесение новых товаров}

\paragraph{Релиз Альфа}
Пользователь может занести товар вручную, открыв стандартную веб-форму и введя все необходимые параметры товара.
Параметры товара:
\begin{itemize}
	\item Цена на продажу
	\item Название - должны храниться в отдельном справочнике, который делится на категории и подкатегории (что бы не было "Кока-Кола", "Кока кола", "Coca Cola")
	\item Закупочная цена
	\item Сколько мы заплатили по факту
	\item Кол-во товара
\end{itemize}
При занесении цены у пользователя должен быть следующий выбор:
\begin{itemize}
	\item Использовать старую цену (если такой товар уже есть)
	\item Ввести новую цену и обновить эту цену для всех уже существующих товаров. При этом новую цену можно ввести в виде двух параметров (на выбор) - в качестве накидки процентов на закупочную цену или в свободной форме 
\end{itemize}
При этом дефолтно просто загружается текущая цена и если пользователь ничего не меняет, то при сохранении срабатывает сценарий 1. Если пользователь вводит цену в абсолютном значении или в процентуальном прибавлении/убавлении, то срабатывает второй сценарий.

Для каждого товара также необходимо предусмотреть индивидуальные прайслисты клиента. Т.е. индивидуальный прайслист всегда приоритетнее всех остальных ценовых настроек.
\paragraph{Релиз Бета}
Необходимо подключить функционал считывание штрих-кодов, а также занесения товаров при принятии товаров от поставщика.\todo[inline]{Тут очень важно ещё уточнить как это в действительности происходит. У складовщика скорее всего  будет просто считыватель штрих-кодов? Также скорее всего он должен будет просто принять данные (считать коды), а потом у себя на компьютере увидеть список товаров, которые он принял и начать их обрабоку}

\paragraph{Релиз Гамма}

\subsubsection{Просмотр товаров}

Просмотр товара осуществляется в стандартном табличном виде с делением на категории и подкатегории. При нажатии на товар можно просмотреть его детальные параметры, а также удалить товар. 

\paragraph{Скоропортящийся товар}
Важно вынести отдельным элементом (по сути это будет простой предзаготовленный фильтр) - скоропортящиеся продукты. Нажав на данный фильтр видны все скоропортящиеся товары, отсортированные по дате истечения срока годности (самая ближайшая дата в самом начале).

\paragraph{Релиз Бета или Гамма}
Добавляем просмотр в виде плиток.

\subsubsection{Редактирование товара}

Для редактирования доступны не только сами товары, но и опции для категорий и подкатегорий. Так например нужно иметь возможность менять цены для всей категории товаров (добавить 30\% например). Также очень важно для каждого поля (параметра) товара иметь отдельные права на редактирование. Параметры, доступные для просмотра/редактирования:
\begin{itemize}
	\item Название товара (только через выпадающий список)
	\item Количество - только администратор может менять это поле! Обязательно нужна проверка на права!
	\item Цена на продажу
	\item Закупочная цена - как и при занесении товара, тут можно оставить как есть либо обновить (опять же - либо абсолютным, либо процентуальным значением)
	\item Задолженность по товару - только админ может редактировать
	\item Все выплаты по товару - только админ
	\item Описание товара
\end{itemize}

На этой же странице доступна история изменений в товаре.

\subsubsection{История изменений товара}
История фиксирует все возможные манипуляции с конкретным товаром. Приход, уход (продажу), изменение параметров, все фиксируется и выводится в формате:
\\[0.5cm](дата)(значение)(действие)(было)(стало)
\\[0.5cm]Например: 17/04/2015 - количество - приход - было: 0 - стало: 640
Для истории для каждого поля нужно хранить два его значения (действие). Так, для количества это может быть - приход (мы получаем товар) и продажа (мы расстаемся с товаром). Для задолженности это будут - оплата (мы платим по счетам) и кредиторка (мы откладываем платеж, задолженность растет).

\subsection{Касса}

Данный модуль позволяет оформить заказ и оплатить его в режиме оффлайн. Для создания заказа у данного модуля есть свой интерфейс, позволяющий продавцу искать необходимые товары и добавлять их в виртуальную корзину. После того, как продавец перешел к оплате он выбирает метод "наличные" или "банковской картой" и в зависимости от выбора срабатывает драйвер кассового аппарата и открывает кассу либо драйвер устройства, которое считывает данные банковской карты.
\\[0.5cm]
Для поиска товара продавцу нужен простой интерфейс - поиск товара по названию, а также вывод товаров простым списком, поделенным на категории/подкатегории. Методы оплаты на данный момент:
\begin{itemize}
	\item Наличный расчет
	\item Оплата банковской картой
	\item Безналичный расчет (выставляем счет)
\end{itemize}

Касса ни в коем случае не должна иметь возможность пробивать больше товара, чем находится в модуле "товары/склад".
\paragraph{Идея для релиза Бета/Гамма}
Может быть здесь сразу сделать вывод шаблона интернет-магазина? Там есть и поиск и вывод товара по категориям/подкатегориям.

\subsection{Финансы}

Финансы представляют собой обзор дебиторской и кредиторской задолженностей, а также набор отчетов.

\subsubsection{Дебиторка/Кредиторка}

В табличной форме показываем пользователю задолженность фирмы перед поставщиками и задолженность третьих лиц перед магазином.

\paragraph{Дебиторка(должны нам)}
Рассчитывается следующим образом - товар ушел, денег нет (например продажа по счету).

\paragraph{Кредиторка(должны мы)}
Рассчитываем так - товар оприходован, но не вся сумма оплачена (разница между полями "закупочная стоимость" и "фактическая оплата")

\paragraph{Отчеты}
\subparagraph{Доходы}
Все кассовые операции, которые являются фактическими оплатами

\subparagraph{Расходы}
Расходы представляют собой как отчетность, так и настройки по расходам. В настройках по расходам можно заносить фиксированные платежи (аренда, коммунальные услуги...). По умолчанию в расходы попадают закупочные стоимости товаров.

\subsection{Сотрудники}

Помимо стандартных CRUD действий для самих сотрудников имеем: настройки заработной платы и бонусы. Набор настроек для зарплаты:

\begin{itemize}
	\item Фиксированная часть
	\item Проценты
	\item Период выплат
\end{itemize}

Для бонусов имеем следующие настройки:

\begin{itemize}
	\item Опция - с общих продаж или с группы товаров (например только с продаж мяса)
\end{itemize}

\subsection{Клиенты}

По аналогии с сотрудниками - базовые CRUD действия для сущности "клиент". Помимо этого имеем привязку прайслиста к конкретному клиенту.

\subsubsection{Индивидуальный прайслист}
При создании прайслиста можно указать две опции - фикс. цены на каждый продукт или скидки в процентах. При этом для скидок работает ступенчатая модель - приоритетнее всего скидка на индивидуальный товар, за ней скидка на группу товаров, далее скидка на все товары.
Пример - на весь товар клиенту ООО Ромашка даются 5\% скидка. На алкогольную продукцию даются 10\%. Соответственно 10 процентов на алкоголь превалиируют над 5 процентами на все товары в разделе "алкоголь". Скидки не суммируются.

\subsection{Задачи}

На данный момент просто устанавливаем битрикс-чат.

\subsection{Производство}

Данный модуль подразумевает собственный интерфейс (аналогично играм вроде "Собери Пиццу"), который позволяет пользователю изъять из оборота (зарезервировать для себя) определенные товары и на выходе получить новый товар.

Пример - изымаем 200 кг мяса, 100 кг муки и 100 шт. яиц. На выходе получаем 1000 бургеров. 

\subsubsection{Процесс изъятия}

Для начала пользователь-производственник подает заявку товароведу на изъятие определенного перечня товара (в заявке указываем причину, конечный продукт, список ингридиентов). Эта заявка находится переходит несколько статусов - "ожидание", "принято", "отклонено (с причиной)", "необходимо уточнение".

\subsection{Поставщики}

Самый обычный список фирм-поставщиков со стандартными CRUD-действиями.

\subsection{Группы пользователей}

Для разделения прав важно - каждый модуль обладает своими правами доступа. Например работнику производства нельзя давать доступ к модулю "касса", а продавцу к модулю "производство".
Для товаров важно - разделение прав по категориям товаров. Например сотрудникам мясного цеха нельзя давать доступ к молочной продукции.

\end{document}