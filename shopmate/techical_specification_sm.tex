\documentclass[DIV=calc, paper=a4, fontsize=11pt]{scrartcl} % Документ принадлежит классу article, а также будет печататься в 12 пунктов.
\usepackage{ucs}
\usepackage[T1,T2A]{fontenc}
\usepackage[utf8x]{inputenc} % Включаем поддержку UTF8
\usepackage[russian]{babel} % Пакет поддержки русского языка
\usepackage{titling} % Allows custom title configuration

%for frames
\usepackage{framed}

%For image using
\usepackage{graphicx}

%Numbering subsubsubsections etc
\setcounter{secnumdepth}{5}

%For code highlightning
\usepackage{listings}

%Further enumeration
\usepackage{enumitem}
\setenumerate[1]{label=\theparagraph.\arabic*.}
\setenumerate[2]{label*=\arabic*.}
\setenumerate[3]{label*=\arabic*.}


%For referencing within enumeration lists
\usepackage{enumitem}

%Packages for word-like comment style
\usepackage{todonotes}

%Package for images
\usepackage{float}
\floatstyle{boxed}
\restylefloat{figure}

%For a nicer reference
\usepackage{fancyref}

\usepackage{titlesec}


\titleformat*{\section}{\LARGE\bfseries}
\titleformat*{\subsection}{\Large\bfseries}
\titleformat*{\subsubsection}{\large\bfseries}
\titleformat*{\paragraph}{\large\bfseries}
\titleformat*{\subparagraph}{\large\bfseries}

%For some math formulas if needed
\usepackage{mathtools}

% Some nice visualization
%\usepackage[svgnames]{xcolor} % Enabling colors by their 'svgnames'
\usepackage{fullpage}
%\renewcommand{\headrulewidth}{0.0pt} % No header rule
%\renewcommand{\footrulewidth}{0.4pt} % Thin footer rule
% End visualization

%smart enumeration
\renewcommand{\labelenumi}{\arabic{enumi}.}
\renewcommand{\labelenumii}{\arabic{enumi}.\arabic{enumii}}


\title{Система для учета магазина - shopmate (рабочее название)}
\date{15/04/2015}

\begin{document}

\maketitle

\section{Описание проекта}
Система shopmate позволяет вести оперативную деятельность (продуктового) магазина, начиная с оформлении поставки и заканчивая фактической продажей. Shopmate делится на модули, каждый из которых может работать автономно, но раскрывает весь свой потенциал только при взаимодействии с другими модулями. 

\section{Технические уточнения}

\subsection{Термины}

        \begin{itemize}
        \item Блок - некий визуальный элемент, выделяющийся либо графически (в виде рамок, очертаний), либо по смыслу (совокупность похожих элементов)
        \item Компонент - часть содержания, имеющего закрытое визуально представление. Одна страница сайта может состоять из нескольких компонентов.
        \item Фронтэнд - для пользователя видимая оболочка веб-страницы
        \item Бэкэнд - невидимые для пользователя математические алгоритмы
        \item CMS - все работы происходят на основе системы управления содержанием - CMS 1C Bitrix (1С Битрикс)
        \item Модуль - является описанием общего функционала, который не может быть классифицирован как привязанный к определенной странице. Он может встречаться на любой странице в любом месте. Модуль может состоять из нескольких компонентов. Также модуль может содержать в себе части логики фронтэнда и бэкэнда.
        \item Хэдер - верхняя часть сайта, обладающая определенной структурой, которая видна сквозняком на всех или почти всех страницах сайта. Также используется обозначение "шапка".
        \item Футер - нижняя часть сайта. Функционал аналогичен хэдеру. Также используется обозначение "подвал".
        \item Тетрадка - рабочее и финальное название для кабинета салона красоты.
        \item Зеркало - рабочее название для сайта, который доступен клиенту (позже этот сайт будет по-умолчанию доступен по адресу getbam.ru)
    \end{itemize}


\subsection{Технические требования к сайту}
Сайт должен быть адаптивным, поддерживать IE 8+ и использовать технологию композита.
Тут ещё больше технического бла-бла.


\section{Модули}

Каждый модуль должен обладать индивидуальными правами доступа.

\subsection{Дашборд (Главная)}
На главной странице доступна сводка для текущего пользователя. \todo[inline]{Важно определиться с набором оповещений, на основе уже существующих модулей}

Пример возможных оповещений:

\begin{itemize}
	\item Магазин принял накладную, не полностью оплатив весь товар - напоминание должно попадать в дашборд
	\item Клиент оплатил товар через безналичный расчет \todo[inline]{Что тут выводим в качестве напоминания? Сразу после оплаты? Или когда есть факт просрочки оплаты? При поступлении денег напоминаем?}
\end{itemize}

Важно составить требования к дополнительным возможным напоминаниям.

\subsection{Товары/Склад}

Данный модуль позволяет просматривать наличие товаров, а также заносить приход новых партий товара ("оприходывание"). Для хранения данных о товаре должен использоваться стандартный компонент товаров и складской системы Битрикс-магазина для того, что бы в любой момент можно было к системе shopmate подключить интернет-продажу (тем более оффлайн-продажа отличается только одной опцией от интернет-магазина - оплатой наличными). Важно разделять просмотр товаров (данная часть программы представляет собой список уже занесенных товаров) и оприходывание товара (поступает новая партия товара). При оприходывании я заношу параметры конкретной партии (поставщика, закупочную цену и т.д.).

\paragraph{Справочник товаров} \label{new_goods}
На стороне суперадминки (битриксовая админка) имеем справочник товаров, состоящий из:

\begin{itemize}
	\item Название товара
	\item Штрихкод товара
	\item Упаковка товара
	\item Единица измерения
\end{itemize}

Упаковка отличается от единицы измерения. Например товар может быть Кока-Колой, единица измерения - 1шт, упаковка - 0.5л.
\\[0.5cm]
Магазин не имеет доступа к этому справочнику, не может редактировать значения. Если он оприходует товар, которого нет в справочнике, то данная накладная сохраняется и одновременно отправляется оповещение суперадмину, что было добавлено новое значение в справочник. Суперадмин должен аппрувить это значение (он может либо подтвердить сразу, либо отредактировав - в таком случае товар у магазина отобразится уже с новым названием), только после этого оно станет доступным всем остальным пользователям для поиска.

\begin{framed}
\paragraph{Комментарий от клиента}
	Необходимо стандартизировать название товаров для всех магазинов изначально. Это очень важный пункт. Перед запуском продаж мы зальем в систему большой справочник, который виден только нам (100т наименований), и когда магазин сканирует штрих-код для добавления товара к себе в систему, то наш справочник уже дает ему все данные по этому товару. Другими словами, при добавлении товара в свой справочник, все, что нужно сделать товароведу - это отсканировать штрих-код. Если товара вдруг не окажется в нашем справочнике, то нужно дать возможность отправить нам отсканированный штрих-код на модерацию, после которой товар можно будем добавить
\end{framed}


\subsubsection{Оприходывание новых товаров}

\paragraph{Релиз Альфа}
Если приходит новая партия товара, то пользователь заносит этот товар через накладную, одна накладная может содержать одну или несколько позиций товаров с определенными параметрами. Накладных может быть неограниченно количество и создаваться они будут часто.
\\[0.5cm]
Первично пользователь должен внести название накладной (добавляем маску, что бы он понимал как заносить данные) и дату (по умолчанию подгружаем текущую дату, но также даем возможность поменять (опять же, с маской)). Далее пользователь выбирает поставщика (текстовое поле с саджестом - данные берутся из таблицы "поставщики" конкретного магазина). После этого в каждой строке он может занести одну партию товара. Каждая строка состоит из определенных параметров товара.

Параметры товара:

\begin{itemize}
	\item Штрихкод (можно занести вручную, а также с помощью сканнера). Если человек занес штрихкод, то базовые параметры (название) должны подгружаться автоматически. Если он занес штрихкод, но система не нашла такого товара, то должна предложить ему создать новый товар (открываем форму в попапе).
	\item Название - хранятся в справочнике (см. пункт \ref{subsubsec:new_goods}), при занесение должен работать саджест (или подгружаем уже по штрихкоду)
	\item Упаковка - либо выбираем из справочника, либо подгружаем по штриходу из параметров товара
	\item Единица измерения - либо выбираем из справочника, либо подгружаем по штриходу из параметров товара
	\item Кол-во по накладной (должно соответствовать в точности бумажному документу)
	\item Кол-во по факту (если пришло меньше (или больше), то нельзя это фиксировать в поле "кол-во по накладной", нужно обязательно отдельно вынести), по умолчания дублируем значение "кол-во по накладной" - поле можно редактировать
	\item Цена на продажу (см. отдельный пункт ниже)
	\item Закупочная цена
	\item Сумма с НДС - значение рассчитываем на основе параметра поставщика (18\% или 0\%)
	\item Размер НДС - как и с суммой включая НДС, эти поля нередактируемы, а рассчитываются из параметров товара.
	\item Дату истечения срока годности, если товар скоропортящийся
\end{itemize}

Для всех значений, которые подгружаются после считывания штрихкода необходимо запретить редактирование (так как подгружаются значения из справочника).

При занесении цены у пользователя должен быть следующий выбор:
\begin{itemize}
	\item Использовать старую цену (если такой товар уже есть) - стандартный функционал.
	\item Ввести новую цену и обновить эту цену для всех уже существующих товаров. При этом новую цену можно ввести в виде двух параметров (на выбор) - в качестве накидки процентов на закупочную цену или в свободной форме 
\end{itemize}
При этом дефолтно просто загружается текущая цена. Но также у пользователья есть возможность ввести новую цену в абсолютном значении или в процентуальном прибавлении/убавлении. Для сохранения первого сценария нужно просто оставить поле с дефолтным значением и ничего не делать. 
\\[0.5cm]
Для изменения цены пользователь может сразу в обзоре позиций накладной занести абсолютное значение, либо нажав на "+" открыть всплывающее окно, которое дает несколько вариантов на выбор:

\begin{itemize}
	\item Оставить все как есть (если пользователь нажал по ошибке)
	\item Накинуть процент - вводим значение процента...
	\item ...и выбираем к чему накинуть процент:
	\item на закупочную цену текущий поставки
	\item на самую дорогую закупку
	\item на среднезакупочную цену
\end{itemize}


Для каждого товара также необходимо предусмотреть индивидуальные прайслисты клиента. Т.е. индивидуальный прайслист всегда приоритетнее всех остальных ценовых настроек.
\paragraph{Релиз Бета}
Необходимо подключить функционал считывание штрих-кодов, а также занесения товаров при принятии товаров от поставщика.\todo[inline]{Тут очень важно ещё уточнить как это в действительности происходит. У складовщика скорее всего  будет просто считыватель штрих-кодов? Также скорее всего он должен будет просто принять данные (считать коды), а потом у себя на компьютере увидеть список товаров, которые он принял и начать их обрабоку}

\paragraph{Релиз Гамма}

Необходимо иметь возможность предоплачивать накладные еще до их поставки, как это будет, поймем позже.

\subsubsection{Просмотр товаров}

Просмотр товара осуществляется в стандартном табличном виде с делением на категории и подкатегории. При нажатии на товар можно просмотреть его детальные параметры, а также удалить товар. На странице просмотра доступны следующие поля:

\begin{itemize}
	\item Название товара
	\item Общее кол-во на складе
	\item Среднезакупочная цена (рассчитываем так - сумма всех закупок поделенная на кол-во всех закупок)
	\item Текущая цена на продажу
	\item Срок годности (если есть)
	\item Баланс - отражает стоимость всех складских запасов товара	
\end{itemize}


\paragraph{Скоропортящийся товар}
Важно вынести отдельным элементом (по сути это будет простой предзаготовленный фильтр) - скоропортящиеся продукты. Нажав на данный фильтр видны все скоропортящиеся товары, отсортированные по дате истечения срока годности (самая ближайшая дата в самом начале). Если по конкретному скоропорту занесено несколько накладных, то в общем обзоре берем самую ближайшую дату истечения срока годности.

\paragraph{Релиз Бета или Гамма}
Добавляем просмотр в виде плиток.

\subsubsection{Редактирование товара/карточка товара}

Также очень важно для каждого поля (параметра) товара иметь отдельные права на редактирование. Параметры, доступные для просмотра/редактирования:
\begin{itemize}
	\item Название товара (только через выпадающий список)
	\item Количество - только администратор может менять это поле! Обязательно нужна проверка на права!
	\item Цена на продажу
	\item Последняя закупочная цена за единицу товара (если товар новый и накладных не было, то оставляем поле пустым) 
	\item Закупочная цена - как и при занесении товара, тут можно оставить как есть либо обновить (опять же - либо абсолютным, либо процентуальным значением)
	\item Задолженность по товару - только админ может редактировать
	\item Все выплаты по товару - только админ
	\item Описание товара
\end{itemize}

На этой же странице доступна история изменений в товаре.

\paragraph{История изменений товара}
История фиксирует все возможные манипуляции с конкретным товаром. Приход, уход (продажу), изменение параметров, все фиксируется и выводится в формате:
\\[0.5cm](дата)(значение)(действие)(было)(стало)
\\[0.5cm]Например: 17/04/2015 - количество - приход - было: 0 - стало: 640
Для истории для каждого поля нужно хранить два его значения (действие). Так, для количества это может быть - приход (мы получаем товар) и продажа (мы расстаемся с товаром). Для задолженности это будут - оплата (мы платим по счетам) и кредиторка (мы откладываем платеж, задолженность растет).

\paragraph{Релиз Бета-Гамма}
Для редактирования доступны не только сами товары, но и опции для категорий и подкатегорий. Так например нужно иметь возможность менять цены для всей категории товаров (добавить 30\% например). 

\subsubsection{Накладные}
Отдельным от товаров табом находятся накладные. Здесь виден обзор всех накладных, которые были занесены через "оприходывание товара". Для обзора доступны следующие поля:

\begin{itemize}
	\item Номер накладной
	\item Кол-во занесенных в накладную товаров
	\item Общая закупочная цена по накладной
	\item Дата поступления накладной
	\item Сколько оплачено по накладной (берем значение, которое занес пользователь, делим его на общую закупочную сумму)
	\item Если в накладной есть скоропорт, какая самая ранняя дата истечения срока годности
\end{itemize}

Просмотр накладной полностью идентичен занесению новой накладной, только с разницей, что значения уже не редактируемы (очень важно!)

\paragraph{Фильтры для скоропорта}
Очень важно для накладных добавить две дефолтные опции фильтра - "показать только накладные со скоропортом" и "показать накладные с просроченным товаром". Данные виды подгружают все накладные со скоропортящимися товарами, а также сортируют их по дате истечения срока годности. 

\subparagraph{Просроченные товары}
Здесь необходимо вывести только товары, истечение срока годности которых уже в прошлом. Также подсвечиваем строки с таким товаром красным цветом.

\subparagraph{Просто скоропорт}
Здесь выводим все товары, срок годности у которых ещё не истек. Те товары, которые увядают в ближайшие три дня подсвечиваем темножелтым. В ближайшие 7 дней - бледножелтым.

\subsection{Касса}

Данный модуль позволяет оформить заказ и оплатить его в режиме оффлайн. Для создания заказа у данного модуля есть свой интерфейс, позволяющий продавцу искать необходимые товары и добавлять их в виртуальную корзину. После того, как продавец перешел к оплате он выбирает метод "наличные", "банковской картой" или "выставить счет" и в зависимости от выбора срабатывает драйвер кассового аппарата и открывает кассу либо драйвер устройства, которое считывает данные банковской карты. \todo[inline]{По поводу выставления счета что-то кроме печати оного нужно?}

Для поиска товара продавцу нужен простой интерфейс - поиск товара по названию, а также вывод товаров простым списком, поделенным на категории/подкатегории. Также форма поиска должна принимать штрихкоды. Методы оплаты на данный момент:
\begin{itemize}
	\item Наличный расчет
	\item Оплата банковской картой
	\item Безналичный расчет (выставляем счет)
	\item Комбинированный платеж
	\item Продажа в долг
\end{itemize}

Касса ни в коем случае не должна иметь возможность пробивать больше товара, чем находится в модуле "товары/склад".

\subsubsection{Оформление покупки}
Это стандартный вид модуля "Касса" - пользователь видит свой логин, рядом кнопку "закрыть кассу" и "сменить пользователя" \todo[inline]{Нужно понять, как именно закрывают кассу и меняют пользователя}
Далее система находится в режиме ожидания - выводим номер чека, и построчно информацию о товаре, которую пользователь (кассир) вбивает. При этом курсор всегда должен автоматически прыгать в поле "название" после занесения предыдущего товара (при открытии страницы, т.е. при виде по умолчанию курсор также должен быть в поле "название"). Для ввода товара кассиру доступны следующие действия:

\begin{itemize}
	\item Сканировать штрихкод с товара специальным сканером. При сканировании товара с одним и тем же штрихкодом должен увеличиваться счетчик общего кол-ва товаров в строке. При этом после сканирования в поле "название" попадает именно штрихкод.
	\item Вбить штрихкод вручную.
	\item Ввести название товара - поле "название" обладает саджестом, который должен искать по всей базе товаров \todo[inline]{Внимание, здесь нужно особенно внимательно подойти к разработке и максимально использовать поисковую индексацию - см. как подключить Sphinx для работы с Битриксом}
\end{itemize}

Для строки товара, в которой находится на данный момент кассир можно простым нажатием на соответствующие кнопки изменить кол-во товара. Уже оформленные строки менять нельзя, можно только удалить и добавить новую строку.
\\[0.5cm]
После оформления всех товарных позиций кассир может внести клиентскую информацию (опционально). По умолчанию доступно только поле для поиска клиента и если оно не оформлено, то покупка попадает к типу клиента "ритейл". 

\paragraph{Поиск клиента}
Если оформляется заказ для постоянного клиента (или например для компании - и оплата безналом), то кассир может по следующим параметрам искать клиента:

\begin{itemize}
	\item ФИО (ищем как и в физиках, так и в юриках (в юриках должно быть контактное лицо))
	\item Название компании
	\item Телефон
\end{itemize}

Саджест должен быть по форме:
\\COMPANY-NAME - CONTACT-PERSON-NAME - PHONE(optional)

\paragraph{Оплата заказа}
После выбора метода оплаты происходят различные действия. Если выбрана оплата наличными, то кассир нажимает на соответствующую кнопка и открывается всплывающее окно, которое отображает сумму чека, позволяет вбить размер купюры, которую дал клиент и отправляет запрос на драйвер, который открывает кассовый аппарат.
\\[0.5cm]
Если выбрана оплата картой, то срабатывает интерфейс с прибором эквайринга. Данный интерфейс должен (по аналогии с онлайн оплатой через эквайринг) посылать запрос на оплату и принимать статус (оплата прошла/отклонена).
\\[0.5cm]
Для метода оплаты через безналичный расчет просто показываем форму счета, которую можно распечатать \todo[inline]{Мб с опцией отправить по почте?}

\subparagraph{Комбинированный платеж}
Данное окно позволяет разбить сумму платежа на различные методы оплаты. Например из суммы чека в 1000 рублей 500 оплачиваются наличными, 500 картой. Имеем комбинированный интерфейс.

\subparagraph{Продажа в долг}
Данное окно предоставляет возможность продать товар в долг. При этом кассир должен выбрать пользователя из таблицы "клиенты" (поиск с саджестом) или создать нового клиента. Только после этого он может оформить чек. Чек при этом должен помечаться особым статусом ( = "продажа в долг" ). Для оформления нового клиента на момент продажи товара в долг обязательны следующие поля:

\begin{itemize}
	\item Фамилия
	\item Имя
	\item Отчество
	\item Адрес
	\item Номер паспорта
\end{itemize}


\paragraph{Завершение заказа}
Необходимо точно определить, что должно происходить с отмененным заказом. Наверное не удаляем его полностью, а сохраняем, помечаем как "удален".
\\[0.5cm]
В любом случае необходимо обновлять кол-во товаров (либо уменьшать, если оплата успешная или ничего не делать, если оплату отменили).



\paragraph{Идея для релиза Бета/Гамма}
Может быть здесь сразу сделать вывод шаблона интернет-магазина? Там есть и поиск и вывод товара по категориям/подкатегориям.

\subsubsection{Обзор покупок}
Данная страница отображает все покупки (чеки и счета) за управляемый период времени (по умолчанию это текущий день). Для просмотра доступны следующие поля:

\begin{itemize}
	\item ID клиента
	\item ФИО клиента или название компании-клиента
	\item телефон
	\item сумма покупки 
	\item дата покупки
\end{itemize}

Для каждого фильтра должны присутствовать свои собственные настройки доступа. Так, например, рядовой кассир может видеть только чеки своей смены (и не видит фильтр "за период" и "выберите дату от... до...").

\paragraph{Просмотр покупки}
Каждую покупку (чек/счет) можно просмотреть, нажав в строке покупки на "просмотреть". При этом открывается детальный обзор всех товаров данной покупки. Доступны следующие данные:

\begin{itemize}
	\item Номер чека
	\item Наименование товара
	\item Кол-во товара
	\item Цена за штуку товара
	\item Общая сумма по каждому товару
	\item Общая сумма чека
\end{itemize}

Если покупка была сделана клиентом в долг, то на данном экране также можно погасить долг. При этом "оформить чек" также как и при стандартной покупке открывает выбор оплаты (только "в долг" на этот раз нет) и дает возможность получить деньги наличным или иным расчетом. После этого чек меняет свой статус на "оплачен".

\subparagraph{Возврат чека}
При нажатии на "вернуть позиции чека" открывается окно, в котором кассир может выбрать возврат на весь чек (на все позиции данного чека) либо вернуть только отдельные позиции, кол-во которых он может указать в данном окне. При этом часть товара, который вернули необходимо прибавить обратно к количеству доступного товара на складе. Чек также помечаем статусом "возврат".

\subsection{Финансы}

Таб финансы представляет собой короткую сводку (некий финансовый dashboard), которая показывает прибыль текущего месяца, оборот, закупки, расходы на зарплату и прочие расходы текущего месяца. Рассчитывается так:

\begin{itemize}
	\item Оборот - все продажи месяца (все уходы товара и приходы денег через модуль "касса")
	\item Закупки - все оприходывания товаров месяца (модуль "товары" -> оприходывание)
	\item Расходы на зарплату - все затраты из модуля "сотрудники"
	\item Прочие расходы - расходы из модуля "финансы"
	\item Прибыль - Из оборота вычитываем все вышеописанные расходы
\end{itemize}

Помимо сводки имеются ссылки на дебиторскую и кредиторскую задолженности, на отчеты и на статьи расходов. Под ссылками на дебиторку и кредиторку расположены небольшие виджеты, которые иллюстрируют информацию из данных блоков. Показываем: общую задолженность, задолженность месяца и график сумм задолженностей по месяцам (полностью аналогично прототипу).


\subsubsection{Дебиторка/Кредиторка}

В табличной форме показываем пользователю задолженность фирмы перед поставщиками и задолженность третьих лиц перед магазином.

\paragraph{Дебиторка(должны нам)}
Рассчитывается следующим образом - товар ушел, денег нет (например продажа по счету). В данном табе каждая строка представляет собой счет, по которому ещё не прошла оплата. При нажатии на строку открывается страница просмотра счета.
\\[0.5cm]
Параметры обзора:

\begin{itemize}
	\item Номер счета/чека/название фирмы клиента
	\item Дата поступления накладной
	\item Просрочка - рассчитывается автоматически на основе параметров "срок отрочки платежа" у клиента и даты оформления чека
	\item Сумма задолженности
\end{itemize}

\paragraph{Кредиторка(должны мы)}
Рассчитываем так - товар оприходован, но не вся сумма оплачена (разница между полями "закупочная стоимость" и "фактическая оплата"). Т.е. в данном табе нужно выводить все те накладные, по которым не было 100\% оплаты. При нажатии на строку с накладной открывается страница самой накладной.
\\[0.5cm]
Можно переключать между видом "накладные" и "поставщики". В первом задолженности группируются по накладным, во втором по поставщикам. 
\\[0.5cm]
Параметры обзора:

\begin{itemize}
	\item Номер накладной/название фирмы поставщика
	\item Дата поступления накладной
	\item Просрочка - рассчитывается автоматически на основе параметров "срок отрочки платежа" у поставщика и даты прихода накладной
	\item Сумма задолженности
\end{itemize}

\subparagraph{Погашение задолженности}
Также имеется кнопка, позволяющая погасить текущую задолженность (полностью или частично). При этом должны обновляться денежные параметры, т.е. сумма погашения должна попадать в расходы магазина.


\subsubsection{Отчеты}
Для всех отчетов общими являются элементы, позволяющие управлять отрезком времени, за которые генерируется отчет. По умолчанию график показывает текущий месяц, где каждое деление графика - отдельный день. Для управления используются стандартные элементы - выпадающий список ("за текущий квартал", "за текущее полугодие", "за текущий год") и два datepicker'а для выбора кастомного периода времени.

\paragraph{Продажи отделов}
Показываем все кассовые операции с возможностью выбрать отдел и соответствующей корректировкой графика (показывать только данные по отделу). Про отделы см. пункт \ref{subsec:newgoods}.

\paragraph{Товарооборот}
По умолчанию видно все продажи и все закупки по всем товарам. Дополнительные поля фильтрации позволяют показать только товары определенной группы или только определенный товар.

\paragraph{Доходы vs Расходы}
В статью доходы попадают все кассовые операции, которые являются фактическими оплатами. В расходы попадает сумма всех расходов компании (зарплата, закупки, прочие расходы)

\subsubsection{Статьи расхода}
Статьи расходов представляют собой настройки по расходам, не касающихся зарплаты и закупок. В настройках по расходам можно заносить фиксированные платежи (аренда, коммунальные услуги...), вариабельные платежи (например электричество). Если пользователь установил, что платеж вариабельный, то он должен установить дату напоминания. Начиная с этой даты в данном табе, как и на главной будет срабатывать подсказка - "вы не внесли необходимую статью расходов"

\subsubsection{Скидки}

\paragraph{Релиз Гамма}
На данной странице отображены все типы скидок (здесь можно создать новый тип + привязать его к клиенту и просмотреть уже существующие)


%Новая глава

\subsection{Сотрудники}

\subsubsection{Страница обзора}

На данной странице имеются фильтры и информация о сотрудниках в табличной форме. Фильтры следующие:

\begin{itemize}
	\item Подразделение (в котором работает сотрудник - мясной цех, бухгалтерия...)
	\item Тип сотрудника (его должность - рядовой сотрудник, управленец...)
	\item Дата выхода на работу
\end{itemize}

Общая информация состоит из следующих значений:

\begin{itemize}
	\item ФИО сотрудника
	\item Подразделение, в котором он работает 
	\item Должность сотрудника
	\item Процент продаж сотрудника
	\item Дата выхода сотрудника на работу
	\item Заработная плата сотрудника
\end{itemize}

В конце строки с данными сотрудника есть возможность совершить определенный набор действий с данным сотрудником. Набор состоит из следующих действий:
\begin{itemize}
	\item Выдать з/п - возможность занести факт выдачи заработной платы. Можно выбрать - полную или ввести собственное значение.
	\item Оштрафовать - выводим форму штрафа, состоящую из суммы штрафа и причины штрафа (сотруднику приходит оповещение на почту)
	\item Премировать сотрудника - аналогично штрафу
	\item Просмотреть карточку сотрудника - попадаем на карточку сотрудника
\end{itemize}

\subsubsection{Карточка сотрудника}

На данной странице доступна для просмотра и редактирования (соответствующими ролями - кадровик и администратор магазина) следующая информация:

\begin{itemize}
	\item ФИО сотрудника
	\item Подразделение, в котором он работает 
	\item Должность сотрудника
	\item Процент продаж сотрудника
	\item Дата выхода сотрудника на работу
	\item Период выплат (начиная с даты выхода на работу, в соответствии с указанным здесь значением бухгалтер получает оповещения об оплате)
	\item Заработная плата сотрудника
	\item Эл. почта сотрудника
	\item Адрес сотрудника
	\item Заметки о сотруднике
\end{itemize}

Для бонусов имеем следующие настройки:

\begin{itemize}
	\item Со всех продаж
	\item С группы товаров (например только с продаж мяса - можно отметить несколько отделов)
\end{itemize}

\subsection{Клиенты}

\subsubsection{Страница обзора}

На данной странице расположены фильтры и обзор данных клиентов в табличной форме. Фильтры включают в себя:

\begin{itemize}
	\item Тип клиента (физ.лицо/юрлицо)
	\item Постоянный или нет клиент
	\item Дата последней покупки
	\item Наша скидка для клиента
	\item Задолженность клиента перед нами
\end{itemize}

Также по ссылке "детально" можно перейти на страницу с карточкой клиента.

\subsubsection{Карточка клиента}

По аналогии с сотрудниками - yа данной странице доступна для просмотра и редактирования (соответствующими ролями - менеджер отдела продаж и администратор магазина) следующая информация (для физлиц): 

\begin{itemize}
	\item Контактное лицо
	\item Контактный телефон
	\item Контактная электронная почта
	\item Постоянный клиент - да/нет
	\item Номер \& Дата договора (опционально)
	\item Сколько дней отсрочки даем
	\item Адрес 
	\item Заметки
	\item Индивидуальный прайс-лист (о нем подробнее ниже)
\end{itemize}

Если клиент - юр.лицо, то поля следующие:

\begin{itemize}
	\item Название фирмы клиента
	\item ИНН
	\item БИК
	\item ОГРН
	\item Контактное лицо (идентично с предыдущем пунктом для физика)
	\item Контактный телефон
	\item Контактная электронная почта
	\item Постоянный клиент - да/нет
	\item Номер \& Дата договора (опционально)
	\item Сколько дней отсрочки даем
	\item Адрес 
	\item Заметки
	\item Индивидуальный прайс-лист (о нем подробнее ниже)
\end{itemize}


Также в отдельном визуальном блоке доступна для просмотра история по всем покупкам и оплатам клиента и общая задолженность клиента перед нами. Строка покупки ведет на просмотр чека покупки. \todo[inline]{Как и в поставщиках тут тоже нужно решить куда ведет ссылка на оплату клиента} 

Помимо этого имеем привязку прайслиста к конкретному клиенту.

\paragraph{Индивидуальный прайслист}

При редактировании карточки клиента можно выбрать либо уже существующий индивидуальный прайс-лист (например "Мясо - 5\%"), либо создать новый.
\\[0.5cm]
При создании прайслиста можно указать две опции - фикс. цены на каждый продукт или скидки в процентах. При этом для скидок работает ступенчатая модель - приоритетнее всего скидка на индивидуальный товар, за ней скидка на группу товаров, далее скидка на все товары.
\\[0.5cm]
Пример - на весь товар клиенту ООО Ромашка даются 5\% скидка. На алкогольную продукцию даются 10\%. Соответственно 10 процентов на алкоголь превалиируют над 5 процентами на все товары в разделе "алкоголь". Скидки не суммируются.

\subsection{Задачи}

На данный момент просто устанавливаем битрикс-чат.

\subsubsection{Релиз Бета-Гамма}
На момент подготовки релиза необходимо будет сформулировать требования к модулю "задачи".

\subsection{Производство}

Данный модуль подразумевает собственный интерфейс (аналогично играм вроде "Собери Пиццу"), который позволяет пользователю изъять из оборота (зарезервировать для себя) определенные товары и на выходе получить новый товар. Создаются так называемые правила, которые позволяют по кассе оформить тот или иной товар.
\\[0.5cm]
Пример - изымаем 200 кг мяса, 100 кг муки и 100 шт. яиц. На выходе получаем 1000 бургеров, которые можно пробить по кассе. Параметры погрешности в свою очередь важны для проверки - если погрешность указана и в мясе она например 10 \%, то на выходе уже нельзя пробить 1000 бургеров.
\\[0.5cm]
Исходные компоненты могут также состоять в свою очередь из собственных ингредиентов. Так например нельзя сделать из булки и котлеты бургер, если булки нет в товарах. Соответственно саму булку нужно сначала сделать из муки, яиц и воды. Таких вложенностей может быть большое количество.
\\[0.5cm]
По общему процессу производства существуют три схемы:

\begin{itemize}
	\item Из А получается Б и В (была туша говядины, получилась корейка и грудинка)
	\item Из Б и В получается А, при чём товары могут совпадать с вариантом номер 1 (были мука и дрожи, получилась булка)
	\item Из А и Б получается В, Г и Д (были молоко и дрожи, получились сыр, творог и сгущенка, при чём в процессе одного производства, вариант очень редкий)
\end{itemize}

Если система не находит исходного ингредиента в товарах, то она должна перед сохранением потребовать у пользователя создать необходимый ингредиент. При этом каждый раз система спрашивает, будет ли привязка 1 к нескольким или несколько к одному.

\begin{framed}
	Например - конечный результат: бургер. Пользователь вводит ингредиенты - котлета и булка. Ни первого ни второго в товарах нет, соответственно система просит уточнить, из чего состоят котлета и булка, для чего показывает дополнительные две формы. Для котлеты пользователь выбирает привязку несколько к одному и вносит - фарш, лук, жир. Лук есть в системе, это значение принимается, но фарша и жира нет. Соответственно система спрашивает дальше у пользователя, из чего состоят эти ингредиенты. Пользователь выбирает привязку один к нескольким, вносит "лопатка бычка" - этот товар есть на складе и в качестве товара на выходе получает фарш и жир.
\end{framed}

Для каждого процесса создания продукта из ингредиентов доступны следующие параметры:

\begin{itemize}
	\item Погрешность - указываем, сколько может на выходе оставаться товара. Например при разделке туши у товаров "корейка" и "лопатка" соотношения примерно 60 к 40, но могут быть в районе погрешностей (например 50-70 и 30-50)
	\item Процент отходов - не всегда товар трансформируется 1 к 1 из исходного в финальный продукт. Так, из 50 кг лопатки может получиться только 40 кг фарша.
	\item Процент потерь при приготовлении - при приготовлении (жарка, варка, сушение...) теряется вес товара. Так, из 100 кг фарша не будет 100 кг котлет для бургеров. Соответственно этот процент потерь необходимо указывать. 
\end{itemize}

Если при приготовлении получается отходный продукт (например второсортное, жилистое мясо), то это также заносим как продукт (=новый товар).

\subsubsection{Обзор}
Страница обзора представляет собой все товары, которые создал пользователь (сотрудник производственного цеха). Для просмотра доступны следующие параметры:

\begin{itemize}
	\item Название нового товара
	\item Вес нового товара (одной порции) - при наведении на символ (?) должны появляться в тултипе ингредиенты товара списком
	\item Кол-во штук на складе 
	\item Опция "создать партию" - данная опция позволяет на основе тех же ингредиентов создать новую партию уже существующего товара. Пользователь просто указывает, сколько штук/кг/литров он хочет создать.
\end{itemize}

Рядовой пользователь-производственник видит только созданные им товары (и соответственной не видит колонку "кем создан" (или видит со значением "мной"), производственник-администратор видит в данном окне все товары.

\subsubsection{Создание нового товара}

На странице создания нового товара есть три основных блока:
\begin{itemize}
	\item Выбор ингредиентов
	\item Дополнительные параметры готового продукта
	\item Блок для создания ингредиентов, если таковых нет на складе
\end{itemize}

Для первого блока имеем поля:

\begin{itemize}
	\item Строка с поиском по складу магазина (имеем саджест), если ингредиента нет в товарах, то показываем третий блок
	\item Кол-во ингредиента
	\item Единица измерения ингредиента (подгружаем автоматически на основе значений в товарах)
	\item Соотношение товара (если выбрана опция "один к нескольким" - имеем слайдер, который подсказывает, что соотношение должно в сумме быть равно 100\%)
	\item Процент потери при приготовлении (идет в рассчитывание себестоимости и может различаться для разных ингредиентов (например мясо ужаривается сильно, а булка - нет))
\end{itemize}

В параметрах готового продукта имеем:

\begin{itemize}
	\item Занесение названия нового товара (например "Бургер "Супервкусный"")
	\item Погрешность (если выбран тип создания товара "из 1 несколько", то важно, иначе не показываем)
	\item Процент отходов (идет в рассчитывание себестоимости)
	\item Сколько весит одна порция (сколько штук/литров в одной порции)
	\item Единицу измерения применять для нового товара
	\item Указать, сколько порций нового товара пользователь хочет сделать в первой партии (в последующих он это указывает через "создать партию"). При этом обозначение единицы измерения должно динамично меняться в зависимости от того, что пользователь ввел пунктом выше.
\end{itemize}

Если при выборе ингредиентов один из них не был найден на складе, то программа подгружает форму, аналогичную как для создания самого товара для создания и ингредиента (при этом таких вложенностей может быть несколько, т.е. если ингредиент состоит в свою очередь из ингредиентов, которых нет на складе, то ниже подгружаем ещё одну такую же форму).
\\[0.5cm]
Параметры идентичны созданию готового продукта, за исключением названия (так как его берем из поля "ингредиенты" готового продукта.

\paragraph{Расчет себестоимости}
Для расчета себестоимости необходимо учитывать все параметры для всех "итераций" вложенности создания продукта, которые указывает пользователь при создании. 
\begin{framed}
	Пример - создание бургера, во время которого необходимо также создать фарш из лопатки бычка. При создании фарша из лопатки также имеется продукт - жир. Соотношение 60 к 40. Теперь имеем значения - погрешность, отходы и потери при приготовлении.
	\\[0.5cm]
	Погрешность ($errorRate$) относится напрямую к соотношению. Погрешность 5 \% означает, что допустимы нормы от 55 к 45 до 65 к 35.
	\\[0.5cm]
	Процент отходов ($garbagePercentage$) означает потери в производстве, 10\% означают, что из 100кг лопатки получится только 90кг общего продукта на выходе.
	\\[0.5cm]
	Процент потерь ($lossPercentage$) при производстве также накладывается на конечный результат (при жарке выходит влага и т.д.)
	\end{framed}
	Таким образом очередность операций такая (псевдокод):
		\begin{lstlisting}[
		    label=listing:RubyTest,
		    float=h,
    		caption=sebestoimost.rb,
		    firstnumber=1,
		    language=Ruby,
		    basicstyle=\ttfamily,
		    keywordstyle=\color{red},
		    stringstyle=\color{blue},
		    frame=single
			]
inputProduct *= garbagePercentage

outputComponents.each do |component|
	component.upperAmount = (inputProduct * (ratio+errorRate)) 
		* lossPercentage
	component.lowerAmount = (inputProduct * ratio-errorRate)
		* lossPercentage
	
	component.upperPrice = (inputProduct.price * (ratio+errorRate))
		/component.upperAmount
	component.lowerPrice = (inputProduct.price * (ratio-errorRate))
		/component.lowerAmount

end
		\end{lstlisting}

\begin{framed}
	Пример (продолжение) - например себестоимость лопатки - 300 рублей/кг. 100 кг стоят 30000 рублей. Конечное кол-во продукта на выходе - 90кг (из-за 10 процентов отходов). Соотношение 60 к 40. Соответсвенно 90*0.6 = 54 кг стоят (30000*0.6 =) 18000 рублей. Фарш ужаривается, получается, что на выходе 43.2 кг котлет с потерей 20\%. Но 43.2 кг имеют ту же себестоимость (18000 рублей). Таким образом себестоимость кг котлет возрасла с 300 рублей/кг до 416 рублей/кг (18000/43.2).
	\\[0.5cm]
	Соответственно если в бургере используется 200грамм котлет, то "котлетная себестоимость" его равна 416*0.2 = 83 рубля.
\end{framed}

\subsubsection{Согласование производства}

\paragraph{Релиз Бета-Гамма}
Для начала пользователь-производственник подает заявку товароведу на изъятие определенного перечня товара (в заявке указываем причину, конечный продукт, список ингридиентов). Эта заявка находится переходит несколько статусов - "ожидание", "принято", "отклонено (с причиной)", "необходимо уточнение".

\subsubsection{Инвентаризация}
Необходимо дублировать некоторые части инвентаризации модуля "товары", так как только при взаимосвязи с инвентаризацией производство становится по настоящему полезным. Суть инвентаризации простая - в две колонки выводятся значения склада (фактические) и допустимые нормы. 
\paragraph{Допустимые нормы}
Допустимые нормы рассчитываются на основе продаж по кассе и значений, указанных в "рецептах" производства. Так, если для производства 1000 бургеров нужно от 40 до 60 кг туши, а на складе только 20кг (и нет других производных товаров, как например фарша), то система должна показывать несходимость. 
\\[0.5cm]
Опции фильтровки для таблицы - категории товара и опция "показать только критичные товары".

\subsection{Поставщики}

Самый обычный список фирм-поставщиков со стандартными CRUD-действиями.

\subsubsection{Страница обзора}

Обзор поставщиков состоит из табличного вида информации по поставщикам и фильтрам для настроек обзора под нужды пользователя. Фильтры следующие:

\begin{itemize}
	\item По типу поставляемой продукции
	\item По типу самого поставщика 
	\item По региону
\end{itemize}

В обзоре доступны для просмотра следующие поля:

\begin{itemize}
	\item Название компании поставщика
	\item Тип поставляемой продукции
	\item Наша скидка у поставщика
	\item Дата последней закупки
	\item Наша задолженность перед поставщиком
\end{itemize}

Также у каждого поставщика доступна ссылка на детальный просмотр профиля поставщика.

\subsubsection{Профиль поставщика}

На данной странице доступна следующая информация для просмотра, если поставщик физлицо:

\begin{itemize}
	\item Контактное лицо
	\item НДС поставщика
	\item Телефон контактного лица
	\item Эл. почта контактного лица
	\item Является ли поставщик постоянным
	\item Номер договора и дата его заключения
	\item Срок отсрочки в днях
	\item Адрес поставщика
	\item Заметки
\end{itemize}

Если юр.лицо, то поля следующие:

\begin{itemize}
	\item Название компании
	\item Контактное лицо
	\item ИНН
	\item БИК
	\item ОГРН
	\item НДС поставщика
	\item Телефон контактного лица
	\item Эл. почта контактного лица
	\item Является ли поставщик постоянным
	\item Номер договора и дата его заключения
	\item Срок отсрочки в днях
	\item Юридический адрес
	\item Фактический адрес
	\item Заметки
\end{itemize}

Также в отдельном визуальном блоке доступна для просмотра история по всем заказам и оплатам поставщика с показом общего долга этому поставщику. Строка заказа ведет на соответствующую накладную, по которой был оформлен товар поставщика, по строке оплаты.


\subsection{Группы пользователей}

Для разделения прав важно - каждый модуль обладает своими правами доступа. Например работнику производства нельзя давать доступ к модулю "касса", а продавцу к модулю "производство".
Для товаров важно - разделение прав по категориям товаров. Например сотрудникам мясного цеха нельзя давать доступ к молочной продукции.
Иерархия делиться следующим образом:

\begin{itemize}
	\item Директор магазина
	\item Управляющий персонал
	\item Бухгалтерия
	\item Рядовой сотрудник
\end{itemize}

А также на различные разделы, например:

\begin{itemize}
	\item Мясной цех
	\item Финансовый отдел
	\item Касса
	\item Зал
\end{itemize}

Любой пользователь может состоять в нескольких группах одновременно.

\subsection{Настройки}

Данная страница позволяет настраивать базовые параметры магазина.

\subsubsection{Релиз Альфа}

В первом релизе доступен самый базовый функционал, который делится на четыре блока:

\begin{itemize}
	\item Главные настройки
	\item Часы работы
	\item Реквизиты
	\item СМС (пока в блоке - релиз Бета)
\end{itemize}

\paragraph{Главные настройки}
Здесь имеется возможность просмотреть и редактировать такие поля, как: название магазина, контактный телефон магазина, электронная почта магазина.

\paragraph{Часы работы}
Здесь указываются часы работы магазина.

\paragraph{Реквизиты}
Здесь указываются оф. бухгалтерские данные:

\begin{itemize}
	\item ИНН
	\item КПП
	\item ОГРН
	\item Расчетный счет
	\item Корреспонденский счет
	\item БИК
	\item Наименование банка
	\item Юридический адрес
	\item Фактический адрес
\end{itemize}

\subsubsection{Релиз Бета}

В данном релизе добавляются - смс-оповещения и тарифный план\todo[inline]{Необходимо сформулировать требования для смс - в каком случае кому отправлять, а также что ещё необходимо магазину помимо СМС и тарифов} 

\end{document}