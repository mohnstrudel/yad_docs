\documentclass[DIV=calc, paper=a4, fontsize=11pt]{scrartcl} % Документ принадлежит классу article, а также будет печататься в 12 пунктов.
\usepackage{ucs}
\usepackage[T1,T2A]{fontenc}
\usepackage[utf8x]{inputenc} % Включаем поддержку UTF8
\usepackage[russian]{babel} % Пакет поддержки русского языка
\usepackage{titling} % Allows custom title configuration

%for frames
\usepackage{framed}

%For image using
\usepackage{graphicx}

%Numbering subsubsubsections etc
\setcounter{secnumdepth}{5}

%For code highlightning
\usepackage{listings}

%Further enumeration
\usepackage{enumitem}
\setenumerate[1]{label=\theparagraph.\arabic*.}
\setenumerate[2]{label*=\arabic*.}
\setenumerate[3]{label*=\arabic*.}


%For referencing within enumeration lists
\usepackage{enumitem}

%Packages for word-like comment style
\usepackage{todonotes}

%Package for images
\usepackage{float}
\floatstyle{boxed}
\restylefloat{figure}

%For a nicer reference
\usepackage{fancyref}

\usepackage{titlesec}


\titleformat*{\section}{\LARGE\bfseries}
\titleformat*{\subsection}{\Large\bfseries}
\titleformat*{\subsubsection}{\large\bfseries}
\titleformat*{\paragraph}{\large\bfseries}
\titleformat*{\subparagraph}{\large\bfseries}

%For some math formulas if needed
\usepackage{mathtools}

% Some nice visualization
%\usepackage[svgnames]{xcolor} % Enabling colors by their 'svgnames'
\usepackage{fullpage}
%\renewcommand{\headrulewidth}{0.0pt} % No header rule
%\renewcommand{\footrulewidth}{0.4pt} % Thin footer rule
% End visualization

%smart enumeration
\renewcommand{\labelenumi}{\arabic{enumi}.}
\renewcommand{\labelenumii}{\arabic{enumi}.\arabic{enumii}}


% General document settings
% % %
% Accepted file upload formats
\newcommand{\AcceptedFormats}{.txt, .csv, .xls, .xlsx}

\title{Система лояльности клиентов}
\date{12/10/2015}

\begin{document}

\maketitle

\section{Описание проекта}

Данный проект (рабочее название mondi) представляет собой программное обеспечение - SaaS (Software as a Service), размещенное на сервере/серверах заказчика и позволяющее клиентам заказчика регистрироваться в сервисе. \\[0.5cm]
Суть системы состоит в возможности для клиентов заказчика, загружать данные о покупке продукции, получения баллов за эти покупки и приобретения на баллы различных продуктов.

\subsection{Описание для сотрудника Монди}
Пользователь (сотрудник) компании Монди, который является заказчиком, может просматривать всех пользователей (клиентов), просматривать профиль каждого пользователя, просматривать загруженные документы и сколько баллов за них было начислено. Также он может просматривать все покупки любого пользователя (клиента).

\subsection{Описание для рядового пользователя}
Рядовой пользователь может регистрироваться в системе, загружать документы покупки (полный перечень в разделе ... ), просматривать каталог товаров и услуг, которые можно получить за баллы и подавать запрос на получение услуги за определенное кол-во баллов.

\subsection{Прототип}
Прототип проекта доступен по ссылке http://yadadya.com/showcase/mondi/prototype/

\section{Технические уточнения}

\subsection{Терминология}
        
        \begin{itemize}
        \item Блок - некий визуальный элемент, выделяющийся либо графически (в виде рамок, очертаний), либо по смыслу (совокупность похожих элементов)
        \item Компонент - часть содержания, имеющего закрытое визуально представление. Одна страница сайта может состоять из нескольких компонентов.
        \item Фронтэнд - для пользователя видимая оболочка веб-страницы
        \item Бэкэнд - невидимые для пользователя математические алгоритмы
        \item CMS - все работы происходят на основе системы управления содержанием - CMS 1C Bitrix (1С Битрикс)
        \item Модуль - является описанием общего функционала, который не может быть классифицирован как привязанный к определенной странице. Он может встречаться на любой странице в любом месте. Модуль может состоять из нескольких компонентов. Также модуль может содержать в себе части логики фронтэнда и бэкэнда.
        \item Хэдер - верхняя часть сайта, обладающая определенной структурой, которая видна сквозняком на всех или почти всех страницах сайта. Также используется обозначение "шапка".
        \item Футер - нижняя часть сайта. Функционал аналогичен хэдеру. Также используется обозначение "подвал".
    \end{itemize}
    
\subsection{Технические требования}
Для работоспобности системы mondi её необходимо установить на unix-машине с настроенным apache2 или nginx сервером.
Для удобства администрирования также рекомендуется использовать phpMyAdmin.
\\[0.5cm]
Минимальная конфигурация сервера:
\begin{itemize}
	\item 512 мегабайт оперативной памяти
	\item 1-ядерный процессор
	\item 1500 мегагерц тактовой частоты
	\item 15000 килобайт процессорного кеша
\end{itemize}


\section{Модули}

\subsection{Перечень}

Как таковых модулей в программе нет.

\section{Страницы}

\subsection{Главная}

Главная страница представляет собой посадочную страницу, которая объясняет суть программы (текстом и изображениями), имеет также несколько областей захвата, которые ведут пользователя на страницу регистрации.

\subsubsection{Регистрация/Логин}
На данной подстранице пользователь может зарегистрироваться в системе или зайти в систему по уже имеющейся связке логин/пароль. Сюда также относится подстраница "забыли пароль?", на которой пользователь может восстановить свой пароль.

\subsection{Личный кабинет}

Личный кабинет делится на два основных блока - ЛК для рядового пользователя и администраторский ЛК (для сотрудника Монди - в дальнейшем "администратор"). Возможности данной системы описываются отдельно для каждой категории пользователей.

\subsubsection{Администратор}

\paragraph{Дашборд}

Дашборд является отправной точкой личного кабинета администратора. Тут он видит:

\begin{itemize}
	\item Сколько пользователей зарегистрировалось в какой промежуток времени
	\item Сколько баллов было получено пользователями в какой промежуток времени
	\item Сколько баллов было потрачено пользователями
	\item Самые частопокупаемые услуги (топ-5 и сколько покупок было сделано)
\end{itemize}

Промежуток времени всегда одинаковый - по умолчанию это текущая неделя с возможность выбрать месяц и квартал.
\\[0.5cm]
Далее администратор может перейти по следующим ссылкам:

\begin{itemize}
	\item Пользователи
	\item Баллы
	\item Каталог (товары/услуги)
	\item Документы
\end{itemize}

\paragraph{Пользователи}
На данной странице администратор видит всех пользователей системы в табличном виде. Показаны следующие столбцы:

\begin{itemize}
	\item Почта пользователя
	\item ФИО
	\item Компания
	\item Кол-во загруженных документов
	\item Кол-во полученных баллов
	\item Кол-во потраченных баллов
\end{itemize}

При нажатии на строку с пользователем администратор попадает на страницу пользователя.

\subparagraph Профиль пользователя.
На данной странице администратор видит профиль пользователя, который разделен вкладками. Первая вкладка - данные пользователя:

\begin{itemize}
	\item Логин
	\item Почта
	\item ФИО
	\item Компания
	\item Телефон
\end{itemize}

Вторая вкладка - документы. На данной вкладке в табличном формате видно все документы, которые загружал пользователь и кол-во баллов, которые он за документы получил. Поля следующие:

\begin{itemize}
	\item Дата загрузки документа
	\item Название документа (с возможностью нажать на него и просмотреть)
	\item Тип документа
	\item Кол-во полученных баллов
\end{itemize}

Третья вкладка - баллы. На данной вкладке в табличном формате видны все покупки клиента, а также общее кол-во баллов, которые он заработал. Поля следующие:

\begin{itemize}
	\item Наименования товара/услуги (по нажатию можно пройти на страницу услуги)
	\item Кол-во списанных баллов
	\item Дата приобритения
\end{itemize}

\paragraph{Баллы}

На данной странице администратор видит все покупки пользователей. В табличной форме он видит следующие поля:

\begin{itemize}
	\item Наименование товара/услуги (по нажатию администратор переходит на страницу услуги)
	\item Пользователь, купивший услугу/товар
	\item Кол-во баллов, которое было потрачено
	\item Дата покупки
\end{itemize}

\paragraph{Каталог}

На данной странице администратор видит все позиции каталога в табличной форме. Также ему доступна кнопка "создать новую позицию". Поля следующие:

\begin{itemize}
	\item Наименование позиции каталога
	\item Стоимость позиции в баллах
	\item Дата создания
	\item Дата последней покупки данной позиции
\end{itemize}

При нажатии на наименование (или на "создать новую позицию") администратор попадает на страницу редактирования (создания новой) позиции. Здесь он может задать следующие параметры:

\begin{itemize}
	\item Название
	\item Стоимость в баллах
	\item Описание
	\item Фотографии
\end{itemize}

\paragraph{Документы}

На данной странице администратор видит перечень всех документов в табличной форме. Поля следующие:

\begin{itemize}
	\item Наименование документа (кликабельно, при нажатии можно скачать документ)
	\item ФИО и логин загрузившего пользователя
	\item Сколько баллов было начислено за документ
\end{itemize}










\end{document}