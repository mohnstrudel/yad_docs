\documentclass[DIV=calc, paper=a4, fontsize=11pt]{scrartcl} % Документ принадлежит классу article, а также будет печататься в 12 пунктов.
\usepackage{ucs}
\usepackage[T1,T2A]{fontenc}
\usepackage[utf8x]{inputenc} % Включаем поддержку UTF8
\usepackage[russian]{babel} % Пакет поддержки русского языка
\usepackage{titling} % Allows custom title configuration

%for frames
\usepackage{framed}

%For image using
\usepackage{graphicx}

%Numbering subsubsubsections etc
\setcounter{secnumdepth}{5}

%For code highlightning
\usepackage{listings}

%Further enumeration
\usepackage{enumitem}
\setenumerate[1]{label=\theparagraph.\arabic*.}
\setenumerate[2]{label*=\arabic*.}
\setenumerate[3]{label*=\arabic*.}


%For referencing within enumeration lists
\usepackage{enumitem}

%Packages for word-like comment style
\usepackage{todonotes}

%Package for images
\usepackage{float}
\floatstyle{boxed}
\restylefloat{figure}

%For a nicer reference
\usepackage{fancyref}

\usepackage{titlesec}


\titleformat*{\section}{\LARGE\bfseries}
\titleformat*{\subsection}{\Large\bfseries}
\titleformat*{\subsubsection}{\large\bfseries}
\titleformat*{\paragraph}{\large\bfseries}
\titleformat*{\subparagraph}{\large\bfseries}

%For some math formulas if needed
\usepackage{mathtools}

% Some nice visualization
%\usepackage[svgnames]{xcolor} % Enabling colors by their 'svgnames'
\usepackage{fullpage}
%\renewcommand{\headrulewidth}{0.0pt} % No header rule
%\renewcommand{\footrulewidth}{0.4pt} % Thin footer rule
% End visualization

%smart enumeration
\renewcommand{\labelenumi}{\arabic{enumi}.}
\renewcommand{\labelenumii}{\arabic{enumi}.\arabic{enumii}}


% General document settings
% % %
% Accepted file upload formats
\newcommand{\AcceptedFormats}{.txt, .csv, .xls, .xlsx}

\title{Система лояльности клиентов}
\date{12/10/2015}

\begin{document}

\maketitle

\section{Описание проекта}

Данный проект (рабочее название mondi) представляет собой программное обеспечение - SaaS (Software as a Service), размещенное на сервере/серверах клиента и позволяющее собирать данные об оборудовании, которым обеспечивает заказчик своих клиентов. Данные оборудования хранятся в двух различных базах данных (Oracle, MSSql) и представляют собой складские данные и данные сервисного центра.
\\[0.5cm]
Суть системы состоит в обработке этих данных и предоставления сотрудникам заказчика, а также сотрудникам клиентов заказчика отчетов на основе собранных данных. Отчеты должны генерироваться как в онлайн (в окне браузера), так и быть доступными для скачивания.

\subsection{Описание для клиента-банка}
Пользователь (сотрудник) банка, который является клиентом заказчика, может просматривать все данные системы, но ограничен диапазоном только своего банка и правами доступа внутри своего банка. Так, региональный представитель видит все филиалы региона (но не видит другие регионы), а главный директор видит все регионы.

\subsection{Описание для заказчика}
Пользователь (сотрудник) фирмы заказчика видит все товары и все банки и может строить отчеты по всем данным.

\subsection{Прототип}
Прототип проекта доступен по ссылке http://yadadya.com/showcase/trackmate/prototype/

\section{Технические уточнения}

\subsection{Терминология}
        
        \begin{itemize}
        \item Блок - некий визуальный элемент, выделяющийся либо графически (в виде рамок, очертаний), либо по смыслу (совокупность похожих элементов)
        \item Компонент - часть содержания, имеющего закрытое визуально представление. Одна страница сайта может состоять из нескольких компонентов.
        \item Фронтэнд - для пользователя видимая оболочка веб-страницы
        \item Бэкэнд - невидимые для пользователя математические алгоритмы
        \item CMS - все работы происходят на основе системы управления содержанием - CMS 1C Bitrix (1С Битрикс)
        \item Модуль - является описанием общего функционала, который не может быть классифицирован как привязанный к определенной странице. Он может встречаться на любой странице в любом месте. Модуль может состоять из нескольких компонентов. Также модуль может содержать в себе части логики фронтэнда и бэкэнда.
        \item Хэдер - верхняя часть сайта, обладающая определенной структурой, которая видна сквозняком на всех или почти всех страницах сайта. Также используется обозначение "шапка".
        \item Футер - нижняя часть сайта. Функционал аналогичен хэдеру. Также используется обозначение "подвал".
    \end{itemize}
    
\subsection{Технические требования}
Для работоспобности системы Trackmate её необходимо установить на unix-машине с настроенным apache2 или nginx сервером.
Для удобства администрирования также рекомендуется использовать phpMyAdmin.
\\[0.5cm]
Минимальная конфигурация сервера:
\begin{itemize}
	\item 512 мегабайт оперативной памяти
	\item 1-ядерный процессор
	\item 1500 мегагерц тактовой частоты
	\item 15000 килобайт процессорного кеша
\end{itemize}


\section{Модули}

\subsection{Товары}



\section{Страницы}

\subsection{Главная}











\end{document}