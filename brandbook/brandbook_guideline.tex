\documentclass[DIV=calc, paper=a4, fontsize=11pt]{scrartcl} % Документ принадлежит классу article, а также будет печататься в 12 пунктов.
\usepackage{ucs}
\usepackage[T1,T2A]{fontenc}
\usepackage[utf8x]{inputenc} % Включаем поддержку UTF8
\usepackage[russian]{babel} % Пакет поддержки русского языка
\usepackage{titling} % Allows custom title configuration

%For image using
\usepackage{graphicx}

%For referencing within enumeration lists
\usepackage{enumitem}

%Packages for word-like comment style
\usepackage{todonotes}

%Package for images
\usepackage{float}
\floatstyle{boxed}
\restylefloat{figure}

%For a nicer reference
\usepackage{fancyref}

\usepackage{titlesec}


\titleformat*{\section}{\LARGE\bfseries}
\titleformat*{\subsection}{\Large\bfseries}
\titleformat*{\subsubsection}{\large\bfseries}
\titleformat*{\paragraph}{\large\bfseries}
\titleformat*{\subparagraph}{\large\bfseries}

%For some math formulas if needed
\usepackage{mathtools}

% Some nice visualization
%\usepackage[svgnames]{xcolor} % Enabling colors by their 'svgnames'
\usepackage{fullpage}
%\renewcommand{\headrulewidth}{0.0pt} % No header rule
%\renewcommand{\footrulewidth}{0.4pt} % Thin footer rule
% End visualization

%smart enumeration
\renewcommand{\labelenumi}{\arabic{enumi}.}
\renewcommand{\labelenumii}{\arabic{enumi}.\arabic{enumii}}


\title{Руководство по оформлению брендбука}
\date{12/09/2014}

\begin{document}
\maketitle


\chapter{Общая структура}

\section{Перечень страниц}
На данный перечень можно ориентироваться при создании брендбука с нуля:

    \begin{enumerate}
        \item О проекте - описать сам проект, небольшую историю работы с ним (что привело к каким решениям) 
        \item Логотип - основной логотип, упрощенный, монохромный, разные варианты логотипа, разные цвета для логотипа, недопустимое использование логотипа
        \item Цвет - основные цвета, дополнительные цвета, градиент, лого на фоне
        \item Паттерны - из чего состоит, как использовать
        \item Шрифт - примеры использования
        \item Объекты - объекты как примеры использования фирменного стиля
    \end{enumerate}



\chapter{Логотип}

\section{Оформление логотипа}

\paragraph{Логотип и его история} На первой странцие показываем логотип в его обычной форме (=как есть), объясняем, какие идеи послужили при его придумывании.
\paragraph{Техническая сторона логотипа}
Далее можно оформить логотип в техническом стиле, т.е. разметкой, показывающей, какие формы используются. Пример на логотипе Якитории:
        \begin{figure}[H]
        \centering
        \includegraphics[width=100mm]{technical_logo.jpg}
        \caption{Логотип, оформленный с технической разметкой. \label{technical_logo.jpg}}
        \end{figure}
        
    или
    
    \begin{figure}[H]
        \centering
        \includegraphics[width=70mm]{technical_logo2.jpg}
        \caption{Второй вариант логотипа, оформленный с технической разметкой. \label{technical_logo2.jpg}}
        \end{figure}
\paragraph{Использование логотипа}
На данной странице (в данной части) показываем клиенту, как можно использовать логотип. Желательно показать следующее:
    \begin{enumerate}
        \item Логотип на разных фонах - как пример
            \begin{figure}[H]
            \centering
            \includegraphics[width=100mm]{logo_diff_background.jpg}
            \caption{Логотип с различным фоном \label{fig:logo_diff_background.jpg}}
            \end{figure}
        \item Размещение логотипа на различных поверхностях (вывески, здания, билборды...)
        \item Размещение логотипа на различных товарах (футболки, коробки, кепки, сувениры - ключи, матрешки, кошельки...)
        \item Лого в разных цветах (самого логотипа)
        \item Показываем как логотип нельзя использовать ни в коем случае - различные цветовые комбинации, как самого логотипа, так и фона. Текстуры, на фоне которых нельзя логотип использовать.
        \begin{figure}[H]
            \centering
            \includegraphics[width=256px]{logo_not_to.jpg}
            \caption{Как не использовать лого\label{fig:logo_not_to.jpg}}
            \end{figure}
    \end{enumerate}
Естественно всегда представлять те наборы предметов, которые актуальны для клиента, т.е. если это нефтяная компания, то скорее предметом послужит деловая папка или грузовик-цистерна, а не воздушный шар.


\section{Цвета}

\subsection{Оформление фирменных цветов}
Показываем в оригинальной композиции, какие цвета будут использоваться. Желательно для этого использовать область разворота брэндбука. Показываем для каждого цвета все его характеристики - RGB значения, НЕХ и другие разные стандарты. Оригинальность может заключаться в следующем:
    \begin{enumerate}
        \item Цвета в виде капель
        \item Цвета в виде слайдера
        \item Оформление в виде волн различных цветов
        \item Различные элементы, соответствующие стилю и направлению клиента, разные части которого покрашены в фирменные цвета. Например:
            \begin{enumerate}
                \item Если компания занимается недвижимостью, то можно представить изометрический дом, у которого каждая стена покаршена в один из фирм. цветов
                \item Если у компании логотип - символ ДНК, то струи сплетения можно покрасить различными цветами
            \end{enumerate}
    \end{enumerate}
Пример оформления:
            \begin{figure}[H]
            \centering
            \includegraphics[width=128px]{colors.jpg}
            \caption{Пример оформления страницы с цветами \label{fig:colors.jpg}}
            \end{figure}
            
или
            \begin{figure}[H]
            \centering
            \includegraphics[width=512px]{colors2.jpg}
            \caption{Дополнительный пример оформления страницы с цветами\label{fig:colors2.jpg}}
            \end{figure}
    
\subsection{Паттерны}
Если клиенту необходим паттерн, то также оформляем его отдельной страницей (можно и разворотом). Типичный пример для паттерна - уменьшенный логотип или часть логотипа. Также паттерн можно показать в его типичном использовании.
\\[0.5cm]
Например для использования паттерна можно взять объекты, для которых возможно такое пользование - галстук, бэкграунд под фирменные листы А4...
\\[0.5cm]
Пример использования:
            \begin{figure}[H]
            \centering
            \includegraphics[width=128px]{pattern.jpg}
            \caption{Пример использования паттернов на объектах \label{fig:pattern.jpg}}
            \end{figure}
или

\begin{figure}[H]
            \centering
            \includegraphics[width=256px]{pattern2.jpg}
            \caption{Второй пример использования паттернов на объектах \label{fig:pattern2.jpg}}
            \end{figure}
            
\section{Шрифты}

\subsection{Оформление фирменных шрифтов}
На одну-две страницы показываем фирменные шрифты. На первой шрифты оформлены нейтрально, т.е. можно половину страницы закрасить черным и типовое предложение написать белым, вторая половина - фон белый, шрифт черный. На второй странице шрифт можно отобразить, например, на оф. бланке, на октрытке, в окне браузера (как будто пользователь пишет электронное письмо).

            \begin{figure}[H]
            \centering
            \includegraphics[width=256px]{typefont.jpg}
            \caption{Пример оформления шрифтов \label{fig:typefont.jpg}}
            \end{figure}


\section{Объекты}

\subsection{Оформление объектов}
На разворотах или каждый на отдельной странице можно показать элементы, оформленные в фирменном стиле. Элементы должны подходить по тематике к направлению деятельности клиента. Возможные примеры таких элементов:
    \begin{enumerate}
        \item Визитки, ручки, термосы, бейджики, визитница, кружки
        \item Наружняя реклама
        \item Воздушные шары
        \item Машины (грузовики, легковые, индустриальные, тракторы...)
        \item Подарочные упаковки
        \item Печатные материалы - конверты, бланки бумаг, открытки, планшеты, папки, обложки
        \item Зонты, одежда
    \end{enumerate}
    
Примеры оформления:

            \begin{figure}[H]
            \centering
            \includegraphics[width=256px]{object1.jpg}
            \caption{Первый пример объекта в фирменном стиле \label{fig:object1.jpg}}
            \end{figure}
            
            \begin{figure}[H]
            \centering
            \includegraphics[width=384px]{object2.jpg}
            \caption{Второй пример объекта в фирменном стиле \label{fig:object2.jpg}}
            \end{figure}
            
            \begin{figure}[H]
            \centering
            \includegraphics[width=384px]{object3.jpg}
            \caption{Третий пример объекта в фирменном стиле \label{fig:object3.jpg}}
            \end{figure}
    
\subsection{Общее}
Каждый элемент брендбука снабжаем описанием, что это за элемент, как его использовать или в каких случаях использовать не надо. Под цветом всегда пишем текстом, что это за цвет. Также к каждому элементу прикрепляем картинку флешки 
            \begin{figure}[H]
            \centering
            \includegraphics[width=64px]{memory.png}
            \caption{Символ с флешкой \label{fig:memory.png}}
            \end{figure}
и пишем путь и название файла, что бы клиент смог быстро найти то, что ищет в дигитальном формате.





        




\end{document}

