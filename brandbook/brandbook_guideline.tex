\documentclass[DIV=calc, paper=a4, fontsize=11pt]{scrartcl} % Документ принадлежит классу article, а также будет печататься в 12 пунктов.
\usepackage{ucs}
\usepackage[T1,T2A]{fontenc}
\usepackage[utf8x]{inputenc} % Включаем поддержку UTF8
\usepackage[russian]{babel} % Пакет поддержки русского языка
\usepackage{titling} % Allows custom title configuration

%For image using
\usepackage{graphicx}

%For referencing within enumeration lists
\usepackage{enumitem}

%Packages for word-like comment style
\usepackage{todonotes}

%For a nicer reference
\usepackage{fancyref}

\usepackage{titlesec}

\titleformat*{\section}{\LARGE\bfseries}
\titleformat*{\subsection}{\Large\bfseries}
\titleformat*{\subsubsection}{\large\bfseries}
\titleformat*{\paragraph}{\large\bfseries}
\titleformat*{\subparagraph}{\large\bfseries}

%For some math formulas if needed
\usepackage{mathtools}

% Some nice visualization
%\usepackage[svgnames]{xcolor} % Enabling colors by their 'svgnames'
\usepackage{fullpage}
%\renewcommand{\headrulewidth}{0.0pt} % No header rule
%\renewcommand{\footrulewidth}{0.4pt} % Thin footer rule
% End visualization

%smart enumeration
\renewcommand{\labelenumi}{\arabic{enumi}.}
\renewcommand{\labelenumii}{\arabic{enumi}.\arabic{enumii}}


\title{Руководство по оформлению брендбука}
\date{12/09/2014}

\begin{document}
\maketitle


\chapter{Логотип}

\section{Оформление логотипа}

\paragraph{Логотип и его история} На первой странцие показываем логотип в его обычной форме (=как есть), объясняем, какие идеи послужили при его придумывании.
\paragraph{Техническая сторона логотипа}
Далее можно оформить логотип в техническом стиле, т.е. разметкой, показывающей, какие формы используются. Пример на логотипе Якитории:
        \begin{figure}[ht!]
        \centering
        \includegraphics[width=100mm]{technical_logo.jpg}
        \caption{Логотип, оформленный с технической разметкой. \label{technical_logo.jpg}}
        \end{figure}
        
    или
    
    \begin{figure}[ht!]
        \centering
        \includegraphics[width=100mm]{technical_logo2.jpg}
        \caption{Второй вариант логотипа, оформленный с технической разметкой. \label{technical_logo2.jpg}}
        \end{figure}
\paragraph{Использование логотипа}
На данной странице (в данной части) показываем клиенту, как можно использовать логотип. Желательно показать следующее:
    \begin{enumerate}
        \item Логотип на разных фонах - как пример
    \end{enumerate}

        




\end{document}

