\documentclass[DIV=calc, paper=a4, fontsize=11pt]{scrartcl} % Документ принадлежит классу article, а также будет печататься в 12 пунктов.
\usepackage{ucs}
\usepackage[T1,T2A]{fontenc}
\usepackage[utf8x]{inputenc} % Включаем поддержку UTF8
\usepackage[russian]{babel} % Пакет поддержки русского языка
\usepackage{titling} % Allows custom title configuration

%for frames
\usepackage{framed}

%For image using
\usepackage{graphicx}

%Numbering subsubsubsections etc
\setcounter{secnumdepth}{5}


%Further enumeration
\usepackage{enumitem}
\setenumerate[1]{label=\theparagraph.\arabic*.}
\setenumerate[2]{label*=\arabic*.}
\setenumerate[3]{label*=\arabic*.}


%For referencing within enumeration lists
\usepackage{enumitem}

%Packages for word-like comment style
\usepackage{todonotes}

%Package for images
\usepackage{float}
\floatstyle{boxed}
\restylefloat{figure}

%For a nicer reference
\usepackage{fancyref}

\usepackage{titlesec}


\titleformat*{\section}{\LARGE\bfseries}
\titleformat*{\subsection}{\Large\bfseries}
\titleformat*{\subsubsection}{\large\bfseries}
\titleformat*{\paragraph}{\large\bfseries}
\titleformat*{\subparagraph}{\large\bfseries}

%For some math formulas if needed
\usepackage{mathtools}

% Some nice visualization
%\usepackage[svgnames]{xcolor} % Enabling colors by their 'svgnames'
\usepackage{fullpage}
%\renewcommand{\headrulewidth}{0.0pt} % No header rule
%\renewcommand{\footrulewidth}{0.4pt} % Thin footer rule
% End visualization

%smart enumeration
\renewcommand{\labelenumi}{\arabic{enumi}.}
\renewcommand{\labelenumii}{\arabic{enumi}.\arabic{enumii}}


\title{Макет технического задания для салонов красоты (рабочее название - Beauty Booking или BB)}
\date{07/10/2014}

\begin{document}

\maketitle

\section{Описание проекта}
Сайт представляет собой платформу, на которой клиенты могут искать бьюти-услуги, а салоны и фрилансеры предлагать такие услуги. Сайт состоит из двух компонент - компонент для клиента представляет собой удобный поиск и бронирование похода к мастеру. Для мастера (салон или фрилансер) сайт представляет собой полноценный административный рабочий инструмент. Мастер может принимать заявки в режиме онлайн, вести учетную запись клиентов, получать заказы с сайта (с компонента для клиентов), продвигать свои услуги.
\subsection{Описание компонента для клиента}
Для клиента сервис собой представляет красивый сайт, позволяющий искать услуги по запросу клиента. Особенностью сайта является его полная адаптивность и облик мобильного приложения, который полностью копирует мобильную версию сайта (также и по функционалу). 
\\[0.5cm]
Подробнее о клиентской части позже, так как изначально разрабатывается часть для мастера.
\subsection{Описание компонента для мастера}
Мастер/салон может зарегистрировать себя на сайте, получив доступ к определенной части сайта, недоступной для обычного пользователя. Эта часть называется "кабинет". В кабинете мастер/салон может вести всю свою предпринимательскую деятельность, связанную с бьюти-услугами - принимать онлайн-заявки от клиентов, просматривать финансовые показатели, определять финансовые параметры услуг, проводить промо-акции, вести учетную запись клиентов, проводить рассылку для пользователей и многое другое (конкретный перечень в списке описания кабинета).
\subsection{Пользователи}
Пользователями системы являются люди, которые хотят получить бьюти-услуги (стрижку, маникюр, массаж...), а также люди, эти услуги предоставляющие (это могут быть отдельные мастера, а также салоны) - в дальнейшем будет использоваться обозначение "мастер", но под этим подразумевается как и фрилансер, так и компания.


\section{Технические уточнения}

\subsection{Термины}
    \begin{enumerate}
    
    \item hello

        \begin{enumerate}
            \item test
                \begin{enumerate}
                    \item further test
                \end{enumerate}
        \end{enumerate}
        \item Блок - некий визуальный элемент, выделяющийся либо графически (в виде рамок, очертаний), либо по смыслу (совокупность похожих элементов)
        \item Компонент - часть содержания, имеющего закрытое визуально представление. Одна страница сайта может состоять из нескольких компонентов.
        \item Фронтэнд - для пользователя видимая оболочка веб-страницы
        \item Бэкэнд - невидимые для пользователя математические алгоритмы
        \item CMS - все работы происходят на основе системы управления содержанием - CMS 1C Bitrix (1С Битрикс)
        \item Модуль - является описанием общего функционала, который не может быть классифицирован как привязанный к определенной странице. Он может встречаться на любой странице в любом месте. Модуль может состоять из нескольких компонентов. Также модуль может содержать в себе части логики фронтэнда и бэкэнда.
        \item Хэдер - верхняя часть сайта, обладающая определенной структурой, которая видна сквозняком на всех или почти всех страницах сайта. Также используется обозначение "шапка".
        \item Футер - нижняя часть сайта. Функционал аналогичен хэдеру. Также используется обозначение "подвал".
    \end{enumerate}


\subsection{Технические требования к сайту}
Сайт должен быть адаптивным, поддерживать IE 8+ и использовать технологию композита.
Тут ещэ больше технического бла-бла.


\section{Модули}
Каждую страницу, а также определенные составляющие страниц можно отключить или включить для разных групп пользователей по желанию администратора сайта. Так например модуль 'Финансы' включает в себя как и отдельную страницу со сводкой финансов, так и виджеты на других страницах, показывающие короткую статистику.
\subsection{}


\section{Страницы}

\subsection{Сайт для клиента}

\subsection{Кабинет мастера}

\subsubsection{Кокпит мастера}
Данный раздел кабинета мастера представляет собой сводку важных действий и информации для работы мастера. Делится на несколько частей - быстра сводка с фильтрами по дате, очередь клиентов, быстрая статистика по услугам и мастерам, быстрая статистика по финансам и клиентам.

\paragraph{Быстрая сводка} 
состоит из - под кнопки оформленных - блоков, показывающих следующую информацию:

    \begin{enumerate}          
        \item Клиенты на подходе - кол-во клиентов, которые придут в ближайшие X минут, где X - можно управлять в настройках кабинета. Стандартно задается 60 минут.
        \item Загрузка - рассчитывается из графика мастеров на основе выбранного фильтра по времени. Если фильтр - 'сегодня', то расчет происходит следующим образом: рабочие-часы-мастера/кол-во-заказанных-услуг*(время-услуги+пауза-между-услугой). Формально выглядит так: 
            $$
            L_m = \frac{W_m}{\sum n_m * (l_n + P)}
            $$
Где $L_m$ = загрузка конкретного мастера, $W_m$ = рабочие часы мастера (управляется в настройках), $n_m$ = услуга $n$, забронированная на мастера $m$, $l_n$ = длинна услуги $n$ в минутах (управляется в настройках услуг), $P$ = время перерыва (управляется в настройках, по умолчанию равно 10 минутам). $n$ идет от 1 до $N$, где $N =$ последняя услуга дня. $m$ идет от 1 до $M$, где $M$ = общее кол-во занятых мастеров.
\\[0.5cm]
При выборе других фильтров расчет происходит идентично, только $n$ охватывает не только услуги одного дня, а все услуги выбранного фильтра.
        \item Ожидаемая прибыль за выбранный период времени является ожидаемой суммой по всем (предварительным) заказам на выбранный период. Заказ всегда является предварительным до тех пор, пока мастер не подтвердит фактический приход клиента. Формальная формула для одного мастера:
            
            $$
            I_m = n_m * c_n
            $$
Где $I_m$ = возможный приход мастера за выбранный период, $c_n$ = стоимость услуги $n$. Общий доход является суммой всех доходов по каждому мастеру:

            $$
            I = \sum_1^M I_m
            $$

        \item \label{paragraph:add_new_client}Добавить нового клиента/запись - вызывает всплывающее окно, позволяющее быстро создать нового клиента (с записью если необходимо). Такая функция необходима, если клиент не использует онлайн форму, а напрямую звонит или приходит в салон. При этом всплывающее окно позволяет внести ФИО и телефон (и если находит уже существующего человека по какому-либо из этих параметров, то позволяет выбрать его и прикрепить запись на услугу к конкретному лицу, если не находит, то создает новую запись). Также можно ввести мастера, если клиент сразу знает, к кому хочет попасть (при этом всплывает ещё одно окно, показывающее распорядок дня этого мастера), если нет, то показывается календарь на день, который выбрал клиент всех мастеров.

    \end{enumerate}

\paragraph{Очередь клиентов} \label{paragraph:plashki}
После быстрой сводки следует блок с очередью клиентов. В данном блоке плашками отображаются ближайшие клиенты на подходе. Плашки сортируются по дате актуальности, т.е. самые актуальные клиенты, которые вот-вот должны подойти отображаются сверху. Если клиент выбрал при онлайн-заявке уже конкретного мастера, то фотография мастера также отображается в плашке справа. Если клиент мастера не выбрал, то сотрудник может сам назначить мастера для клиента. Для этого он должен нажать на плашку и в правой области блока отобразятся фотографии мастеров, которые доступны для выбранной клиентом услуги в выбранное клиентом время. Сотрудник затем перетягивает фотографию мастера на плашку клиента, тем самым закрепляя мастера за клиентом. Для фрилансеров по-умолчанию работает автоприкрепление, так как они являются единственными мастерами.
\\[0.5cm]
В плашке можно подтвердить приход клиента или сообщить о его отказе (при подтверждении и отказе перерасчитывается финансовые параметры, так как оценочная сумма обновляется уже подтвержденными данными). Автоматически отказы не происходят. Так, мастера, которые забыли подтвердить приходы клиентов, или если у них не хватило времени, могут сделать это в любое удобное время. 
\\[0.5cm]
В плашке отображается дополнительно следующая информация по клиенту:

    \begin{enumerate}
        \item Иконка с полом (М/Ж)
        \item Категория услуги и услуга (например: волосы - окрашивание)
        \item Имя клиента
        \item Время забронированной услуги
    \end{enumerate}
    
\paragraph{Быстрая статистика по услугам и клиентам}
Данная область позволяет получить быстрый доступ к самым востребованным данным - статистике по услугам и статистике по мастерам. Область поделена на два блока (услуги и мастера). В будущем область будет настриваемой, что бы пользователи системы могли выводить в области некие виджеты.

\subparagraph{Услуги}
Для услуг необходимо отображать топ X (X управляется в настройках, по умолчанию равно 10) услуг, которыми пользуются клиенты. При этом необходимо считать как и брони (т.е. даже если бронь отменяется, в счетчик услуги идет +1) так и мануально созданные услуги.

\subparagraph{Мастера}
Аналогично предыдущей области, если клиент выбрал мастера при онлайн-заявке или нзвал мастера сразу при звонке, считается рейтинг мастеров.

\paragraph{Быстрая статистика по финансам и посетителям}
Данная область расположена под блоком очередь клиентов. Область в будущем также будет настриваемой, что бы пользователи системы могли подключить свои собственные виджеты.
\subparagraph{Финансы}
Статистика по финансам показывает сколько денег было заработано через онлайн-заявки (т.е. сколько денег было заработано по полному пройденному циклу заявки - от получения до подтверждения прихода клиента).
\subparagraph{Посетители}
Статистика посетителей показывает сколько клиентов просмотрело на сайте ББ любые доступные услуги данного мастера.

\subsubsection{Календарь}
Календарь позволяет просматривать загруженность всех мастеров на день или неделю. Дополнительной фильтрации календарь не предусматривает (даже при двух мастерах вид на месяц будет заграможден), но при этом пользователь может переходить между неделями/днями. Также пользователь может скроллить между мастерами, если их столбцы не помещаются на одном экране.

\paragraph{Поиск мастера}
Для больших студий актуально - поиск мастера. Поиск происходит от двух букв и ищет по имени, фамили или отчеству. При этом результат поиска можно сохранить под определенным названием (например можно найти двух мастеров и сохранить их в группе 'массажисты').
\paragraph{Добавить нового пользователя}
Функционал полностью дублирует пункт \ref{paragraph:add_new_client}
\paragraph{Выбор даты}
Небольшой виджет календаря в левом сайдбаре позволяет выбрать конкретную дату, на которую при выборе пролистывается основной календарь. Также виджет поддерживает ссылки на недели (т.е. можно нажать на неделю и календарь пролистает до этой недели).
\paragraph{Список заявок}
Все заявки без мастера видны в этом списке, также в левом сайдбаре. Заявки представляют собой похожий на плашки (пункт \ref{paragraph:plashki}) функционал, только в обрезанном виде. Их можно перетаскивать в область календаря и 'кидать' на мастеров.
\paragraph{Вид календаря}
Календарь представлен по умолчанию видом на день, где в первом столбце идут временные отсеки (по умолчанию один отсек равен часу, т.е. 8-9, 9-10, 10-11 и т.д.). Последующие столбцы отображают (каждый столбец - одного) мастеров.
\\[0.5cm]
Столбцы 2...n проскролливаются, если не помещаются полноценно на всем экране. При этом у ширины есть минимальное значение и максимальное, которое является областью календаря (минус ширину первого столбца), поделенного на кол-во столбцов за вычетом первого. Пример - 1 мастер: календарь состоит из двух столбцов (время и мастер), второй столбец занимает 100\% ему отведенной ширины (это ширина всей области под календарь минус ширина первого столбца). Если мастеров 2, то каждый занимает по 50\% ширины. Но это идет не до бесконечности, а до заданного минимального значения. 


\subsubsection{Финансы}
На странице расположены три типа фильтра и область для показа (визуального) результата (типов всего три - столбцы, пирог и диаграмма). Фильтры трех типов:
По направлению - затраты против доходов, оборот, прибыль
По параметрам компании/клиентов - услуги, пол клиента, сотрудник, марка косметики
По дате - неделя, месяц, квартал, год
\\[0.5cm]
Все три типа можно совмещать, т.е. получить например график оборота по марке косметики за квартал или затраты vs. доходы по полу клиента. 
\\[0.5cm]
По разным направлениям показываем разные типы визуальных результатов. Например для затраты против доходов для нескольких сотрудников пироговый чарт не имеет смысла. Правило, как показывать когда какие чарты - всегда видна диаграмма+столбцы. Если направление выбрано 'затраты против расходов' и в параметрах более одного наименования, то не показываем pie-chart. Во всех остальных случаях показываем также pie-chart.


\subsubsection{Клиенты}

Меню действий для клиентов
Поиск по клиентам
Список клиентов
Что делать если нажать на клиента

\subsubsection{Склад}

\subsubsection{Помощь}

\subsubsection{Отчеты}

\subsubsection{Настройки}

\section{Примеры использования}











\end{document}