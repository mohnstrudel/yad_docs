\documentclass[DIV=calc, paper=a4, fontsize=11pt]{scrartcl} % Документ принадлежит классу article, а также будет печататься в 12 пунктов.
\usepackage{ucs}
\usepackage[T1,T2A]{fontenc}
\usepackage[utf8x]{inputenc} % Включаем поддержку UTF8
\usepackage[russian]{babel} % Пакет поддержки русского языка
\usepackage{titling} % Allows custom title configuration

%for frames
\usepackage{framed}

%For image using
\usepackage{graphicx}

%Numbering subsubsubsections etc
\setcounter{secnumdepth}{5}


%Further enumeration
\usepackage{enumitem}
\setenumerate[1]{label=\theparagraph.\arabic*.}
\setenumerate[2]{label*=\arabic*.}
\setenumerate[3]{label*=\arabic*.}


%For referencing within enumeration lists
\usepackage{enumitem}

%Packages for word-like comment style
\usepackage{todonotes}

%Package for images
\usepackage{float}
\floatstyle{boxed}
\restylefloat{figure}

%For a nicer reference
\usepackage{fancyref}

\usepackage{titlesec}


\titleformat*{\section}{\LARGE\bfseries}
\titleformat*{\subsection}{\Large\bfseries}
\titleformat*{\subsubsection}{\large\bfseries}
\titleformat*{\paragraph}{\large\bfseries}
\titleformat*{\subparagraph}{\large\bfseries}

%For some math formulas if needed
\usepackage{mathtools}

% Some nice visualization
%\usepackage[svgnames]{xcolor} % Enabling colors by their 'svgnames'
\usepackage{fullpage}
%\renewcommand{\headrulewidth}{0.0pt} % No header rule
%\renewcommand{\footrulewidth}{0.4pt} % Thin footer rule
% End visualization

%smart enumeration
\renewcommand{\labelenumi}{\arabic{enumi}.}
\renewcommand{\labelenumii}{\arabic{enumi}.\arabic{enumii}}


\title{Макет технического задания для салонов красоты (рабочее название - Beauty Booking или BB)}
\date{07/10/2014}

\begin{document}

\maketitle

\section{Описание проекта}
Сайт представляет собой платформу, на которой клиенты могут искать бьюти-услуги, а салоны и фрилансеры предлагать такие услуги. Сайт состоит из двух компонент - компонент для клиента представляет собой удобный поиск и бронирование похода к мастеру. Для мастера (салон или фрилансер) сайт представляет собой полноценный административный рабочий инструмент. Мастер может принимать заявки в режиме онлайн, вести учетную запись клиентов, получать заказы с сайта (с компонента для клиентов), продвигать свои услуги.
\subsection{Описание компонента для клиента}
Для клиента сервис собой представляет красивый сайт, позволяющий искать услуги по запросу клиента. Особенностью сайта является его полная адаптивность и облик мобильного приложения, который полностью копирует мобильную версию сайта (также и по функционалу). 
\\[0.5cm]
Подробнее о клиентской части позже, так как изначально разрабатывается часть для мастера.
\subsection{Описание компонента для мастера}
Мастер/салон может зарегистрировать себя на сайте, получив доступ к определенной части сайта, недоступной для обычного пользователя. Эта часть называется "кабинет". В кабинете мастер/салон может вести всю свою предпринимательскую деятельность, связанную с бьюти-услугами - принимать онлайн-заявки от клиентов, просматривать финансовые показатели, определять финансовые параметры услуг, проводить промо-акции, вести учетную запись клиентов, проводить рассылку для пользователей и многое другое (конкретный перечень в списке описания кабинета).
\subsection{Пользователи}
Пользователями системы являются люди, которые хотят получить бьюти-услуги (стрижку, маникюр, массаж...), а также люди, эти услуги предоставляющие (это могут быть отдельные мастера, а также салоны) - в дальнейшем будет использоваться обозначение "мастер", но под этим подразумевается как и фрилансер, так и компания.
\subsection{Конкуренты}
Текущие конкуренты сайта:
\\[0.5cm]
http://hairbook.ru - русскоязычная аудитория
\newline http://wahanda.com - аудитория Великобритании
\newline http://styleseat.com - американская аудитория

\section{Технические уточнения}

\subsection{Термины}

        \begin{itemize}
        \item Блок - некий визуальный элемент, выделяющийся либо графически (в виде рамок, очертаний), либо по смыслу (совокупность похожих элементов)
        \item Компонент - часть содержания, имеющего закрытое визуально представление. Одна страница сайта может состоять из нескольких компонентов.
        \item Фронтэнд - для пользователя видимая оболочка веб-страницы
        \item Бэкэнд - невидимые для пользователя математические алгоритмы
        \item CMS - все работы происходят на основе системы управления содержанием - CMS 1C Bitrix (1С Битрикс)
        \item Модуль - является описанием общего функционала, который не может быть классифицирован как привязанный к определенной странице. Он может встречаться на любой странице в любом месте. Модуль может состоять из нескольких компонентов. Также модуль может содержать в себе части логики фронтэнда и бэкэнда.
        \item Хэдер - верхняя часть сайта, обладающая определенной структурой, которая видна сквозняком на всех или почти всех страницах сайта. Также используется обозначение "шапка".
        \item Футер - нижняя часть сайта. Функционал аналогичен хэдеру. Также используется обозначение "подвал".
    \end{itemize}


\subsection{Технические требования к сайту}
Сайт должен быть адаптивным, поддерживать IE 8+ и использовать технологию композита.
Тут ещё больше технического бла-бла.


\section{Модули}
Каждую страницу, а также определенные составляющие страниц можно отключить или включить для разных групп пользователей по желанию администратора сайта. Так например модуль 'Финансы' включает в себя как и отдельную страницу со сводкой финансов, так и виджеты на других страницах, показывающие короткую статистику.
\subsection{Виджет для собственных сайтов} \label{subsection:widget}
Если у салона уже есть собственный сайт, то он может установить сниппет - js код - который позволит пользователям через данный виджет бронировать услуги. Данная бронь попадает, как и через главный сайт getbam в кабинет салона.
\\[0.5cm]
С технической точки зрения виджет должен отправлять post-request с параметрами салона (id) и параметрами клиента. 
Виджет состоит из выбора услуги, доступной в данном салоне, выбора времени и мастера (опционально).
В качестве настроек виджету можно задать базовые css-параметры, отвечающие за цвет (background) виджета, форму и цвет кнопки отправки. Также для более продвинутых пользователей нужно предоставить возможность редактировать css вручную.

\subsection{Проверка салона на честность}
Возможная проблема - салон записывает клиентов через систему, но как подходит очередь клиента, администратор салона (или фрилансер) нажимает на кнопку отказа (=клиент не появился). Для пресечения подобных мер необходимо реализовать модуль мониторинга, который будет агрегировать информацию по салонам и предоставлять её нам (как главным администраторам).

\subsubsection{Доверительный интервал}
Самая главная настройка 

\subsection{Чат}
Информация дополняется

\subsection{Сайт салона}
Конструктор сайта для салона. Позволяет задать базовые параметры по дизайну (цветовая гамма, вид кнопок), а также по наполнению (области для информации сайта - баннеры, услуги, акции и прочее). Данный функционал является наименее приоритетным.

\section{Страницы}

\subsection{Сайт для клиента}

\subsubsection{Главная страница}

На главной странице, идя сверху вниз присутствуют несколько логических блоков. Первый - верхнее меню, позволяющее пользователю выбрать из категорий самых распространенных видов услуг. При нажатии на ссылку будет выпадать прямоугольный блок, который в свою очередь несет свой набор действий (утверждается). Также на уровне верхнего меню есть поиск и кнопки входа/регистрации для компании (регистрации для пользователя как такогого нет - пользователь может нажать на "войти" и там создать новую учетную запись).
\\Второй блок - область баннера, также содержит форму поиска процедуры. В самом низу данного блока ещё раз повторяется меню, аналогичное первому блоку. Пункты данного меню также открывают прямоугольные области.
\\Третий блок - рекомендации редакции. Некий слайдер топовых услуг.
\\Четвертый блок - меняем информацию под собственные нужды.
\\Пятый блок - топовые предложение от топовых студий (проплачивается студиями на основе хитрого алгоритма).
\\Шестой блок - футер: содержит подписку на ньюслеттер, ссылки на скачивание мобильных приложений, ссылки на соц.сети, ссылка на подстраницы.

\paragraph{Верхнее меню}
При нажатии на один из пунктов верхнего меню появляется прямоугольная область, содержащая (утверждается - скорее всего ссылки на подкатегории, а также лучшие предложения из данной категории. Например если ссылка верхнего меню - "Волосы", то в прямоугольной области перечнем могут идти "Стрижка", "Наращивание волос", "Наворачивание бигудей". В качестве лучших предложений можно выводить салоны или рекламные продукты или ещё что-то, связанное с волосами). Прямоугольная область слегка прозрачная и открывается \textbf{под} верхнем меню. 
\\[0.5cm]
Поиск - позволяет искать по таким областям, как - салоны, места, услуги, \todo[inline]{мастера - нужны?}

Войти/Зарегистрировать бизнес - при нажатии на "войти" открывается также прямоугольная область, внутри которой расположена форма для входа. Также в этой области расположена кнопка "создать учетную запись", которая ведет на новую страницу, на которой пользователю доступны куча полей для непосредственно регистрации.
При нажатии на "зарегистрировать бизнес" открывается новая страница, аналогичная регистрации для простого смертного, только отличающаяся по кол-ву полей. Точный перечень утверждается (например: название компании, адрес, имя пользователя (по умолчанию становится администратором), почтовый адрес, тип бизнеса (салон, спа-центр, кожанная клиника...), телефон, некоторая инфа для статистики (например - как вы до сих пор управляете заказами, ваша главная причина для данной регистрации))

\paragraph{Область баннера}
Кол-во салонов и кол-во услуг выводим динамически

\subsubsection{Идеи}

\paragraph{Ответственное бронирование}
Вступительный текст для пользователя - "Просьба бронировать ответственно. Часто салоны представляют собой независимых мастеров, для которых каждый отказ является сильным ударом. Мы знаем, что планы меняются, и если вы не в состоянии прийти на встречу, пожалуйста, поступите правильно и предупредите их как можно раньше. Если вы выбрали опцию "оплатить при встрече" и не появились в салоне, то в следующий раз жта опция будет заблокирована для вас. Надеемся на ваше понимание.

\paragraph{Проверка на доступность мастера}
При отправки подтверждения на услугу программа должна проверять доступность мастера в указанное клиентом время. При этом сама проверка проходит непосредственно после нажатия на кнопку "подтвердить бронь". Если возникает ошибка, то клиенту показывается сообщение об ошибке "К сожалению кто-то быстрее вас забронировал эту услугу. Пожалуйста, выберите другое время или другого мастера". Клиент нажимает на "ок" и возвращается к форме выбора времени и мастера, все другие данные при этом должны сохраняться.
\\[0.5cm]
Это задумано для того, что бы избежать двойных занесений если 1) другой пользователь быстрее забронировал эту услугу у того же мастера в то же время, что и первый пользователь или 2) салон сам занес нового клиента к тому же мастеру в то же время.


\subsection{Кабинет студии}

\subsubsection{Кокпит студии}
Данный раздел кабинета мастера представляет собой сводку важных действий и информации для работы мастера. Делится на несколько частей - быстра сводка с фильтрами по дате, очередь клиентов, быстрая статистика по услугам и мастерам, быстрая статистика по финансам и клиентам.

\paragraph{Быстрая сводка} 
состоит из - под кнопки оформленных - блоков, показывающих следующую информацию:

    \begin{enumerate}          
        \item Клиенты на подходе - кол-во клиентов, которые придут в ближайшие X минут, где X - можно управлять в настройках кабинета. Стандартно задается 60 минут.
        \item Загрузка - рассчитывается из графика мастеров на основе выбранного фильтра по времени. Если фильтр - 'сегодня', то расчет происходит следующим образом: рабочие-часы-мастера/кол-во-заказанных-услуг*(время-услуги+пауза-между-услугой). Формально выглядит так: 
            $$
            L_m = \frac{W_m}{\sum n_m * (l_n + P)}
            $$
Где $L_m$ = загрузка конкретного мастера (в процентах), $W_m$ = рабочие часы мастера (управляется в настройках), $n_m$ = услуга $n$, забронированная на мастера $m$, $l_n$ = длинна услуги $n$ в минутах (управляется в настройках услуг), $P$ = время перерыва (управляется в настройках, по умолчанию равно 10 минутам). $n$ идет от 1 до $N$, где $N =$ последняя услуга дня. $m$ идет от 1 до $M$, где $M$ = общее кол-во занятых мастеров.
\\[0.5cm]
При выборе других фильтров расчет происходит идентично, только $n$ охватывает не только услуги одного дня, а все услуги выбранного фильтра.
        \item Ожидаемая прибыль за выбранный период времени является ожидаемой суммой по всем (предварительным) заказам на выбранный период. Заказ всегда является предварительным до тех пор, пока мастер не подтвердит фактический приход клиента. Формальная формула для одного мастера:
            
            $$
            I_m = n_m * c_n
            $$
Где $I_m$ = возможный приход мастера за выбранный период, $c_n$ = стоимость услуги $n$. Общий доход является суммой всех доходов по каждому мастеру:

            $$
            I = \sum_{m=1}^M I_m
            $$

        \item \label{paragraph:add_new_client}Добавить нового клиента/запись - вызывает всплывающее окно, позволяющее быстро создать нового клиента (с записью если необходимо). Такая функция необходима, если клиент не использует онлайн форму, а напрямую звонит или приходит в салон. При этом всплывающее окно позволяет внести ФИО и телефон (и если находит уже существующего человека по какому-либо из этих параметров, то позволяет выбрать его и прикрепить запись на услугу к конкретному лицу, если не находит, то создает новую запись). Также можно ввести мастера, если клиент сразу знает, к кому хочет попасть (при этом всплывает ещё одно окно, показывающее распорядок дня этого мастера), если нет, то на этом регистрация аппойнтмента заканчивается.

    \end{enumerate}

\paragraph{Очередь клиентов} \label{paragraph:plashki}
После быстрой сводки следует блок с очередью клиентов. В данном блоке плашками отображаются ближайшие клиенты на подходе. Плашки сортируются по дате актуальности, т.е. самые актуальные клиенты, которые вот-вот должны подойти отображаются сверху. Если клиент выбрал при онлайн-заявке уже конкретного мастера, то фотография мастера также отображается в плашке справа. Если клиент мастера не выбрал, то сотрудник может сам назначить мастера для клиента. Для этого он должен нажать на плашку и в правой области блока отобразятся фотографии мастеров, которые доступны для выбранной клиентом услуги в выбранное клиентом время. Сотрудник затем перетягивает фотографию мастера на плашку клиента, тем самым закрепляя мастера за клиентом. Для фрилансеров по-умолчанию работает автоприкрепление, так как они являются единственными мастерами.
\\[0.5cm]
В данную очередь попадают клиенты, как занесенные мануально администратором (См. пункт \ref{paragraph:add_new_client}), как и пришедшие с сайта getbam или через виджет на собственном сайте салона (См. пункт \ref{subsection:widget})
\\[0.5cm]
В плашке можно подтвердить приход клиента или сообщить о его отказе (при подтверждении и отказе перерасчитывается финансовые параметры, так как оценочная сумма обновляется уже подтвержденными данными). Автоматически отказы не происходят. Так, мастера, которые забыли подтвердить приходы клиентов, или если у них не хватило времени, могут сделать это в любое удобное время. 
\\[0.5cm]
В плашке отображается дополнительно следующая информация по клиенту:

    \begin{enumerate}
        \item Иконка с полом (М/Ж)
        \item Категория услуги и услуга (например: волосы - окрашивание)
        \item Имя клиента
        \item Время забронированной услуги
    \end{enumerate}
    
\paragraph{Быстрая статистика по услугам и клиентам}
Данная область позволяет получить быстрый доступ к самым востребованным данным - статистике по услугам и статистике по мастерам. Область поделена на два блока (услуги и мастера). В будущем область будет настриваемой, что бы пользователи системы могли выводить в области некие виджеты.

\subparagraph{Услуги}
Для услуг необходимо отображать топ X (X управляется в настройках, по умолчанию равно 10) услуг, которыми пользуются клиенты. При этом необходимо считать как и брони (т.е. даже если бронь отменяется, в счетчик услуги идет +1) так и мануально созданные услуги.

\subparagraph{Мастера}
Аналогично предыдущей области, если клиент выбрал мастера при онлайн-заявке или назвал мастера сразу при звонке, считается рейтинг мастеров.

\paragraph{Быстрая статистика по финансам и посетителям}
Данная область расположена под блоком очередь клиентов. Область в будущем также будет настриваемой, что бы пользователи системы могли подключить свои собственные виджеты.
\subparagraph{Финансы}
Статистика по финансам показывает сколько денег было заработано через онлайн-заявки (т.е. сколько денег было заработано по полному пройденному циклу заявки - от получения до подтверждения прихода клиента).
\subparagraph{Посетители}
Статистика посетителей показывает сколько клиентов просмотрело на сайте ББ любые доступные услуги данного мастера.

\subsubsection{Календарь}
Календарь позволяет просматривать загруженность всех мастеров на день или неделю. Дополнительной фильтрации календарь не предусматривает (даже при двух мастерах вид на месяц будет заграможден), но при этом пользователь может переходить между неделями/днями. Также пользователь может скроллить между мастерами, если их столбцы не помещаются на одном экране.

\paragraph{Поиск мастера}
Для больших студий актуально - поиск мастера. Поиск происходит от двух букв и ищет по имени, фамили или отчеству. При этом результат поиска можно сохранить под определенным названием (например можно найти двух мастеров и сохранить их в группе 'массажисты').
\paragraph{Добавить нового пользователя}
Функционал полностью дублирует пункт \ref{paragraph:add_new_client}
\paragraph{Выбор даты}
Небольшой виджет календаря в левом сайдбаре позволяет выбрать конкретную дату, на которую при выборе пролистывается основной календарь. Также виджет поддерживает ссылки на недели (т.е. можно нажать на неделю и календарь пролистает до этой недели).
\paragraph{Список заявок}
Все заявки без мастера видны в этом списке, также в левом сайдбаре. Заявки представляют собой похожий на плашки (пункт \ref{paragraph:plashki}) функционал, только в обрезанном виде. Их можно перетаскивать в область календаря и 'кидать' на мастеров.
\paragraph{Вид календаря}
Календарь представлен по умолчанию видом на день, где в первом столбце идут временные отсеки (по умолчанию один отсек равен часу, т.е. 8-9, 9-10, 10-11 и т.д.). Последующие столбцы отображают (каждый столбец - одного) мастеров.
\\[0.5cm]
Столбцы 2...n проскролливаются, если не помещаются полноценно на всем экране. При этом у ширины есть минимальное значение и максимальное, которое является областью календаря (минус ширину первого столбца), поделенного на кол-во столбцов за вычетом первого. Пример - 1 мастер: календарь состоит из двух столбцов (время и мастер), второй столбец занимает 100\% ему отведенной ширины (это ширина всей области под календарь минус ширина первого столбца). Если мастеров 2, то каждый занимает по 50\% ширины. Но это идет не до бесконечности, а до заданного минимального значения. 


\subsubsection{Финансы}\label{subsubsec:finance}
На странице расположены три типа фильтра и область для показа (визуального) результата (типов всего три - столбцы, пирог и диаграмма). Фильтры трех типов:
По направлению - затраты против доходов, оборот, прибыль
По параметрам компании/клиентов - [пусто], услуги, пол клиента, сотрудник, марка косметики
По дате - неделя, месяц, квартал, год
\\[0.5cm]
Все три типа можно совмещать, т.е. получить например график оборота по марке косметики за квартал или затраты vs. доходы по полу клиента. 
\\[0.5cm]
По разным направлениям показываем разные типы визуальных результатов. Например для затраты против доходов для нескольких сотрудников пироговый чарт не имеет смысла. Правило, как показывать когда какие чарты - всегда видна диаграмма+столбцы. Если направление выбрано 'затраты против расходов' и в параметрах более одного наименования, то не показываем pie-chart. Во всех остальных случаях показываем также pie-chart. (При выборе параметра 'пусто' график строится без фильтрации по услугам, клиентам и прочего.


\subsubsection{Клиенты}
На данной странице отображены клиенты. Помимо табличного представления самих клиентов на странице возможен поиск клиентов, набор определенных быстрых действий (касающихся всех клиентов). 

\paragraph{Быстрые действия с клиентами} Также на странице можно совершить действия, такие как: создать рассылку, что-то ещё \todo[inline]{Что можно ещё добавить? Скидки определенным клиентам?}

\paragraph{Поиск} По клиентам должен работать поиск, поиска принимает фио, мэйл, дату визита и телефон. Он должен работать в режиме подсказки и без строго заданной маски. Т.е. телефон +7 903 227-88 74 должен найтись и по 8874, и по 8903227 и по 8 903 227 88 74 и по 8(903)227-88. 

\paragraph{Обзор клиентов} Клиенты выводятся на данной странице в табличной форме. Из столбцов доступны - ФИО, мэйл, телефон, последнее посещение, оборот с клиента (оборот настраивается, видно ли всем или нет. По умолчанию видно администратору), кол-во букингов

\paragraph{Опции каждого клиента} Нажав на конкретного клиента администратор сайта может открыть его карточку - на карточке доступны такие действия, как просмотреть и одновременно редактировать клиента, назначить аппойнтмент или удалить его. При просмотре данные подгружаются сразу в формате редактирования и доступны поля, такие как - аллергия, любимые направления, примечания (примечания видны при назначении нового апойнтмента), имя, телефон, эл.почта, подтверждение о согласии на получение рассылки, пол, день рождения. Также для клиента можно назначить из его карточки новый аппойнтмент.

\subsubsection{Склад}
Пока разрабатывается

\subsubsection{Помощь}
На данной странице администратор сайта может просмотреть номер службы техподдержки, а также направить писменный запрос. Для всех запросов заводятся тикеты, статус которых пользователь может просматривать. Тикеты делятся на актуальные, закрытые и отклоненные.

\subsubsection{Отчеты}
Отчеты представляют собой расширенный список данных по финансовым параметрам. Они представляют собой таблицы, по которым строятся графики из пункта (\ref{subsubsec:finance}). Таблицы можно получить по следующим данным: услуга, клиент, пол клиента, мастер, дата, время(длительность), прибыль. 
\\[0.5cm]
Для пользователя по умолчанию доступны следующие быстрые фильтры для отчетов: 
    \begin{itemize}
        \item Расходы vs. доходы/Прибыль/Оборот за месяц за месяц/квартал/полугодие/год
        \item 
        \item 
    \end{itemize}


\subsubsection{Настройки салона}
В настройках можно добавлять/удалять мастеров, назначать каждому мастеру заранее указанные услуги и изменять общие данные мастера. Мастеров можно группировать (например начинающие мастера, продвинутые или профи)
\\[0.5cm]
Также можно добавлять услуги, которые предоставляет салон (эти же услуги можно привязывать к мастерам). Для каждой услуги можно установить свой уровень цен (либо единый для всех, либо для различных мастеров/групп мастеров цены разные)
\\[0.5cm]
Рабочее время - устанавливается рабочее время салона, а также допустимые паузы между приемами мастера.
Сервисы оплаты позволяют добавить различные способы оплаты, которые становятся доступными клиенту. Также каждому мастеру можно установить "блоки неработы", т.е. временные промежутки, которые могут быть произволными, во время которых мастер не работает (так, например, можно установить обеденный перерыв или короткую пятницу или ещё что-то)

\subsubsection{Общие настройки (личный кабинет)}
В общих настройках имеется информация об учетной записи салона - номер лицевого счета, информация о тарифе и возможности перехода между тарифными планами, информация о купленных смс-уведомлениях, история оплаты (коротко, по месяцам и детально, обзор каждого платежа за месяц), документы (акты и счета-фактуры), рейтинг и комментарии (клиенты могут после посещения авторизоваться на сайте и если мастер подтвердил их приход (а точнее не отклонил, так как автоподтверждение происходит по истечении 30и минут), то клиент может оставить отзыв о салоне) - в комментариях у салона должна быть возможность ответить на комментарий, а также пожаловаться на него (при этом создается тикет в службу техподдержки)


\subsection{Кабинет мастера-одиночки}
Данный кабинет фактически мало чем отличается от кабинета студии. Все настройки мастера-одиночки являются настройками студии по умолчанию, если занесен только один мастер.

\subsection{Кабинет администратора сети салонов}
Кабинет для управления сразу несколькими салонами. Работает следующим образом - у администратора сети есть доступ к версии кабинета, которая похожа на кабинет салона, за отсутствием оперативных действий (администратор сети не может распределять клиентов например). 

\subsubsection{Сотрудники и салоны}
Аналогично закладке "клиенты" у салона для администратора дополнительно ещё одна закладка - "сотрудники и салоны". Здесь он может просмотреть всех сотрудников, отфильтровать их по салонам, найти нужного сотрудника, посмотреть некую сводку по салонам (сколько сотрудников, какой оборот, какая загруженность мастеров) \todo[inline]{Надо будет ещё опций придумать}
Также администратор может создавать новые салоны и заносить сотрудников в новые салоны. При этом администратор может создавать группы для салонов (и на основе этих групп проводить дополнительную фильтрацию результатов). Например можно создать группу "премиум салоны" или группу "бюджетные салоны".

\subsubsection{Связь существующих салонов}
Для существующих салонов возможна двойная идентификация. Сначала салон может в настройках занести "материнскую" компанию, на основе этого администратор материнской компании получает заявку на подвтерждение. Аналогично обратное - администратор головного офиса может подать заявку на подключение салона, и салон в свою очередь должен подтвердить эту заявку.


\section{Примеры использования}



\begin{framed}
    \subsection{Администратор салона Оксана}
    Оксана получила задание от владельца бизнеса продвинуть его салон в онлайн-сфере. Погуглив Оксана обнаружила сервис бьютибукинга, который заманивает большим кол-вом скаченных приложений (для мобильных телефонов) и большим кол-вом заказов клиентов через главный сайт бьютибукинга. Она также привлекается бесплатной регистрацией для салонов. Оксана регистрирует свой салон на сайте заполнив три короткие формы - название+адрес+время работы, услуги, мастера. При этом она положительно отмечает, что любой шаг можно пропустить для заполнения позже. 
    \\[0.5cm]
    После короткой регистрации Оксана уже находится в кокпите приложения и может начинать работу. Она быстро знакомится с главными функциями кокпита (и понимает, что для начала для полноценной работы другие разделы сайта ей пока и не нужны) и заносит своего первого клиента. Через некоторое время Оксана видит, что через сайт бьютибукинга к ней поступил клиент. Оксана выбирает клиента и назначает его на мастера Зинаиду. 
    \\[0.5cm]
    Как только клиент пришел в салон, Оксана отмечает факт физического прихода - нажимает на галочку подтверждения. После этого клиент пропадает из очереди клиентов и значение о верятном доходе становится фактическим и идет в отчет системы.
    
\end{framed}

\begin{framed}
    \subsection{Клиент Зинаида}
    Пока разрабатывается
\end{framed}











\end{document}