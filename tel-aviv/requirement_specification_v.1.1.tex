\documentclass[DIV=calc, paper=a4, fontsize=11pt]{scrartcl} % Äîêóìåíò ïðèíàäëåæèò êëàññó article, à òàêæå áóäåò ïå÷àòàòüñÿ â 12 ïóíêòîâ.
\usepackage{ucs}
\usepackage[T1,T2A]{fontenc}
\usepackage[utf8x]{inputenc} % Âêëþ÷àåì ïîääåðæêó UTF8
\usepackage[russian]{babel} % Ïàêåò ïîääåðæêè ðóññêîãî ÿçûêà
\usepackage{titling} % Allows custom title configuration

%For image using
\usepackage{graphicx}

%For framed boxes
\usepackage{framed}

%For referring to names
\usepackage{nameref}

%Include paragraphs into numbering
\setcounter{secnumdepth}{5}

%For referencing within enumeration lists
\usepackage{enumitem}

%Packages for word-like comment style
\usepackage{todonotes}

%Package for nice image include
\usepackage{float}
\floatstyle{boxed}
\restylefloat{figure}

%For a nicer reference
\usepackage{fancyref}

\usepackage{titlesec}

\titleformat*{\section}{\LARGE\bfseries}
\titleformat*{\subsection}{\Large\bfseries}
\titleformat*{\subsubsection}{\large\bfseries}
\titleformat*{\paragraph}{\large\bfseries}
\titleformat*{\subparagraph}{\large\bfseries}

%For some math formulas if needed
\usepackage{mathtools}

% Some nice visualization
%\usepackage[svgnames]{xcolor} % Enabling colors by their 'svgnames'
\usepackage{fullpage}
%\renewcommand{\headrulewidth}{0.0pt} % No header rule
%\renewcommand{\footrulewidth}{0.4pt} % Thin footer rule
% End visualization

%smart enumeration
\renewcommand{\labelenumi}{\arabic{enumi}.}
\renewcommand{\labelenumii}{\arabic{enumi}.\arabic{enumii}}



\title{Макет технического задания} % Заглавие документа
\date{\today} % Дата создания

\begin{document}
\maketitle

\section{Вводная}
Сайт представляет собой электронную версию журнала "Тель-Авив - Москва". Пользователь может просматривать предоставленную информацию - читать новости, записи в блогах, просматривать видео. Также пользователь может принимать активное участие в доступных для него областях сайта - писать комментарии.

\subsection{Обозначения}

\begin{itemize}
\item Блок - некий визуальный элемент, выделяющийся либо графически (в виде рамок, очертаний), либо по смыслу (совокупность похожих элементов)
\item Компонент - часть содержания, имеющего закрытое визуально представление. Одна страница сайта может состоять из нескольких компонентов.
\item Фронтэнд - для пользователя видимая оболочка веб-страницы
\item Бэкэнд - невидимые для пользователя математические алгоритмы
\item CMS - все работы происходят на основе системы управления содержанием - CMS 1C Bitrix (1С Битрикс)
\item Модуль - является описанием общего функционала, который не может быть классифицирован как привязанный к определенной странице. Он может встречаться на любой странице в любом месте. Модуль может состоять из нескольких компонентов. Также модуль может содержать в себе части логики фронтэнда и бэкэнда.
\item Хэдер - верхняя часть сайта, обладающая определенной структурой, которая видна сквозняком на всех или почти всех страницах сайта. Также используется обозначение "шапка".
\item Футер - нижняя часть сайта. Функционал аналогичен хэдеру. Также используется обозначение "подвал".
\end{itemize}

\subsection{Общая структура сайта}

На сайте представлены информационные материалы. Структура каждого материала идентична, различается только (настраевыемыми) параметрами, которые может настраивать автор статьи. Пользователи могут интерагировать со статьями в виде комментариев.

\section{Фронтэнд}
\subsection{Модули}


\begin{enumerate}
\item Статья - главный и самый важный элемент содержания; является страницей, которая может содержать как текст, изображения так и видео. Статья имеет перечень свойств (например категории и рубрики), которые описаны ниже:
    \begin{enumerate}
    \item{Категория} - категория сайта, к которой относится статья. Таких категорий несколько: новости, блоги, видеохроника и события. Не у всех пользователей есть права писать в каждую категорию и субкатегорию. Например новые записи в новости, видеохронику и в события может добавлять только администратор, в то время как в в блоги новые записи может добавлять более широкий круг лиц, также не каждый модератор может писать во все субкатегории блогов (например для субкатегории "еда" есть один круг лиц, которые могут писать туда, для субкатегории "одежда" - другой круг лиц). У каждой категории может быть неограниченное количество \label{subcategory} подкатегорий. Для новостей это может быть "мировые новости" или "новости раввината".
    \item{Рубрика} - представляет собой де факто также категорию, которая не выводится на сайте как таковая, т.е. нет ссылки в меню с названием \quotedblbase{Рубрики}, а сразу выводятся подрубрики (подрубрики находятся на одинаковом системном уровне, что и субкатегории и могут иногда использоваться как синонимы). Таких 13 - календарь, колумнисты, мнение раввина, точка зрения, бизнес, герои, личности, быть евреем, кино, книги, благотворительность, чтение, красота, роскошь, кошерная мода, мировая мода, мода Израиля, дети, еда.
        \begin{figure}[ht!]
        \centering
        \includegraphics[width=90mm]{ss_10-09-2014.jpg}
        \caption{Меню в шапке \label{overflow}}
        \end{figure}
    \item{Короткое описание} - поле, содержащие не полный текст статьи. Применение - см. пункт \ref{announce} на стр. \pageref{announce}.
    \item{Главный текст} - поле, содержащее главный текст статьи. Открывается отдельной страницой и показывает все содержимое.
    \item{Баннеры} - помимо картинок/видео в теле статьи пользователь может загрузить отдельные баннеры, которые исполняют роль визуального заголовка статьи (для каждого баннера пользователь может подгрузить описание, автор сохраняется автоматически)
    \item{Тэги} - для каждой статьи можно сохранять набор тегов
    \item{Рейтинг} - каждая статья может оцениваться пользователями по десятибальной шкале
    \item{Дата, место и участники} - если создается статья в категории "события", то для нее нужно указать дату события и место проведения события. Также можно добавить участников встречи (т.е. пригласить других пользователей). 
    \end{enumerate}
\item \label{announce}{Анонс} (= превью) - анонс представляет собой часть полноценной статьи, которую может написать администратор, модератор или любой другой пользователь, обладающий соответствующими правами.

        \begin{figure}[H]
        \centering
        \includegraphics[width=512px]{tel_aviv_scheme.png}
        \caption{Визуальное оформление структуры сайта. \label{fig:tel_aviv_scheme.png}}
        \end{figure}

\end{enumerate}

\subsection{Страницы}
\subsubsection{Список статей}
Данная страница представляет собой список блоков с информацей, состоящей из предпросмотра баннера , анонса текста (см. пункт \ref{announce} на стр. \pageref{announce}) и системной информации о статье. Блоки идентичны по оформлению и различаются только в содержании. Также над каждым списком расположен элемент фильтрации статей по дате (см. рис. \ref{fig:filtering.jpg} на стр. \pageref{fig:filtering.jpg}).
            \begin{figure}[ht!]
            \centering
            \includegraphics[width=90mm]{filtering.jpg}
            \caption{Страница списка статей с фильтрацией \label{fig:filtering.jpg}}
            \end{figure}

\paragraph{Фильтрация статей}
Данный элемент позволяет листать (элементы стрелок справа и слева) между месяцем одного года. При нажатии на иконку календаря (на месяц+год нажимать нельзя) всплывает элемент календаря (см. верстку или элемент календаря в правом сайдбаре (пункт \ref{main} - \nameref{main}), позволяя выбрать дату. При выборе даты с списке статей подгружаются только те статьи, которые были опубликованы в выбранную дату. При перелистывании также происходит фильтрация - как только пользователь нажимает на стрелку влево/вправо то в списке подгружаются статьи того месяца, на который пользователь перелистнул.
\\[0.5cm]
По умолчанию фильтр не активен и показывает последний месяц текущего года. 
\begin{framed}
Например текущий месяц - октябрь 2014. Если пользователь просто заходит на страницу со списком статей он может пролистать постранично до самого начала всех статей (будь это январь 2013 или март 2011). Если пользователь нажимает в элементе фильтра на стрелку влево, то элемент перелистывает на месяц сентябрь. Теперь в списке отображаются только статьи за сентябрь и если пользователь будет листать страницу со списком, то самая последняя статья будет от самой ранней даты сентября (например первое сентября в 00:01). Дальше сентября пагинация в данном случае идти не будет.
\end{framed}


\begin{enumerate}
    \item Предпросмотр баннера является уменьшенным вариантом баннера, также на данном элементе расположена полоса \label{whiteline}, подгружающая в зависимости от принадлежности статьи различное содержание
    \begin{enumerate}
        \item Новости - название субкатегории
        \item Блоги - автор и субкатегорию блогов
        \item Видеохроника - название субкатегории
        \item События - место проведения события
        \item Журнал - главная тема архивного журнала
    \end{enumerate}
    \item Заголовок (текстовый) \& анонс текста
    \item Системная информация о статье - показываем дату создания статьи (формат: сегодня/вчера/конкретная дата + время), комментарии и рейтинг
\end{enumerate}

\paragraph{Теги}
Данный компонент показывает доступные субкатегории той рубрики или категории, в которой находится текущий пользователь. Внимание! Это не привычные тэги, а именно подкатеогрии, но заголовком является именно слово "Тэги".
            \begin{figure}[ht!]
            \centering
            \includegraphics[width=90mm]{tags.jpg}
            \caption{Выбор субкатегорий \label{fig:tags.jpg}}
            \end{figure}

\paragraph{Читайте также}
Данный компонент (которая обычно является частью других страниц) подгружает X статей (X управляется в админке). В элементе полоски показываем принадлежность статьи к категории или рубрики (например это новость или видеоролик), так как в статьях разных разделов могут быть одинаковые теги. Подгружаем только рубрики или категории одинаковых субкатегорий (например если мы в субкатегории "еда", то подгружаем только записи той же субкатегории)

\subsubsection{Главная} \label{main}
На главной странице отображаются несколько составляющих. Формат главной - две колонки (главная и правый сайдбар).

\begin{enumerate}
    \item Горячая статья - отображаем слайдером список из X горячих статей (X управляется в админке), если кол-во статей, которые редактор пометил как горячие превышает значение X, то отображаем X самых актуальных. Блок с горячей статьей состоит из большой картинки и слоя (не полностью закрывающего картинку) с текстовым заголовком, превью текста статьи и (см. пункт \ref{whiteline} на стр. \pageref{whiteline} ) элемента полосы. Для горячих статей в этом элементе выводим значение рубрики или категории.
    \item Место для рекламного баннера - для баннера нужно в админке использовать стандартный компонент Битрикса, позволяющий выбирать между сквозной и статичной рекламой. Сквозная позволяет выбирать между несколькими рекламными баннерами, выбирать кол-во просмотров и деактивировать баннеры, кол-во просмотров которых иссякло. Для статичных баннеров при загрузке рекламного изображения нужно указывать, в течении какого срока показывается изображение пользователю.
    \item \label{subsec:mostpopular}Самое популярное - отображаем статьи любых рубрик и категорий с наивысшим рейтингом.
    \item Цитата дня - отдельно редактируемый сниппет, который подргужает цитату, выбранную администратором, из списка цитат (список редактируется в админке).
    \item Блоги - подгружаем самые последние статьи из категории "блоги".
    \item Наши герои - Подгружаем последние актуальные статьи, относящиеся к рубрике "герои".
    \item Видеохроника - аналогично блогам и героям.
    \item Сайдбар - состоит из списка новостей, места для рекламного баннера, календаря и списка событий.
        \begin{enumerate}
            \item Список новостей состоит из последних записей в статьях категории "новости". При этом верстка самой актуальной статьи отличается от её предшественниц (см. изображение \ref{fig:main_right_sidebar.jpg} на стр. \pageref{fig:main_right_sidebar.jpg})
            \begin{figure}[ht!]
            \centering
            \includegraphics[width=90mm]{main_right_sidebar.jpg}
            \caption{Правый сайдбар \label{fig:main_right_sidebar.jpg}}
            \end{figure}
        \end{enumerate}
    \item События - элемент, позволяющий выбрать дату и подгрузить в блоке со списком событий перечень событий выбранного дня.
    \item Список событий конкретного дня (по умолчанию показываем самые актуальные, начиная с даты "/today"). Список делится на события журнала и события партнеров.
        \begin{enumerate}
            \item События журнала являются событиями (статьями), созданными администратором или соответствующе помеченным пользователем. Флажок событий - желтый
            \item События партнеров являются событиями (статьями), созданными также администраторами, но имеющие значение - "название партнера". Флажок событий - розовый.
        \end{enumerate}
    \item В конце сайдбара находится гиперссылка, ведущая на страницу со всеми событиями месяца. Данная страница представляет собой перечень анонсов событий.
\end{enumerate}

\paragraph{Футер сайта}
В нижней части сайта кроме текстового содержания 4 столбца. Наполнение уточняется, на данный момент это - первый столбец: разделы сайта (категория), второй и третий столбцы: специальные текстовые страницы, выводимые только в футере. Четвертый столбец - иконки соц.сетей. Также в футере расположено облако тегов, где вместо привычных тегов подгружаются субкатегории.

\subsubsection{Личный кабинет}
Описание данного подраздела включает в себя регистрацию, страницу логина и сам личный кабинет.

\paragraph{Регистрация}
Пользователь может зарегистрировать себя как через соц. сеть Facebook, так и полностью заполнив необходимые поля. 
При регистрации через Facebook за пользователя заполняются уже необходимые поля из данных его учетной записи в Facebook и создается новая запись в базе данных. При манульной регистрации пользователь должен заполнить все необходимые поля.
Перечень полей: имя, фамилия, логин, пароль, адрес эл. почты

\paragraph{Вход на сайт}
Пользователь вводит логин \& пароль после чего система перенаправляет его в личный кабинет.

\paragraph{Личный кабинет (ЛК)}
В ЛК пользователь может менять свои регистрационные данные - эл. почту, имя, фамилию.

%Референс - \ref{sec:testlabel} \pageref{sec:testlabel}

\subsubsection{Страница статьи}
На странице конкретной статьи представлена полноценная информация по статье. Данная информация состоит из:

\begin{enumerate}
    \item Хлебных крошек в виде визуальных элементов. 
    \item Заголовок (текст), иконка и кол-во комментариев, иконка и рейтинг пользователей.
    \item Слайдер с баннер(ы) статьи (визуальный заголовок статьи), на них присутствует также элемент полупрозрачной полосы (см. \ref{whiteline} ). На данном элементе отображаем автора и примечание к картинке. Если у статьи много баннеров, то слайдер показывает элементы для перехода между баннерами.
    \item Боди статьи выводит главное содержание статьи (текст, видео, дополнительные изображения)
    \item Подвал статьи - содержит автора и дату публикации
    \item Социальное - содержит модуль pluso.ru, позволяющий рекомендовать статью во всех доступных соц. сетях. Также присутствует собственный метод оценки (каждый пользователь может голосовать только один раз, незарегистрированные не могут голосовать вообще)
    \item Комментарии - каждый зарегистрированный пользователь может оставлять комментарии к статьям. Каждый комментарий состоит из: фотографии участника, имени участника, дату комментария, текст самого комментария. Также для каждого комментария возможно паралельное сохранение этого комментария на странице Facebook. При этом публикуется новая запись на стене пользователя в Facebook (см. рис. \ref{fig:facebook_comment.jpg} на стр. \pageref{fig:facebook_comment.jpg})
        \begin{figure}[ht!]
        \centering
        \includegraphics[width=90mm]{facebook_comment.jpg}
        \caption{Публикация комментара в Facebook \label{fig:facebook_comment.jpg}}
        \end{figure}
    \item Подгружаем блок "самое популярное" (см. пункт \ref{subsec:mostpopular} на странице \pageref{subsec:mostpopular} )
\end{enumerate}


\section{Бэкэнд}

\subsection{Популярные материалы}
Рейтинг статей рассчитывается из голосований пользователей. Строится обычная средневзвешенная оценка.
\begin{equation}
    \frac{1}{n} \sum\limits_{n=1}^{N} rating_n     
\end{equation}


\end{document}
